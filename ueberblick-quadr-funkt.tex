% arara: pdflatex: { shell: on }
% arara: pdflatex: { shell: on }
\documentclass{scrartcl}
\usepackage[ngerman]{babel}
\usepackage[T1]{fontenc}
\usepackage[utf8]{inputenc}
\usepackage{etoolbox}

\usepackage[oldstyle]{libertine}
\csdef{libertine@figurestyle}{LF}
\usepackage[libertine]{newtxmath}
\csdef{libertine@figurestyle}{OsF}
\usepackage{microtype}
\recalctypearea

\setcapindent{1.5em}
\setkomafont{caption}{\normalfont\small\sffamily}
\setkomafont{captionlabel}{\normalfont\small\sffamily\scshape}

\csdef{fps@figure}{htb}
\csdef{fps@table}{htb}

\usepackage{silence}
\WarningFilter{microtype}{I cannot find a protrusion list}

\usepackage{array,booktabs,hhline}
\newcolumntype{C}[1]{>{\centering\arraybackslash}p{#1}}

\usepackage{scrpage2}
\clearscrheadfoot
\pagestyle{scrheadings}
\chead{Seite \thepage}
\cfoot{\small C.\,Niederberger -- aktualisiert am \today}

\let\vary\relax
\usepackage{exsheets}

\usepackage{tkz-fct}
\usetkzobj{all}

\usepackage{xcolor}
\colorlet{grid}{black!10}

\usepackage[sort]{acro}
\acsetup{
  first-style=short ,
  short-format=\scshape ,
  list-short-format = ,
  page-ref = plain ,
  list-heading = section ,
  list-name = Begriffe
}

\DeclareAcronym{Funktion}{
  short = Funktion ,
  long  = {
    Eine eindeutige \begriff*{Zuordnung}.  Eine Zahl der
    \begriff*{Definitionsmenge} bekommt \emph{genau eine} Zahl der
    \begriff*{Wertemenge} zugeordnet, und zwar so, dass man eindeutig sagen
    kann, welche es ist.
  }
}
\DeclareAcronym{Funktionsgleichung}{
  short = Funktionsgleichung ,
  long  = {
    Die Vorschrift, nach der man den Wert aus der \begriff*{Wertemenge} bilden
    kann, der einer Zahl aus der \begriff*{Definitionsmenge} zugeordnet wird.
  }
}
\DeclareAcronym{Gerade}{
  short = Gerade ,
  long  = {
    Eine lineare \begriff*{Funktion}.  Ihre Wertepaare liegen im Schaubild
    alle auf einer geraden Linie.
  } ,
  alt   = Geraden
}
\DeclareAcronym{Steigung}{
  short = Steigung ,
  long  = {
    Ein Maß dafür, wie steil der \begriff*{Graph} einer \siehe\aca*{Gerade} im
    Koordinatensystem liegt.
  }
}
\DeclareAcronym{y-Achsenabschnitt}{
  short = $y$-Achsenabschnitt ,
  long  = {
    Die Stelle, an der der \begriff*{Graph} einer \siehe\aca*{Gerade} die
    $y$-Achse schneidet.
  }
}
\DeclareAcronym{Parabel}{
  short = Parabel ,
  long  = {
    Der \begriff*{Graph} einer quadratischen Funktion.  Oft auch als Synonym
    für \emph{quadratische Funktion} verwendet.
  } ,
  alt   = Parabeln
}
\DeclareAcronym{Graph}{
  short = Graph ,
  long  = {
    Die graphische Darstellung einer \begriff*{Funktion} im
    Koordinatensystem.
  } ,
  alt   = Graphen
}
\DeclareAcronym{Zuordnung}{
  short = Zuordnung ,
  long  = {
    Eine Gruppe von \siehe\aca*{Tupel}, die nach einer bestimmten Regel
    gebildet werden.
  }
}
\DeclareAcronym{Tupel}{
  short = Tupel ,
  long  = {
    Wertepaare oder Zahlenpaare wie $\punkt*(1,2)$.  In der Regel werden sie
    als Koordinaten eines Punktes im Koordinatensystem interpretiert.
  } ,
  alt   = Tupeln
}
\DeclareAcronym{Definitionsmenge}{
  short = Definitionsmenge ,
  long  = {
    Die Menge der Zahlen, denen Zahlen aus einer zweiten Menge, der
    \begriff*{Wertemenge} zugeordnet werden.
  }
}
\DeclareAcronym{Wertemenge}{
  short = Wertemenge ,
  long  = {
    Die Menge der Zahlen, die denen einer ersten Menge, der
    \begriff*{Definitionsmenge} zugeordnet werden.
  }
}
\DeclareAcronym{Wertetabelle}{
  short = Wertetabelle ,
  long  = {
    Eine Tabelle, in der einige \begriff*{Tupel} einer \begriff*{Zuordnung}
    aufgelistet sind.
  }
}

\makeatletter
\newcommand*\menge[1]{\mathbb{#1}}
\newcommand*\punkt{\@ifstar{\@punkt}{\@Punkt}}
\newcommand*\@Punkt[1]{#1\@punkt}
\def\@punkt(#1,#2){(#1\vert#2)}

\newcommand*\siehe{$\rightarrow$~}
\newcommand*\begriff{\siehe\ac}
\makeatother

\usepackage[framemethod=tikz]{mdframed}
\xdefinecolor{beispiel}{rgb}{0.02,0.04,0.48}
\newcounter{beispiel}
\newmdenv[
  linecolor  = beispiel ,
  rightline  = false ,
  leftline   = false ,
  frametitle = {\refstepcounter{beispiel}\textbf{Beispiel~\thebeispiel.}}
]{beispiel}

\usepackage{hyperref}
\hypersetup{
  colorlinks ,
  title = Quadratische Funktionen
}

\begin{document}

\begin{center}
  \Huge\sffamily Quadratische Funktionen
\end{center}

\tableofcontents

\section{Funktionen}
\subsection{Allgemeines -- Begriffsklärung}

Unter einer \begriff{Funktion} versteht man eine \emph{eindeutige}
\begriff{Zuordnung}.  Zuordung bedeutet, dass einer Zahl aus einer Gruppe
(oder besser: Menge) eine Zahl aus einer zweiten Gruppe zugeordnet bekommt, so
dass Zahlenpaare entstehen.  Die erste Gruppe nennt man
\begriff{Definitionsmenge} $\menge{D}$ und eine unbekannte Zahl dieser Gruppe
bekommt das Symbol $x$.  Die zweite Gruppe nennt man \begriff{Wertemenge}
$\menge{W}$, eine Zahl dieser Gruppe bekommt das Symbol $y$ oder $f(x)$.  Eine
Tabelle, in der einige der Zahlenpaare (man nennt sie auch \begriff{Tupel})
aufgelistet sind, nennt man \begriff{Wertetabelle}.  Ein Beispiel dafür ist
Tabelle~\ref{tab:wertetabelle}.

\begin{table}
  \centering
  \caption{Beispiel für eine Wertetabelle.}
  \label{tab:wertetabelle}
  \begin{tabular}{>{$}l<{$}||*6{>{$}C{1cm}<{$}|}>{$}C{1cm}<{$}}
     x & -3 & -2 & -1 &  0 &  1 &  2 &  3  \\
    \hline
     y &  3 &  0 & -1 &  0 &  3 &  8 & 15
  \end{tabular}
\end{table}

Diese Zahlenpaare kann man in einem Koordinatensystem einzeichen.  Für die
Werte aus Tabelle~\ref{tab:wertetabelle} wurde das in
Abbildung~\ref{fig:wertepaare} gemacht.

\begin{figure}
  \centering
  \begin{tikzpicture}[scale=.5]
    \tkzInit[xmin=-5,xmax=5,ymin=-2,ymax=9]
    \tkzGrid
    \tkzAxeXY
    \tkzDefPoints{-3/3/a,-2/0/b,-1/-1/c,0/0/d,1/3/e,2/8/f}
    \tkzDrawPoints[color=red,shape=cross out](a,b,c,d,e,f)
  \end{tikzpicture}
  \caption{Die Wertepaare aus Tabelle~\ref{tab:wertetabelle} in einem
    Koordinatensystem}.
  \label{fig:wertepaare}
\end{figure}

Bei einer \begriff{Funktion} liegen alle Punkte dann in der Regel auf einer
Linie, dem \siehe\aca{Graph} der \begriff{Funktion}.  In
Abbildung~\ref{fig:wertepaare-linie} wurde das für die Punkte aus
Tabelle~\ref{tab:wertetabelle} gemacht.

\begin{figure}
  \centering
  \begin{tikzpicture}[scale=.5]
    \tkzInit[xmin=-5,xmax=5,ymin=-2,ymax=9]
    \tkzGrid
    \tkzAxeXY
    \tkzDefPoints{-3/3/a,-2/0/b,-1/-1/c,0/0/d,1/3/e,2/8/f}
    \tkzDrawPoints[color=red,shape=cross out](a,b,c,d,e,f)
    \tkzFct[color=red]{(x+1)**2-1}
  \end{tikzpicture}
  \caption{Die Wertepaare aus Tabelle~\ref{tab:wertetabelle} liegen alle auf
    einer Linie.}
  \label{fig:wertepaare-linie}
\end{figure}

\subsection{Lineare Funktionen -- Geraden}

Lineare Funktionen sind eine spezielle Sorte von Funktionen, bei denen alle
Wertepaare im Koordinatensystem auf einer geraden Linie liegen.  Man nennt sie
auch \siehe\aca{Gerade}.  Lineare Funktionen werden nach einer bestimmten
Vorschrift gebildet.  Ihre allgemeine Form lautet
\begin{equation}
  \label{eq:lineare-funktion}
  y = m\cdot x + c \quad.
\end{equation}
Man nennt $m$ die \begriff{Steigung} und $c$ den \begriff{y-Achsenabschnitt}.

\begin{beispiel}
  Für die Gerade $y=2x+1$ kann man folgende \begriff{Wertetabelle} aufstellen:
  \begin{center}
    \begin{tabular}{>{$}l<{$}||*6{>{$}C{1cm}<{$}|}>{$}C{1cm}<{$}}
     x & -3 & -2 & -1 &  0 &  1 &  2 &  3  \\
    \hline
     y & -5 & -3 & -1 &  1 &  3 &  5 &  7
  \end{tabular}
  \end{center}
  Ihr \begriff{Graph} sieht dann so aus:
  \begin{center}
    \begin{tikzpicture}[scale=.5]
      \tkzInit[xmin=-4,xmax=4,ymin=-6,ymax=6]
      \tkzGrid
      \tkzAxeXY
      \tkzDefPoints{-3/-5/a,-2/-3/b,-1/-1/c,0/1/d,1/3/e,2/5/f}
      \tkzDrawPoints[color=red,shape=cross out](a,b,c,d,e,f)
      \tkzFct[color=red]{2*x+1}
    \end{tikzpicture}
  \end{center}
\end{beispiel}


\subsection{Quadratische Funktionen -- Parabeln}

\printacronyms

\end{document}
