% arara: pdflatex
% arara: pdflatex
\documentclass{scrartcl}

\usepackage[T1]{fontenc}
\usepackage[utf8]{inputenc}
\usepackage[supstfm=libertinesups]{superiors}
\usepackage{libertine}
\usepackage{microtype}

\usepackage[greek,ngerman]{babel}
\usepackage{scrpage2}
\clearscrheadfoot
\pagestyle{scrheadings}
\chead{Seite \thepage}
\cfoot{\small C.\,Niederberger -- aktualisiert am \today}

\usepackage{upgreek}
\usepackage{chemmacros}
\chemsetup{
  option/language         = german,
  chemformula/name-format = \centering ,
%  phases/pos              = sub
}

\DeclareChemReaction[star]{gthreactions}{gather}
\newcommand*\rctref[1]{\{\ref{#1}\}}

\usepackage{chemnum}
\cmpdsetup{cmpd-counter=Alph}

\usepackage{siunitx}
\sisetup{
  detect-all ,
%  per-mode = fraction ,
  per-mode = symbol ,
  fixed-exponent = 0 ,
%  scientific-notation = true ,
  exponent-product = \cdot ,
  list-final-separator = { und } ,
  list-pair-separator = { und } ,
  range-phrase = { bis }
}
\DeclareSIUnit{\graddh}{grad\,dH}

\usepackage{chemfig}
\setatomsep{1.8em}
\setcrambond{2.5pt}{1pt}{2pt}
\definesubmol{C}{{\vphantom{CH_3}C}H_}
\definesubmol{m}{C(-[2]H)(-[6]H)}

\usepackage[load-headings]{exsheets}
\SetupExSheets{
  totoc ,
  headings = block-rev
}

\usepackage{fnpct}

\usepackage{booktabs}
\usepackage{collcell}
\newcolumntype{C}{>{\collectcell\ch}l<{\endcollectcell}}

\usepackage{tikz}
\colorlet{angcolor}{green!50!black}
\colorlet{lencolor}{red!50!black}

\usepackage{multicol}
\usepackage[colorlinks]{hyperref}

\begin{document}

\begin{center}
  \Huge\sffamily Nomenklatur von Kohlenwasserstoffen
\end{center}

\addsec{Übungen}

\begin{question}[name=Unverzweigte Alkane]
  Geben Sie die Strukturformeln von Methan bis Decan der homologen Reihe an.
\end{question}
\begin{solution}[name=Unverzweigte Alkane]
  \begin{tabular}[c]{lc}
    Methan & \chemfig{H-!m-H} \\
    Ethan  & \chemfig{H-!m-!m-H} \\
    Propan & \chemfig{H-!m-!m-!m-H} \\
    Butan  & \chemfig{H-!m-!m-!m-!m-H} \\
    Pentan & \chemfig{H-!m-!m-!m-!m-!m-H} \\
    Hexan  & \chemfig{H-!m-!m-!m-!m-!m-!m-H} \\
    Heptan & \chemfig{H-!m-!m-!m-!m-!m-!m-!m-H} \\
    Octan  & \chemfig{H-!m-!m-!m-!m-!m-!m-!m-!m-H} \\
    Nonan  & \chemfig{H-!m-!m-!m-!m-!m-!m-!m-!m-!m-H} \\
    Decan  & \chemfig{H-!m-!m-!m-!m-!m-!m-!m-!m-!m-!m-H}
  \end{tabular}
\end{solution}

\begin{question}[name=Einfach verweigte Alkane I]
  Geben Sie die Strukturformeln folgender Alkane an.
  \begin{tasks}(4)
    \task \iupac{2\-Methyl\|butan}
    \task \iupac{3\-Ethyl\|hexan}
    \task \iupac{4\-Propyl\|heptan}
    \task \iupac{5\-Butyl\|nonan}
  \end{tasks}
\end{question}
\begin{solution}[name=Einfach verweigte Alkane I]
  \begin{tasks}
    \task \chemfig{!C3-!C{}(-[6]!C3)-!C2-!C3}
    \task \chemfig{!C3-!C2-!C{}(-[6]!C2-!C3)-!C2-!C2-!C3}
    \task \chemfig{!C3-!C2-!C2-!C{}(-[6]!C2-!C2-!C3)-!C2-!C2-!C3}
    \task \chemfig{!C3-!C2-!C2-!C2-!C{}(-[6]!C2-!C2-!C2-!C3)-!C2-!C2-!C2-!C3}
  \end{tasks}
\end{solution}

\begin{question}[name=Einfach verzweigte Alkane II]
  Benennen Sie folgende Moleküle.
  \begin{tasks}
    \task \chemfig{!C3-!C2-!C2-!C{}(-[6]!C3)-!C3}
    \task \chemfig{!C3-!C2-!C{}(-[6]!C3)-!C2-!C2-!C3}
    \task \chemfig{!C3-!C2-!C2-!C{}(-[6]!C2-!C3)-!C2-!C2-!C2-!C3}
    \task \chemfig{!C3-!C2-!C2-!C2-!C{}(-[6]!C2-!C2-!C3)-!C2-!C2-!C3}
  \end{tasks}
\end{question}
\begin{solution}[name=Einfach verzweigte Alkane II]
  \begin{tasks}(4)
    \task \iupac{2\-Methyl\|pentan}
    \task \iupac{3\-Methyl\|hexan}
    \task \iupac{4\-Ethyl\|octan}
    \task \iupac{4\-Propyl\|octan}
  \end{tasks}
\end{solution}

\begin{question}[name=Mehrfach verzweigte Alkane I]
  Geben Sie die Strukturformeln folgender Alkane an.
  \begin{tasks}(2)
    \task \iupac{2,2\-Di\|methyl\|butan}
    \task \iupac{2,4\-Di\|methyl\|hexan}
    \task \iupac{3\-Ethyl\-3\-methyl\|pentan}
    \task \iupac{3,3,5\-Tri\|methyl\|heptan}
    \task \iupac{5\-Butyl\-4,4\-di\|ethyl\-2,7\-di\|methyl\|nonan}
  \end{tasks}
\end{question}
\begin{solution}[name=Mehrfach verzweigte Alkane I]
  \begin{tasks}
    \task\chemfig{!C3-C(-[6]!C3)(-[2]!C3)-!C2-!C3}
    \task\chemfig{!C3-!C{}(-[6]!C3)-!C2-!C{}(-[6]!C3)-!C2-!C3}
    \task\chemfig{!C3-!C2-C(-[6]!C3)(-[2]!C2-!C3)-!C2-!C3}
    \task\chemfig{!C3-!C2-C(-[6]!C3)(-[2]!C3)-!C2-!C{}(-[6]!C3)-!C2-!C3}
    \task
      \chemfig{
        !C3-!C{}(-[6]!C3)
        -!C2-C(-[6]!C2-[6]!C3)(-[2]!C2-!C3)
        -!C{}(-[6]!C2-[6]!C2-[6]!C3)
        -!C2-!C{}(-[6]!C3)-!C2-!C3
      }
  \end{tasks}
\end{solution}

\begin{question}[name=Mehrfach verzweigte Alkane II]
  Benennen Sie folgende Moleküle.
  \begin{tasks}
    \task \chemfig{!C3-!C{}(-[6]!C3)-!C{}(-[6]!C3)-!C2-!C2-!C3}
    \task \chemfig{!C3-C(-[6]!C3)(-[2]!C3)-!C2-!C3}
    \task \chemfig{!C3-C(-[6]!C3)(-[2]!C3)-!C2-!C{}(-[6]!C3)-!C3}
    \task \chemfig{!C3-!C2-!C{}(-[6]!C2-!C3)-!C{}(-[2]!C3)-!C2-!C{}(-[6]!C3)-!C3}
    \task
      \chemfig{
        !C3-!C2-C(-[2]!C3)(-[6]!C2-[6]!C3)
        -!C2-C(-[6]!C2-[6]!C2-!C3)(-[2]!C2-!C2-!C3)
        -!C2-!C{}(-[6]!C3)
        -!C3
      }
  \end{tasks}
\end{question}
\begin{solution}[name=Mehrfach verzweigte Alkane II]
  \begin{tasks}(2)
    \task \iupac{2,3\-Di\|methyl\|hexan}
    \task \iupac{2,2\-Di\|methyl\|butan}
    \task \iupac{2,2,4\-Tri\|methyl\|pentan}
    \task \iupac{5\-Ethyl\-2,4\-di\|methyl\|heptan}
    \task \iupac{6\-Ethyl\-2,6\-di\|methyl\-4,4\-di\|propyl\|octan}
  \end{tasks}
\end{solution}

\begin{question}[name=Alkene I]
  Geben Sie die Strukturformeln folgender Alkene an.
  \begin{tasks}(3)
    \task \iupac{Propen}
    \task \iupac{2\-Penten}
    \task \iupac{1,3\-Buta\|di\|en}
    \task \iupac{2,4\-Hepta\|di\|en}
    \task \iupac{1,3,5\-Hepta\|tri\|en}
  \end{tasks}
\end{question}
\begin{solution}[name=Alkene I]
  \begin{tasks}(2)
    \task \ch{CH2=CH-CH3}
    \task \ch{CH3-CH2-CH=CH-CH3}
    \task \ch{CH3-CH=CH-CH=CH-CH2-CH3}
    \task \ch{CH3-CH=CH-CH=CH-CH=CH2}
  \end{tasks}
\end{solution}

\begin{question}[name=Alkene II]
  Benennen Sie folgende Moleküle.
  \begin{tasks}(2)
    \task \ch{CH2=CH2}
    \task \ch{CH3-CH=CH-CH3}
    \task \ch{CH2=CH-CH=CH-CH3}
    \task \ch{CH2=CH-CH2-CH=CH2}
    \task \ch{CH3=C=CH-CH2-CH=CH2}
  \end{tasks}
\end{question}
\begin{solution}[name=Alkene II]
  \begin{tasks}(3)
    \task \iupac{Ethen}
    \task \iupac{2\-Buten}
    \task \iupac{1,3\-Penta\|di\|en}
    \task \iupac{1,4\-Penta\|di\|en}
    \task \iupac{1,2,5\-Hexa\|tri\|en}
  \end{tasks}
\end{solution}

\begin{question}[name=Kohlenwasserstoffe I]
  Geben Sie die Strukturformeln folgender Moleküle an.
  \begin{tasks}
    \task \iupac{3,3\-Di\|methyl\|pent\-1\-en}
    \task \iupac{7\-Ethyl\-2,3\-di\|methyl\-4\-propyl\-3,5\-nona\|di\|en}
    \task \iupac{5\-Ethyl\-2,4\-di\|methyl\-6\-propyl\-2,4,7\-nona\|tri\|en}
  \end{tasks}
\end{question}
\begin{solution}[name=Kohlenwasserstoffe I]
  \begin{tasks}(2)
    \task \chemfig{[:30]=-[:-30](-[:-70])(-[:-110])--[:-30]}
    \task
      \chemfig{
        [:30]-(-[2])
        -[:-30](-[6])
        =(-[2]-[:30]-[:-30])
        -[:-30]=-[:-30](-[6]-[:-30])
        --[:-30]
      }
    \task
      \chemfig{
        [:30]-(-[2])
        =[:-30]-(-[2])
        =[:-30](-[6]-[:-30])
        -(-[2]-[:30]-[:-30])
        -[:-30]=-[:-30]
      }
  \end{tasks}
\end{solution}

\begin{question}[name=Kohlenwasserstoffe II]
  Benennen Sie folgende Moleküle.
  \begin{tasks}(2)
    \task \chemfig{[:30]=-[:-30]-(-[2]-[:30])=[:-30]}
    \task \chemfig{[:30]-(-[:70])(-[:110])-[:-30]=}
    \task
      \chemfig{
        [:30]=-[:-30](-[6]-[::-60]-[::-60])
        -(-[2]-[::60])
        -[:-30]=(-[2])
        -[:-30](-[6])=
      }
  \end{tasks}
\end{question}
\begin{solution}[name=Kohlenwasserstoffe II]
  \begin{tasks}(2)
    \task \iupac{2\-Ethyl\-1,4\-penta\|di\|en}
    \task \iupac{3,3\-Di\|methyl\-1\-but\|en}
    \task \rlap{\iupac{5\-Ethyl\-2,3\-di\|methyl\-6\-propyl\-1,3,7\-octa\|tri\|en}}
  \end{tasks}
\end{solution}

\newpage
\addsec{Lösungen}
\printsolutions

\end{document}
