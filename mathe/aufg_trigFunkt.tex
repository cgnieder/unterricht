% arara: pdflatex: { shell: on }
\documentclass{scrartcl}
\usepackage[utf8]{inputenc}
\usepackage[ngerman]{babel}
\usepackage{libertine}
\usepackage[libertine]{newtxmath}
% \show\vary
% \mathchar"432
\let\vary\relax
\usepackage{exsheets}

\usepackage{tkz-fct}
\usetkzobj{all}

\usepackage{xcolor}
\colorlet{grid}{black!10}

\usepackage{scrpage2}
\clearscrheadfoot
\pagestyle{scrheadings}
\chead{Seite \thepage}
\cfoot{\small C.\,Niederberger -- aktualisiert am \today}

\begin{document}

\begin{center}
  \Huge\sffamily Trigonometrische Funktionen
\end{center}

\addsec{Aufgaben}

\begin{question}
  Bestimme die Funktionsgleichungen der Sinusfunktionen im folgenden Schaubild:

\begin{tikzpicture}[scale=.7]
  \tkzInit[xmin=-1,xmax=16,ymin=-4,ymax=6]
  \tkzGrid[sub,subxstep=.5,subystep=.5,color=grid]
  \tkzAxeY[font=\footnotesize]
  \tkzAxeX[trig=2,font=\footnotesize]
  \tkzClip
  \foreach\v in {.5,1,...,5}
    {\tkzVLine[dotted]{\v*\FPpi}}
  \tkzFct[color=red,domain=0:16]{2*sin(x-pi)+3}
  \tkzFct[color=green,domain=0:16]{1.5*sin(2*(x-pi/2))-1}
  \tkzFct[color=blue,domain=0:16]{3*sin(.5*x)+1.5}
  \tkzFct[color=brown,domain=0:16]{sin(x/3)+4}
\end{tikzpicture}
\end{question}
\begin{solution}
  Diese Gleichungen stellen nur \emph{eine} mögliche Lösung dar.
  \begin{tasks}[label-width=3em,item-indent=4.3333em](2)
    \task[Rot:]
      $ f(x) = 2\sin(x-\pi)+3 \vphantom{\big(} $
    \task[Grün:]
      $
        f(x) =
        \frac{3}{2}\sin\bigl(2\left(x-\frac{\pi}{2}\right)\bigr)-1
      $
    \task[Blau:]
      $ f(x) = 3\sin\bigl(\frac{1}{2}x\bigr)+\frac{3}{2} $
    \task[Braun:]
      $ f(x) = \sin\bigl(\frac{x}{3}\bigl)+4 $
  \end{tasks}
\end{solution}

\begin{question}
  Bestimme die Funktionsgleichungen der Sinusfunktionen im folgenden Schaubild:

\begin{tikzpicture}[scale=.7]
  \tkzInit[xmin=-1,xmax=16,ymin=-4,ymax=6]
  \tkzGrid[sub,subxstep=.5,subystep=.5,color=grid]
  \tkzAxeXY[font=\footnotesize]
  \tkzClip
  \tkzFct[color=red,domain=0:16]{sin(.5*pi*x)+3}
  \tkzFct[color=green,domain=0:16]{sin(pi*(x+.5))-1}
  \tkzFct[color=blue,domain=0:16]{.5*sin(.25*pi*(x-1))-3}
  \tkzFct[color=brown,domain=0:16]{4*sin(pi*x/5)+2}
\end{tikzpicture}
\end{question}
\begin{solution}
  Diese Gleichungen stellen nur \emph{eine} mögliche Lösung dar.
  \begin{tasks}[label-width=3em,item-indent=4.3333em](2)
    \task[Rot:]
      $ f(x) = \sin\bigl(\frac{\pi x}{2}\bigr)+3 $
    \task[Grün:]
      $ f(x) = \sin\bigl(\pi\left(x+\frac{1}{2}\right)\bigr)-1 $
    \task[Blau:]
      $ f(x) = \frac{1}{2}\sin\bigl(\frac{\pi}{4}(x+1)\bigr) -3  $
    \task[Braun:]
      $ f(x) = 4\sin\bigl(\frac{\pi x}{5}\bigl) +2 $
  \end{tasks}
\end{solution}

\begin{question}
  Zeichne folgende Funktionen in ein Schaubild im Intervall $[0,5\pi]$:
  \begin{tasks}(2)
    \task $f(x)=\sin(3x)+1 \vphantom{\big(}$
    \task $f(x)=-2\sin\bigl(\frac{1}{2}\left(x+\frac{\pi}{2}\right)\bigr)$
    \task $f(x)=3\sin\bigl(\frac{3}{2}x\bigr)-2$
    \task $f(x)=\frac{5}{2}\sin\bigl(3x-\frac{3\pi}{2}\bigr)+2$
  \end{tasks}
\end{question}

\begin{question}
  Zeichne folgende Funktionen in ein Schaubild im Intervall $[0,10]$:
  \begin{tasks}(2)
    \task $f(x)=\sin(\pi x)-2 \vphantom{\big(}$
    \task $f(x)=2\sin\bigl(\frac{\pi}{2}(x+1)\bigr)$
    \task $f(x)=4\sin\bigl(\frac{3\pi}{2}x\bigr)+1$
    \task $f(x)=0{,}5\sin(0{,}5\pi x)-0{,}5 \vphantom{\big(}$
  \end{tasks}
\end{question}

\addsec{Lösungen}
\printsolutions

\end{document}

