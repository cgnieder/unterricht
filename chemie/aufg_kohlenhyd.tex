% arara: pdflatex
% arara: pdflatex
\documentclass{scrartcl}

\usepackage[T1]{fontenc}
\usepackage[utf8]{inputenc}
\usepackage[supstfm=libertinesups]{superiors}
\usepackage{libertine}
\usepackage{microtype}

\usepackage[greek,ngerman]{babel}
\usepackage{scrpage2}
\clearscrheadfoot
\pagestyle{scrheadings}
\chead{Seite \thepage}
\cfoot{\small C.\,Niederberger -- aktualisiert am \today}

\usepackage{upgreek}
\usepackage{chemmacros}
\chemsetup{
  option/language         = german,
  chemformula/name-format = \centering ,
%  phases/pos              = sub
}

\DeclareChemReaction[star]{gthreactions}{gather}
\newcommand*\rctref[1]{\{\ref{#1}\}}

\usepackage{chemnum}
\cmpdsetup{cmpd-counter=Alph}

\usepackage{siunitx}
\sisetup{
  detect-all ,
%  per-mode = fraction ,
  per-mode = symbol ,
  fixed-exponent = 0 ,
%  scientific-notation = true ,
  exponent-product = \cdot ,
  list-final-separator = { und } ,
  list-pair-separator = { und } ,
  range-phrase = { bis }
}
\DeclareSIUnit{\graddh}{grad\,dH}

\usepackage{chemfig}
\setatomsep{1.8em}
\setcrambond{2.5pt}{1pt}{2pt}

\usepackage[load-headings]{exsheets}
\SetupExSheets{
  totoc ,
  headings = block-rev
}

\usepackage{fnpct}

\usepackage{booktabs}
\usepackage{collcell}
\newcolumntype{C}{>{\collectcell\ch}l<{\endcollectcell}}

\usepackage{tikz}
\colorlet{angcolor}{green!50!black}
\colorlet{lencolor}{red!50!black}

\usepackage{multicol}
\usepackage[colorlinks]{hyperref}

\begin{document}

\begin{center}
  \Huge\sffamily Kohlenhydrate
\end{center}

\addsec{Übungen}
\begin{question}[name=Physikalische Eigenschaften]
  Zeichnen Sie die Strukturformeln der folgenden Verbindungen, ordnen Sie sie
  nach Siedepunkten bzw. Wasserlöslichkeit und begründen Sie:
  \begin{tasks}
    \task Ethan, Ethanol, Dimethylether, Ethanal
    \task Methanol, Glykol (Ethandiol), Glycerin (Propantriol), Sorbitol
      (Hexanhexol)
    \task Ethansäure, Methansäure"=Methyl"=Ester
  \end{tasks}
\end{question}
\begin{solution}[name=Physikalische Eigenschaften]
  Steigende Siedepunkte\slash Wasserlöslichkeit:
  \begin{tasks}
    \task
      \chemfig{H_3|C-CH3} \qquad
      \chemfig{-[:30]O-[:-30]} \qquad
      \chemfig{-[:30]=[:-30]O} \qquad
      \chemfig{-[:30]-[:-30]OH}
    \task
      \chemfig{H_3|C-OH} \qquad
      \chemfig{HO-[:-30]-[:30]-[:-30]OH} \qquad
      \chemfig{HO-[:-30]-[:30](-[2]OH)-[:-30]-[:30]OH}
      
      \chemfig{
        HO-[:-30]-[:30](-[2]OH)
        -[:-30](-[6]OH)
        -[:30](-[2]OH)
        -[:-30](-[6]OH)
        -[:30]-[:-30]OH
      }
    \task
      \chemfig{-[:30](=[2]O)-[:-30]O-[:30]} \qquad
      \chemfig{-[:30](=[2]O)-[:-30]OH}
  \end{tasks}
\end{solution}

\begin{question}[
  name={Stereoisomerie, R\slash S"=Nomenklatur und
    \iupac{\L}/\iupac{\D}"=Reihen}
  ]
  Zeichnen Sie alle möglichen Stereoisomere der folgenden Verbindungen,
  benennen Sie sie nach der R\slash S"=Nomenklatur und ordnen Sie sie jeweils
  der \iupac{\D}- oder \iupac{\L}"=Reihe zu:
  \begin{tasks}(2)
    \task \iupac{2,3\-Di\|chlor\|pentan}
    \task \iupac{1,2,3\-Tri\|chlor\|butan}
    \task \iupac{2,3,4\-Tri\|hydroxy\|butanal}
  \end{tasks}
\end{question}
\begin{solution}[name=Stereoisomerie]
  \begin{tasks}
    \task \iupac{\D}"=Reihe:
    
      \chemfig{-[:30](<[2]Cl)-[:-30](<[6]Cl)-[:30]-[:-30]} \qquad
      \chemfig{-[:30](<:[2]Cl)-[:-30](<[6]Cl)-[:30]-[:-30]}

      \iupac{\cip{2R,3R}\-2,3\-Di\|chlor\|pentan},
      \iupac{\cip{2S,3R}\-2,3-Di\|chlor\|pentan}

      \iupac{\L}"=Reihe:
      
      \chemfig{-[:30](<[2]Cl)-[:-30](<:[6]Cl)-[:30]-[:-30]} \qquad
      \chemfig{-[:30](<:[2]Cl)-[:-30](<:[6]Cl)-[:30]-[:-30]}

      \iupac{\cip{2R,3S}\-2,3\-Di\|chlor\|pentan},
      \iupac{\cip{2S,3S}\-2,3\-Di\|chlor\|pentan}
    \task \iupac{\D}"=Reihe:
    
      \chemfig{Cl-[:-30]-[:30](<[2]Cl)-[:-30](<[6]Cl)-[:30]} \qquad
      \chemfig{Cl-[:-30]-[:30](<:[2]Cl)-[:-30](<[6]Cl)-[:30]}

      \iupac{\cip{2R,3R}\-1,2,3\-Tri\|chlor\|butan},
      \iupac{\cip{2S,3R}\-1,2,3\-Tri\|chlor\|butan}

      \iupac{\L}"=Reihe:
      
      \chemfig{Cl-[:-30]-[:30](<[2]Cl)-[:-30](<:[6]Cl)-[:30]} \qquad
      \chemfig{Cl-[:-30]-[:30](<:[2]Cl)-[:-30](<:[6]Cl)-[:30]}

      \iupac{\cip{2R,3S}\-1,2,3\-Tri\|chlor\|butan},
      \iupac{\cip{2S,3S}\-1,2,3\-Tri\|chlor\|butan}
    \task \iupac{\D}"=Reihe:
    
      \chemfig{HO-[:-30]-[:30](<[2]OH)-[:-30](<[6]OH)-[:30]=[:-30]O} \qquad
      \chemfig{HO-[:-30]-[:30](<[2]OH)-[:-30](<:[6]OH)-[:30]=[:-30]O}
      
     \iupac{\cip{2S,3R}\-2,3,4\-Tri\|hydroxy\|butanal} (\iupac{\D\-Threose}),
     \iupac{\cip{2R,3R}\-2,3,4\-Tri\|hydroxy\|butanal} (\iupac{\D\-Erythrose})

     \iupac{\L}"=Reihe:
     
     \chemfig{HO-[:-30]-[:30](<:[2]OH)-[:-30](<[6]OH)-[:30]=[:-30]O} \qquad
     \chemfig{HO-[:-30]-[:30](<:[2]OH)-[:-30](<:[6]OH)-[:30]=[:-30]O}

     \iupac{\cip{2S,3S}\-2,3,4\-Tri\|hydroxy\|butanal} (\iupac{\L\-Erythrose}),
     \iupac{\cip{2R,3S}\-2,3,4\-Tri\|hydroxy\|butanal} (\iupac{\L\-Threose})
  \end{tasks}
\end{solution}

\begin{question}[name=Polarimetrie]
  Erklären Sie mit Hilfe einer beschrifteten Skizze den Aufbau und die
  Funktionsweise eines einfachen Polarimeters.
\end{question}
\begin{solution}[name=Polarimetrie]
  In Abboldung~\ref{fig:polarimeter} sehen Sie die schematische Darstellung
  eines Polarimeters.  Am ersten optischen Gitter wird Licht einer Lichtquelle
  linear polarisiert, d.h., nur Licht einer Schwingungsebene wird
  hindurchgelassen.  Danach passiert das Licht eine Probenküvette.  Abhängig
  von der Konzentration der Probenlösung in der Küvette (Standard
  \SI{1}{\Molar}) und der Länge {\color{lencolor}$d$} der Küvette wird die
  Schwingungsebene um einen Winkel {\color{angcolor}$\alpha$} gedreht, wenn
  die Probe optisch aktiv ist.
        
  Das zweite optische Gitter kann gedreht werden, um den Drehwinkel
  {\color{angcolor}$\alpha$} festzustellen.  Bei einer Einstellung von
  {\color{angcolor}$\alpha$} wird die Lichtintensität dahinter maximal.
  Der Drehwinkel {\color{angcolor}$\alpha$}, umgerechnet auf eine
  Standard"=Konzentration und eine Standard"=Küvettenlänge (\SI{1}{cm}), ist
  charakteristisch für die optisch aktive Substanz.
\end{solution}

\begin{question}[name=Halbacetalbildung]
  Geben Sie den Reaktionsmechanismus der Halbacetalbildung von
  \iupac{5\-Hydroxy\|hexanal} (\cmpd{hyrdoxyhexanal}) und Ethanol
  (\cmpd{ethanol}) in saurer Lösung an.  Welches Produkt bildet sich, wenn man
  \cmpd{hyrdoxyhexanal} und \cmpd{ethanol} im Verhältnis $1:2$ reagieren
  lässt?
\end{question}
\begin{solution}[name=Halbacetalbildung]
  Mechanismus der Halbacetalbildung:
  \begin{center}
    \schemestart
      \chemfig{R-\lewis{26,O}H}
      \+
      \chemfig{R'-(=[:60]@{Oald1}\lewis{02,O})-[:-60]R''}
      \arrow{<=>[\chemfig{@{Prt}H\pch}]}
      \chemfig{R-@{Oalk1}\lewis{26,O}H}
      \+
      \chemfig{
        R'-@{Cald}
        (=[@{Odb}:60]@{Oald2}\chemabove{\lewis{2,O}}{\scrp}-H)
        -[:-60]R''
      }
      \arrow{<=>}[-90]
      \chemfig{
        R-@{Oet}\chembelow[3pt]{\lewis{6,O}}{\scrp}
        (-[@{Heb}2]H)-(-[:60]OH)(-[:-60]R'')-R'
      }
      \arrow{<=>[$-\ch{H+}$]}[180]
      \chemfig{R-\lewis{26,O}-(-[:60]OH)(-[:-60]R'')-R'}
    \schemestop
  \end{center}
  \chemmove[red,shorten <=3pt,shorten >=3pt,->]{
    \draw (Oald1) ..controls +(0:4mm) and +(90:8mm) .. (Prt) ;
    \draw (Oalk1) ..controls +(-90:8mm) and +(-120:4mm) .. (Cald) ;
    \draw (Heb) ..controls +(180:4mm) and +(135:4mm) .. (Oet) ;
    \draw (Odb) ..controls +(-30:4mm) and +(-30:8mm) .. (Oald2) ;
  }%
  Es enstehen ein ringförmiges Halbacetal und ein offenkettiges Halbacetal.
  Beim Verhältnis $1:2$ ensteht ein Acetal, das Halbacetal reagiert dafür
  unter Wasserabspaltung mit dem restlichen Alkohol.

  \noindent Halbacetale:\bigskip

  \chemfig{*6(--(-)-O-(-OH)--)} \qquad
  \chemfig{
    HO-[:-30]-[:30]-[:-30]-[:30]-[:-30]-[:30](-[2]OH)
    -[:-30]O-[:30]-[:-30]
  }
  \bigskip

  \noindent Acetale:\bigskip

  \chemfig{*6(--(-)-O-(-O-[:30]-[:-30])--)} \qquad
  \chemfig{
    HO-[:-30]-[:30]-[:-30]-[:30]-[:-30]-[:30]
    (-[2]O-[:30]-[:-30])
    -[:-30]O-[:30]-[:-30]
  }
  \qquad
  \chemfig{*7(---(-O-[:-30]-[:30])-O---)}
  \bigskip

  Die verschiedenen Produkte entstehen \emph{nicht} im gleichen Verhältnis!
\end{solution}

\begin{question}[name={Keto"=Enol"=Tautomerie}]
  Geben Sie die Strukturformel und den Namen der Enolform des Acetessigesters
  (\iupac{3\-Oxo\-Butansäure\-Ethyl\-Ester}) an.
\end{question}
\begin{solution}[name={Keto"=Enol"=Tautomerie}]
  \begin{center}
    \schemestart
      \chemfig{-[:30](=[2]O)-[:-30]-[:30](=[2]O)-[:-30]O-[:30]-[:-30]}
      \arrow{<=>[{[\ch{OH-}]}]}[,1.5]
      \chemfig{=[:30](-[2]OH)-[:-30]-[:30](=[2]O)-[:-30]O-[:30]-[:-30]}
      \+
      \chemfig{-[:30](-[2]OH)=[:-30]-[:30](=[2]O)-[:-30]O-[:30]-[:-30]}
    \schemestop
  \end{center}
  \iupac{3\-Hydroxy\-Buta\-3\-en\|säure\-Ethyl\-Ester},
  \iupac{3\-Hydroxy\-Buta\-2\-en\|säure\-Ethyl\-Ester},
  das zweite Produkt entsteht bevorzugt (konjugierte Doppelbindungen!).
\end{solution}

\begin{question}[name={Nukleophile Addition an \ch{C=O}"=Doppelbindungen}]
  Geben Sie die Strukturformeln der Produkte an, die durch nukleophile
  Addition aus den folgenden Molekülen entstehen können.  Berücksichtigen Sie
  insbesondere die Bildung von Ringen.
  \begin{tasks}(2)
    \task Ethanol und Ethanal
    \task \iupac{2\-Hydroxy\|ethanal}
    \task \iupac{2,3,4\-Tri\|hydroxy\|butanal}
  \end{tasks}
\end{question}

\begin{solution}[name=Nukleophile Addition]
  \begin{tasks}
    \task Ethanol und Ethanal
      \begin{center}
        \schemestart
          \chemfig{-[:30]-[:-30]OH}
          \+
          \chemfig{-[:150]=[2]O}
          \arrow
          \chemfig{-[:30]-[:-30]O-[:30](-[2]OH)-[:-30]}
        \schemestop
      \end{center}
    \task \iupac{2\-Hydroxy\|ethanal}
      \begin{center}
        \schemestart
          \chemfig{HO-[:-30]-[:30](-[2,,,,draw=none]\phantom{O})=[:-30]O}
          \+
          \chemfig{HO-[:-30]-[:30](-[2,,,,draw=none]\phantom{O})=[:-30]O}
          \arrow
          \chemfig{HO-[:-30]-[:30](-[2]OH)-[:-30]O-[:30]-[:-30]=[:30]O}
        \schemestop
      \end{center}
    \task \iupac{2,3,4\-Tri\|hydroxy\|butanal}
      \begin{center}
        \schemestart
          \chemfig{HO-[:-30]-[:30](-[2]OH)-[:-30](-[6]OH)-[:30]=[:-30]O}
          \arrow
          \chemfig{HO-*5(-(-HO)--O-(-HO)-)}
        \schemestop
      \end{center}
  \end{tasks}
\end{solution}

\begin{question}[name={Ringbildung}]
  Geben Sie die Strukturformeln aller fünf- und sechsgliedrigen Ringe an, die
  aus den folgenden Molekülen in verdünnter Lauge gebildet werden können:
  \begin{tasks}(2)
    \task \iupac{\D\-Ribose}
    \task \iupac{\D\-Galactose}
  \end{tasks}
\end{question}
\begin{solution}[name=Ringbildung]
  \setatomsep{3em}%
  Es werden die Haworth-Projektionen angegeben.
  
  \iupac{\a\-} und \iupac{\b\-\D\-Ribo\|furanose}:

  \chemfig{
    HO-[7,0.5]-[6,0.3]?<[7,0.7]
    (-[6,0.5]OH)
    -[,,,,line width=3pt]
    (-[6,0.5]OH)
    >[1,0.7](-[6,0.5]OH)-[:160,1.05]O?
  }
  \qquad
  \chemfig{
    HO-[7,0.5]-[6,0.3]?<[7,0.7]
    (-[6,0.5]OH)
    -[,,,,line width=3pt]
    (-[6,0.5]OH)
    >[1,0.7](-[2,0.5]OH)-[:160,1.05]O?
  }
  \bigskip
  
  \iupac{\a\-} und \iupac{\b\-\D\-Galacto\|pyranose}:

  \chemfig{
    HO-[6,0.5,2]?<[7,0.7]
    (-[2,0.5]OH)
    -[,,,,line width=3pt]
    (-[6,0.5]OH)
    >[1,0.7]
    (-[6,0.5]OH)
    -[3,0.7]O-[4]?
    (-[2,0.3]-[3,0.5]HO)
  }
  \qquad
  \chemfig{
    HO-[6,0.5,2]?<[7,0.7]
    (-[2,0.5]OH)
    -[,,,,line width=3pt]
    (-[6,0.5]OH)
    >[1,0.7]
    (-[2,0.5]OH)
    -[3,0.7]O-[4]?
    (-[2,0.3]-[3,0.5]HO)
  }
  \bigskip
  
  \iupac{\a\-} und \iupac{\b\-\D\-Galacto\|furanose}:

  \chemfig{
    HO-[7,0.5]-[6,0.3]
    (-[4,0.5]HO)
    -[6,0.3]?<[7,0.7]
    (-[2,0.5]OH)
    -[,,,,line width=3pt]
    (-[6,0.5]OH)
    >[1,0.7]
    (-[6,0.5]OH)
    -[:160,1.05]O?
  }
  \qquad
  \chemfig{
    HO-[7,0.5]-[6,0.3]
    (-[4,0.5]HO)
    -[6,0.3]?<[7,0.7]
    (-[2,0.5]OH)
    -[,,,,line width=3pt]
    (-[6,0.5]OH)
    >[1,0.7]
    (-[2,0.5]OH)-[:160,1.05]O?
  }
  \setatomsep{1.8em}
\end{solution}

\begin{question}[name={Oxidation der Alkohole und Aldehyde}]
  Formulieren Sie die Gleichungen für die folgenden Reaktionen mit
  Strukturformeln und Oxidationszahlen:
  \begin{tasks}
    \task Ethanol und heißes Kupfer-II-oxid
    \task Ethanal und heißes Kupfer-II-oxid
    \task Ethanal und Fehling"=Lösung (\ch{Cu^2+} und \ch{OH-}"=Ionen)
    \task Propanal und Tollens"=Lösung (\ch{Ag+} und \ch{OH-}"=Ionen)
    \task \iupac{2\-Hydroxy\|propanal} und Fehling"=Lösung
    \task Formulieren Sie die Aufgaben b bis e mit \iupac{\D\-Glucose}
      anstelle des Alkanals.
  \end{tasks}
\end{question}

\begin{solution}[name={Oxidation der Alkohole und Aldehyde}]
  \begin{tasks}
    \task Ethanol und heißes Kupfer"=II"=oxid:
      \begin{center}
        \schemestart
          \chemfig{-[:30]-[:-30]OH}
          \arrow(.base east--){0}[,0]
          \+
          \ch{CuO}
          \arrow(--.base west)
          \chemfig{-[:30]=[:-30]O}
          \arrow(.base east--){0}[,0]
          \+
          \ch{Cu}
          \+
          \ch{H2O}
        \schemestop
      \end{center}
    \task Ethanal und heißes Kupfer-II-oxid:
      \begin{center}
        \schemestart
          \chemfig{-[:30]=[:-30]O}
          \arrow(.base east--){0}[,0]
          \+
          \ch{CuO}
          \arrow(--.base west)
          \chemfig{-[:30](=[2]O)-[:-30]OH}
          \arrow(.base east--){0}[,0]
          \+
          \ch{Cu}
        \schemestop
      \end{center}
    \task Ethanal und Fehling"=Lösung:
      \begin{center}
        \schemestart
          \chemfig{-[:30]=[:-30]O}
          \arrow{0}[,0]
          \+
          \ch{2 Cu^2+}
          \+
          \ch{4 OH-}
          \arrow
          \chemfig{-[:30](=[2]O)-[:-30]OH}
          \arrow{0}[,0]
          \+
          \ch{Cu2O}
          \+
          \ch{2 H2O}
        \schemestop
      \end{center}
    \task Propanal und Tollens"=Lösung:
      \begin{center}
        \schemestart
          \chemfig{-[:-30]-[:30]=[:-30]O}
          \arrow{0}[,0]
          \+
          \ch{2 Ag+}
          \+
          \ch{2 OH-}
          \arrow
          \chemfig{-[:-30]-[:30](=[2]O)-[:-30]OH}
          \arrow{0}[,0]
          \+
          \ch{2 Ag v}
          \+
          \ch{H2O}
        \schemestop
      \end{center}
    \task \iupac{2\-Hydroxy\|propanal} und Fehling"=Lösung:
      \begin{center}
        \schemestart
          \chemfig{-[:-30](-[6]OH)-[:30]=[:-30]O}
          \arrow{0}[,0]
          \+
          \ch{2 Cu^2+}
          \+
          \ch{4 OH-}
          \arrow
          \chemfig{-[:-30](-[6]OH)-[:30](=[2]O)-[:-30]OH}
          \arrow{0}[,0]
          \+
          \ch{Cu2O}
          \+
          \ch{2 H2O}
        \schemestop
      \end{center}
    \task Exemplarisch \iupac{\D\-Glucose} und Tollens:
      \begin{center}
        \schemestart
          \chemfig{
            HO-[:-30]-[:30](<[2]OH)
            -[:-30](<:[6]OH)
            -[:30](<:[2]OH)
            -[:-30](<:[6]OH)
            -[:30]=[:-30]O
          }
          \arrow{->[\ch{2 Ag+ + 2 OH-}][${} -\ch{2 Ag v} - \ch{H2O}$]}[,2]
          \chemfig{
            HO-[:-30]-[:30](<[2]OH)
            -[:-30](<:[6]OH)
            -[:30](<:[2]OH)
            -[:-30](<:[6]OH)
            -[:30](-[2]OH)
            =[:-30]O
          }
        \schemestop
      \end{center}
  \end{tasks}
\end{solution}

\begin{question}[name={Reduzierende Kohlenhydrate}]
  Erläutern Sie den Unterschied zwischen einem reduzierendem und einem
  nicht"=reduzierendem Zucker.  Wie kann man diese Eigenschaft chemisch
  nachweisen (2 Methoden)?
\end{question}
\begin{solution}[name=Reduzierende Kohlenhydrate]
  Ein reduzierender Zuckerist in der Ringform ein Halbacetal und kann sich
  daher in die  offenkettige Form umwandeln.  Damit ist die Aldehyd-Gruppe
  zugänglich, die zur Carboxylgruppe oxidiert werden kann.  Auch Fructose ist
  ein reduzierender Zucker, da sie über Keto"=Enol"=Tautomerie mit der Glucose
  im Gleichgewicht steht.  Nicht reduzierende Zucker wie Saccharose, die kein
  Halbacetal haben, haben auch keine offenkettige Form und damit keine
  oxidierbare Aldehydgruppe.  Kriterium ist also die Zugänglichkeit der
  Aldehydgruppe.  Die Eigenschaft kann z.B. mit Fehling oder Tollens überprüft
  werden.
\end{solution}

\begin{question}[name={Nukleophile Substitution mit Alkoholen}]
  Formulieren Sie die Gleichungen für die folgenden Reaktionen mit
  Strukturformeln:
  \begin{tasks}
    \task Erwärmung einer salzsauren Lösung von Ethanol
    \task Erwärmung einer salzsauren Lösung von Dimethylether
    \task Erwärmung einer salzsauren Lösung von \iupac{1,2\-Propandiol} (3
      Produkte)
  \end{tasks}
\end{question}
\begin{solution}[name={Nukleophile Substitution mit Alkoholen}]
  \begin{tasks}
    \task\setatomsep{1.3em}salzsaure Lösung von Ethanol:\\
      \schemestart
        \ch{HCl}
        \+
        \chemfig{-[:30]-[:-30]OH}
        \arrow
        \chemfig{-[:30]-[:-30]Cl}
        \+
        \ch{H2O}
      \schemestop

    \task salzsaure Lösung von Dimethylether:\\
      \schemestart
        \ch{HCl}
        \+
        \ch{CH3-O-CH3}
        \arrow
        \ch{CH3-Cl}
        \+
        \ch{CH3-OH}
      \schemestop

    \task\setatomsep{1.3em}salzsaure Lösung von \iupac{1,2\-Propandiol}:\\
      \schemestart
        \ch{HCl}
        \+
        \chemfig{-[:30](-[2]OH)-[:-30]-[:30]OH}
        \arrow{->[$- \ch{HCl}$]}
        \chemfig{-[:30](-[2]Cl)-[:-30]-[:30]OH}
        \+
        \chemfig{-[:30](-[2]OH)-[:-30]-[:30]Cl}
        \+
        \chemfig{-[:30](-[2]Cl)-[:-30]-[:30]Cl}
      \schemestop
  \end{tasks}
\end{solution}

\newpage
\addsec{Lösungen}

\begin{figure}
  \centering
  \begin{tikzpicture}[x={(0.866cm,-0.5cm)}, y={(0.866cm,0.5cm)}, z={(0cm,1cm)}, scale=.5,
    %Option for nice arrows
    >=stealth, %
    inner sep=0pt, outer sep=2pt,%
    axis/.style={thick,->},
    wave/.style={thick,color=#1,smooth},
    polaroid/.style={fill=black!60!white, opacity=0.3},
    information text/.style=
      {inner sep=2pt,fill=red!10,rounded corners,text width=2.5cm, text centered}
  ]
    % Frame
    \coordinate (O) at (0,0,0);
    \draw[axis] (O) -- +(19,0,0) node [right] {x};
    \draw[axis] (O) -- +(0,2.5,0) node [right] {y};
    \draw[axis] (O) -- +(0,0,2) node [above] {z};

    \draw[thick,dashed] (-2,0,0) -- (O);

    % Licht vor dem Filter
    \draw[wave=blue, opacity=0.7, variable=\x, samples at={-2,-1.75,...,0}]
      plot (\x, {sin(2.0*\x r)}, 0);
    \draw[wave=black, opacity=0.7, variable=\x, samples at={-2,-1.75,...,0}]
      plot (\x, 0, {sin(2.0*\x r)});
    \draw[wave=black, opacity=0.7, variable=\x, samples at={-2,-1.75,...,0}]
      plot (\x, 0, {-sin(2.0*\x r)});
    \draw[wave=black, opacity=0.7, variable=\x, samples at={-2,-1.75,...,0}]
      plot (\x, {-sin(2.0*\x r)}, 0);
    \draw[wave=black, opacity=0.7, variable=\x, samples at={-2,-1.75,...,0}]
      plot (\x, {-.766*sin(2.0*\x r)}, {-.643*sin(2.0*\x r)});
    \draw[wave=black, opacity=0.7, variable=\x, samples at={-2,-1.75,...,0}]
      plot (\x, {.643*sin(2.0*\x r)}, {.766*sin(2.0*\x r)});
    \foreach \x in{-2,-1.75,...,0}{
      \draw[color=blue, opacity=0.7,->]
        (\x,0,0) -- (\x, {sin(2.0*\x r)}, 0);
      \draw[color=black, opacity=0.7,->]
        (\x,0,0) -- (\x, 0, {sin(2.0*\x r)});
      \draw[color=black, opacity=0.7,->]
        (\x,0,0) -- (\x, 0, {-sin(2.0*\x r)});
      \draw[color=black, opacity=0.7,->]
        (\x,0,0) -- (\x, {-sin(2.0*\x r)}, 0);
      \draw[color=black, opacity=0.7,->]
        (\x,0,0) -- (\x, {-.766*sin(2.0*\x r)}, {-.643*sin(2.0*\x r)});
      \draw[color=black, opacity=0.7,->]
        (\x,0,0) -- (\x, {.643*sin(2.0*\x r)}, {.766*sin(2.0*\x r)});
    }
    \node[information text,above] at (-2,0,1.5){\small Licht aller Schwingungs\-e\-be\-nen};
    % Erstes Gitter
    \filldraw[polaroid] (0,-2,-1.5) -- (0,-2,1.5) -- (0,2,1.5) -- (0,2,-1.5) -- (0,-2,-1.5)
      node[below, sloped, near end]{\small opt. Gitter};%
    % Direction of polarization
    \draw[thick,<->] (0,-1.75,-1) -- (0,-0.75,-1);
    % Polarisiertes Licht zwischen Gitter und Probe
    \draw[wave=blue, variable=\x,samples at={0,0.25,...,6}]
      plot (\x,{sin(2*\x r)},0);
    \foreach \x in{0, 0.25,...,6}
      \draw[color=blue,->] (\x,0,0) -- (\x,{sin(2*\x r)},0);

    \draw (3,3.2,0.7) node [information text]{\small linear pola\-risiertes Licht};

    % Probe
    \begin{scope}[thick]
      % Küvette
      \draw (6,-2,-1.5) -- (6,-2, 1.5)% node [above, sloped, midway]{Probe}
            (6,-2,-1.5) --node[midway,sloped,below,lencolor]{\small$d$} (8,-2,-1.5)
            (6, 2, 1.5) -- (8, 2, 1.5)
            (6,-2, 1.5) -- (8,-2, 1.5)
            (6,-2, 1.5) -- (6, 2, 1.5)
            (8, -2, 1.5) -- (8, 2, 1.5) -- (8, 2, -1.5)
              --node[sloped,below,near end]{\small Probe} (8,-2, -1.5) -- cycle;
      \draw[dashed] (6, 2, 1.5) -- (6, 2,-1.5)
            (6,-2,-1.5) -- (6, 2,-1.5)
            (6, 2,-1.5) -- (8,2,-1.5);
      % Achsen
      \draw[->] (8,0,0) -- ++(0,2.5,0) node [right] {y};
      \draw[->] (8,0,0) -- ++(0,0,2) node [above] {z};
      % Winkel
      \draw[angcolor] (8,1,1) .. controls (8,1.307,.541) .. (8,1.414,0);
      \draw[angcolor] (8,-1.5,-1.5) -- ++ (0,3,3);
      \node[angcolor,above] at (8,1,0) {\small$\alpha$};
    \end{scope}
    % polarisiertes Licht in der Probe
    \draw[wave=blue, variable=\x, samples at={6,6.25,...,8}]
      plot (\x,{(-.1115*(\x-6)+1)*sin(2*\x r)},{.3535*(\x-6)*sin(2*\x r)});
    \foreach \x in{6, 6.25,...,8}
      \draw[color=blue,->]
        (\x,0,0) -- (\x,{(-.1115*(\x-6)+1)*sin(2*\x r)},{.3535*(\x-6)*sin(2*\x r)});

    % polarisiertes Licht zwischen Probe und zweitem Gitter
    \draw[wave=blue, variable=\x, samples at={8,8.25,...,16}] 
      plot (\x, {0.707*sin(2*\x r)},  {0.707*sin(2*\x r)});  %
    \foreach \x in{8,8.25,...,16} {
      \draw[color=blue,->] (\x, 0, 0) --
          (\x, {0.707*sin(2*\x r)}, {0.707*sin(2*\x r)});
    }

    \draw (12,2.4,3.1) node [information text]
      {\small linear polarisiertes und um {\color{angcolor}$\alpha$} gedrehtes Licht};

    % Achsen
    \draw[->,thick] (14,0,0) -- ++(0,2.5,0) node [right] {y};
    \draw[->,thick] (14,0,0) -- ++(0,0,2) node [above] {z};

    % Zweites Gitter
    \filldraw[polaroid]
      (14,-2,-1.5) -- (14,-2,1.5) -- (14,2,1.5) -- (14,2,-1.5) -- (14,-2,-1.5)
      node[below, sloped, near end]{\small opt. Gitter};%
    \draw[thick, <->] (14, -1.5,-1) -- ++(0, .707, 0.707); %Polarization direction
  \end{tikzpicture}
  \caption{Schematische Darstellung eines Polarimeters.}
  \label{fig:polarimeter}
\end{figure}

\printsolutions

\end{document}
