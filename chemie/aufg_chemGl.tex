\documentclass[DIV11]{scrartcl}

\usepackage[T1]{fontenc}
\usepackage[utf8]{inputenc}
\usepackage[supstfm=libertinesups]{superiors}
\usepackage[utopia]{mathdesign}
\usepackage[oldstyle]{libertine}
\usepackage{microtype}

\usepackage[ngerman]{babel}
\usepackage{scrpage2}
\clearscrheadfoot
\pagestyle{scrheadings}
\chead{Seite \thepage}
\cfoot{\small C.\,Niederberger -- aktualisiert am \today}

\usepackage{upgreek}
\usepackage{chemmacros}
\chemsetup{
  option/language         = german,
  chemformula/name-format = \centering ,
  chemformula/format      = \libertineLF
%  phases/pos              = sub
}
\renewrobustcmd*\mch[1][]{\ch{^{#1}-}}
\renewrobustcmd*\pch[1][]{\ch{^{#1}+}}

\usepackage{siunitx}
\sisetup{
%  per-mode = fraction ,
  per-mode = symbol ,
  fixed-exponent = 0 ,
  scientific-notation = fixed ,
  exponent-product = \cdot
}

\usepackage{chemfig}
\renewcommand*\printatom[1]{#1}
\setatomsep{1.8em}

\usepackage{pgfplots,tikz}
\pgfplotsset{
  compat = 1.7 ,
  y tick label style = {
    /pgf/number format/.cd,
    fixed,
    fixed zerofill,
    precision = 2,
    /tikz/.cd} ,
  scaled ticks = false
}

\usepackage[load-headings]{exsheets}
\SetupExSheets{
  totoc ,
  headings = block-rev
}
\usepackage{enumitem}

\usepackage{fnpct}
\usepackage{booktabs}

\usepackage{mdframed}
\newmdenv[linecolor=red!20, linewidth=1ex]{intermission}

\usepackage[colorlinks]{hyperref}

\begin{document}

\begin{center}
  \Huge\sffamily Chemisches Gleichgewicht
\end{center}

\noindent
Im folgenden bedeuten Angaben in eckigen Klammern wie $[\ch{A}]$ die
Stoffmengenkonzentration $c(\ch{A})$ eines Stoffes.  Es gilt also wenn nicht
anders angegeben $[\ch{A}] \equiv c(\ch{A})$.


\addsec{Aufgaben}

\begin{question}[name=Reaktionsgeschwindigkeit]
Bei der Reaktion von \SI{5}{\milli\mole} \ch{Mg} mit \SI{10}{\milli\liter}
\SI{1}{\Molar} Salzsäure wurde das Volumen $V(\ch{H2})$ in \si{\milli\liter}
des entwickelten Wasserstoffgases über die Zeit $t$ in Minuten in die folgende
Tabelle eingetragen:
\begin{center}
 \begin{tabular}{ l*{5}{S[table-format=2.1]}*{2}{S[table-format=3.1]} }
  \toprule
   $t$ in \si{\minute}
     &  1   &  2   &  3   &  4   &  5   &  10   &  15 \\
  \midrule
   $V(\ch{H2})$ in \si{\milli\liter}
     & 29.0 & 50.5 & 66.5 & 78.2 & 87.4 & 106.4 & 110.7 \\
   $n(\ch{H2})$ in \si{\milli\mole} \\
   $n(\ch{Mg^2+})$ in \si{\milli\mole} \\
   $[\ch{Mg^2+}]$ in \si{\milli\mole\per\milli\liter} \\
  \bottomrule
 \end{tabular}
\end{center}
\begin{tasks}
 \task Rechnen Sie die Volumina $V(\ch{H2})$ in die Zahl $n(\ch{H2})$ der
   gebildeten Wasserstoffmoleküle um.  (\SI{1}{\milli\mole} Teilchen =
   \SI{24.8}{\milli\liter})
 \task Stellen Sie die Reaktionsgleichung auf.
 \task Geben Sie mit Hilfe der Reaktionsgleichung die Zahl $n(\ch{Mg^2+})$ der
   gebildeten \ch{Mg^2+}-Ionen an.
 \task Tragen Sie die Konzentrationen $[\ch{Mg^2+}]$ der \ch{Mg^2+}-Ionen in die
   letzte Zeile ein.
 \task Beschreiben Sie den Verlauf der Reaktion in einem
   Konzentrations-Zeit-Diagramm über \num{15} Minuten.
 \task Lesen Sie die Reaktionsgeschwindigkeit bei $t=0,2,5$ und $10$ Minuten
   ab und erstellen Sie daraus ein Reaktionsgeschwindigkeits-Zeit-Diagramm.
 \task Wie verändern sich die beiden Diagramme, wenn man die Temperatur um
   \SI{10}{\celsius} erhöht?
 \task Welche 2 Möglichkeiten gibt es noch, um die Reaktion zu beschleunigen?
\end{tasks}
\end{question}
\begin{solution}
\begin{center}
 \begin{tabular}{ l*{5}{S[table-format=2.3]}*{2}{S[table-format=3.3]} }
  \toprule
   $t$ in \si{\minute}
     &  1     &  2     &  3     &  4     &  5     &  10     &  15 \\
  \midrule
   $V(\ch{H2})$ in \si{\milli\liter}
     & 29.0   & 50.5   & 66.5   & 78.2   & 87.4   & 106.4   & 110.7 \\
   $n(\ch{H2})$ in \si{\milli\mole}
     &  1.29  &  2.25  &  2.97  &  3.59  &  3.90  &   4.75  &   4.94 \\
   $n(\ch{Mg^2+})$ in \si{\milli\mole}
     &  0.65  &  1.13  &  1.49  &  1.75  &  1.95  &   2.38  &   2.47 \\
   $[\ch{Mg^2+}]$ in \si{\milli\mole\per\milli\liter}
     &  0.065 &   .113 &   .149 &   .175 &   .195 &    .238 &    .247 \\
  \bottomrule
 \end{tabular}
\end{center}
\begin{tasks}
 \task Siehe Tabelle
 \task \ch{Mg\sld{} + 2 HCl\aq{} -> MgCl2\aq{} + H2\gas{} ^}
 \task siehe Tabelle
 \task siehe Tabelle
 \task \begin{tikzpicture}[baseline]
        \begin{axis}[
          anchor = north ,
          xlabel = $t$ in \si\minute ,
          ylabel = $c$ in \si{\milli\mole\per\milli\liter}]
         \addplot[color=red,mark=x,draw=none] coordinates {
            (1,.065)
            (2,.113)
            (3,.149)
            (4,.175)
            (5,.195)
           (10,.238)
           (15,.247)
         };
        \end{axis}
       \end{tikzpicture}
 \task Die Reaktionsgeschwindigkeit ist gegeben durch $v=\frac{\Delta
     c}{\Delta t}$,  kann also mit Hilfe der benachbarten Messwerte genähert
   werden.  Für den folgenden Plot wurde jeweils die Differenz zum
   vorhergehenden Messwert verwendet.
   
   \begin{tikzpicture}[baseline]
    \begin{axis}[
      anchor = north ,
      xlabel = $t$ in \si\minute ,
      ylabel = $v$ in \si{\milli\mole\per\milli\liter\per\minute}]
     \addplot[color=red,mark=x,draw=none] coordinates {
        (0,.065)
        (2,.048)
        (5,.020)
       (10,.0086)
     };
    \end{axis}
   \end{tikzpicture}
 \task Eine Erhöhung der Temperatur führt zu einer Beschleunigung der
   Reaktionsgeschwindigkeit.  Das bedeutet, dass der Plot aus e steiler
   starten  wird und sich der Endkonzentration schneller annähert.  Der Plot
   aus f wird bei einer höheren Geschwindigkeit starten aber auch etwas
   steiler abfallen.
 \task Man die Reaktion zum Beispiel beschleunigen, indem die Konzentration
   eines Edukts oder beider Edukte erhöht wird.  Auch die Absenkung des Drucks
   wird hier die Reaktion theoretisch beschleunigen.
\end{tasks}
\end{solution}

\begin{question}[name=Abhängigkeit von der Konzentration]
Bestimmen Sie die Konzentrationsabhängigkeit der Reaktionsgeschwindigkeit für die
folgenden Reaktionen:
\begin{tasks}(2)
 \task \ch{H2 + I2 <=> 2 HI}
 \task \ch{2 HI  <=>  H2 + I2}
 \task \ch{2 NO2 <=> N2O4}
 \task \ch{N2O4  <=>   2 NO2}
 \task \ch{N2 + 3 H2 <=> 2 NH3}
 \task \ch{2 NH3  <=>  N2 + 3 H2}
 \task \ch{IO3- + 3 HSO3- <=> I- + 3 SO4^2- + 3 H+}
 \task \ch{2 I- + S2O8^2- <=> I2 + 2 SO4^2-}
\end{tasks}
\end{question}
\begin{solution}
\begin{tasks}(3)
 \task $v=k\cdot[\ch{H2}][\ch{I2}]$
 \task $v=k\cdot[\ch{HI}]^2$
 \task $v=k\cdot[\ch{NO2}]^2$
 \task $v=k\cdot[\ch{N2O4}]$
 \task $v=k\cdot[\ch{N2}][\ch{H2}]^3$
 \task $v=k\cdot[\ch{NH3}]^2$
 \task $v=k\cdot[\ch{IO3-}][\ch{HSO3-}]^3$
 \task $v=k\cdot[\ch{I-}]^2[\ch{S2O8^2-}]$
\end{tasks}
\end{solution}

\begin{question}[name=Verteilungsgleichgewichte]
Ein Stoff \ch{A} löst sich in Hexan 10 mal besser als in Wasser:
$\frac{[\ch{A}]_\text{Hexan}}{[\ch{A}]_\text{Wasser}} = 10$. \SI{1}{\milli\mole}
dieses Stoffes sind in \SI{100}{\milli\liter} Wasser gelöst.  Um das Wasser zu
reinigen, stehen \SI{100}{\milli\liter} Hexan zur Verfügung.  Berechnen Sie
die im Wasser verbleibende Restmenge des Stoffes, wenn zur Extraktion
\begin{tasks}
  \task 1 mal \SI{100}{\milli\liter} Hexan in einem Arbeitsgang
  \task 2 mal \SI{50}{\milli\liter} Hexan in zwei Arbeitsgängen
  \task 10 mal \SI{10}{\milli\liter} Hexan in zehn Arbeitsgängen
\end{tasks}
verwendet werden.
\end{question}
\begin{solution}
Für die Gleichgewichtskonzentrationen gilt:
\begin{align*}
  \frac{[\ch{A}]_\text{H}}{[\ch{A}]_\text{W}} &= 10 \\
  [\ch{A}]_\text{H} &= 10 \cdot [\ch{A}]_\text{W}
\end{align*}
Die Ausgangsgleichung gilt für jeden einzelnen Waschgang.  Sie ist darum einmal,
zweimal beziehungsweise zehnmal anzuwenden.

\begin{tasks}
  \task Im Gleichgewicht verbleiben noch $x$ \si{\mole} des Stoffes im
    Wasser. Damit ergibt sich die Gleichung 
    \[
      \frac{\SI{1}{\milli\mole}-x}{\SI{100}{\milli\liter}}
      = 10 \cdot \frac{x}{\SI{100}{\milli\liter}}
    \]
    mit der Lösung $x = \SI{.091}{\milli\mole}$.
  \task Zunächst wenden wir einen allgemeinen Ansatz an.  $a$ ist die
    vorhandene Ausgangsmenge.  Die Gleichung
    \[
      \frac{a-x}{\SI{50}{\milli\liter}}
      = 10 \cdot \frac{x}{\SI{100}{\milli\liter}}
    \]
    hat die Lösung $x = \frac{1}{6}\,a$.  Da die Gleichung für jeden Waschgang
    Anwendung findet, beträgt die Endmenge
    \[
      x = \biggl(\frac{1}{6}\biggr)^2\cdot\SI{1}{\milli\mole}
      = \SI{.028}{\milli\mole} \, .
    \]
  \task Wieder stellen wir zunächst einen allgemeinen Ansatz auf mit $a$ für
    die Ausgangsmenge.  Die Gleichung
    \[
      \frac{a-x}{\SI{10}{\milli\liter}}
      = 10 \cdot \frac{x}{\SI{100}{\milli\liter}}
    \]
    hat die Lösung $x = \frac{1}{2}\,a$.  Dieses mal muss die Gleichung 10 mal
    angewendet werden.  Damit beträgt die Endmenge
    \[
      x = \biggl(\frac{1}{2}\biggr)^{10}\cdot\SI{1}{\milli\mole}
        = \SI{.001}{\milli\mole} \, .
    \]
\end{tasks}
\end{solution}

\begin{question}[name=Iodwasserstoffgleichgewicht,ID=HI-Glgw]
Wasserstoff und Iod reagieren zu Iodwasserstoff: \ch{H2 + I2 <=> 2 HI} mit $K = 54.3$.
\begin{tasks}
 \task Formulieren Sie das Massenwirkungsgesetz für diese Reaktion.
 \task Ein Kolben mit einem Volumen $V = \SI{1}{\liter}$ wurde mit \SI{0.1}{\mole}
   Wasserstoff und \SI{0.1}{\mole} Iod gefüllt.  Wie viel Mole Iodwasserstoff werden
   daraus gebildet?
 \task Warum ist das Volumen des Reaktionsgefäßes bei dieser Berechnung unerheblich?
 \task Wie viel Mole Wasserstoff und Iod werden gebildet, wenn \SI{0.1}{\mole}
   Iodwasserstoff eingefüllt wurde?
 \task Berechnen Sie die Ausbeute für b.
\end{tasks}
\end{question}
\begin{solution}
\begin{tasks}
 \task $K = \frac{[\ch{HI}]^2}{[\ch{H2}]\cdot[\ch{I2}]}$
 \task Wir verwenden das Ergebnis aus c, was bedeutet, dass wir uns um das
   Volumen nicht kümmern müssen.  Die Stoffmengen für Wasserstoff und Iod
   betragen im Gleichgewicht $[\ch{H2}]=[\ch{I2}]=0.1-x$ (in \si{\mole}), die
   Stoffmenge von \ch{HI} entsprechend $[\ch{HI}]=2x$.  Daraus folgt für das
   Massenwirkungsgesetz \[54.3 = \frac{(2x)^2}{(0.1-x)^2} \,.\]  Die Lösungen
   dieser quadratischen Gleichung betragen $x_1=0.08$ und $x_2=0.14$.  Die
   zweite Lösung ergibt chemisch keinen Sinn, da das negative Stoffmengen
   für \ch{H2} und \ch{I2} ergeben würde.  Also werden \SI{0.16}{\mole} \ch{HI}
   gebildet.
 \task Das Volumen kürzt sich aus $K$ heraus:
   \[
     K = \frac{[\ch{HI}]^2}{[\ch{H2}]\cdot[\ch{I2}]}
       = \frac{\left(\frac{n(\ch{HI})}{V}\right)^2}{\frac{n(\ch{H2})}{V}\cdot\frac{n(\ch{I2})}{V}}
       = \frac{n^2(\ch{HI})}{n(\ch{H2})\cdot n(\ch{I2})}
   \]
   Alternative Erklärung: die Summe der stöchiometrischen Faktoren beträgt
   $(-1) + (-1) +1 +1 = 0$, also gilt laut rotem Kasten $K_c = K_n \cdot V^0 =
   K_n$.
 \task Die Stoffmengen für Wasserstoff und Iod  betragen im Gleichgewicht
   $[\ch{H2}]=[\ch{I2}]=x$ (in \si{\mole}), die Stoffmenge von \ch{HI}
   entsprechend $[\ch{HI}]=0.1-2x$.  Daraus folgt für das Massenwirkungsgesetz
   \[54.3 = \frac{(0.1-2x)^2}{x^2} \,.\]  Die Lösungen
   dieser quadratischen Gleichung betragen $x_1=-0.02$ und $x_2=0.01$.  Die
   erste Lösung ergibt chemisch keinen Sinn, da das negative Stoffmengen
   für \ch{H2} und \ch{I2} ergeben würde.  Also werden je \SI{0.01}{\mole}
   \ch{H2} und \ch{I2} gebildet.
 \task Bei einem kompletten Umsatz wären in b \SI{0.2}{\mole} \ch{HI}
   gebildet worden.  Tatsächlich werden jedoch nur \SI{0.16}{\mole}
   gebildet.  Also beträgt die Ausbeute $\frac{0.16}{0.2}=\SI{80}{\percent}$.
\end{tasks}
\end{solution}

\begin{question}[name=Massenwirkungsgesetz]
Formulieren Sie das Massenwirkungsgesetz für die folgenden Reaktionen:
\begin{tasks}(2)
 \task \ch{CH4 + 2 O2    <=> CO2 + 2 H2O}
 \task \ch{2 SO2 + O2    <=> 2 SO3}
 \task \ch{H2O + HCl     <=> H3O+ + Cl- }
 \task \ch{2 H2O + H2SO4 <=> 2 H3O+ + SO4^2- }
 \task \ch{3 H2O + H3PO4 <=> 3 H3O+ + PO4^3- }
 \task \ch{CH4 + NH3     <=> HCN + 3 H2 }
\end{tasks}
\end{question}
\begin{solution}
\begin{tasks}(3)
 \task $K=\frac{[\ch{CO2}][\ch{H2O}]^2}{[\ch{CH4}][\ch{O2}]^2}$
 \task $K=\frac{[\ch{SO3}]^2}{[\ch{SO2}]^2[\ch{O2}]}$
 \task $K=\frac{[\ch{H3O+}][\ch{Cl-}]}{[\ch{H2O}][\ch{HCl}]}$
 \task $K=\frac{[\ch{H3O+}]^2[\ch{SO4^2-}]}{[\ch{H2O}]^2[\ch{H2SO4}]}$
 \task $K=\frac{[\ch{H3O+}]^3[\ch{PO4^3-}]}{[\ch{H2O}]^3[\ch{H3PO4}]}$
 \task $K=\frac{[\ch{HCN}][\ch{H2}]^3}{[\ch{CH4}][\ch{NH3}]}$
\end{tasks}
\end{solution}

\begin{question}[name=Estersynthese]
Bei der Reaktion von Ethansäure \ch{CH3COOH} mit Ethanol \ch{C2H5-OH} bildet
sich unter Austritt von Wasser Ethansäureethylester \ch{CH3-COO-C2H5}:
\begin{reaction*}
 CH3COOH + C2H5-OH <=> CH3-COO-C2H5 + H2O
\end{reaction*}
\begin{tasks}
  \task Formulieren Sie das Massenwirkungsgesetz für diese Reaktion.
  \task \SI{1}{\mole} Säure wurden mit \SI{1}{\mole} Alkohol versetzt.  Nach
    einiger Zeit sind \SI{0.8}{\mole} Ester entstanden. Berechnen Sie die
    Gleichgewichtskonstante $K$.
  \task Warum ist das Volumen der Reaktionslösung bei dieser Berechnung
    unerheblich?
  \task Wieviel Mole Ester werden gebildet, wenn \SI{1}{\mole} Säure mit
    \SI{4}{\mole} Alkohol versetzt wurden?
  \task Berechnen Sie die Ausbeute für d.
  \task Berechnen Sie die Ausbeute für d, wenn der Ester jeweils
    abdestilliert wird, so dass nur noch \SI{0.1}{\mole} Ester in der
    Reaktionslösung zurückbleiben.
\end{tasks}
\end{question}
\begin{solution}
  \begin{tasks}
    \task \[ K = \frac{[\ch{CH3-COO-C2H5}]\cdot[\ch{H2O}]}{[\ch{CH3COOH}]\cdot[\ch{C2H5-OH}]} \]
    \task Da das Volumen sich aus der Gleichung herauskürzt (siehe c),
      braucht es nicht beachtet zu werden.  Die Stoffmengen im
      Gleichgewicht betragen $n(\ch{H2O})=n(\ch{CH3-COO-C2H5})=\SI{0.8}{\mole}$
      und $n(\ch{CH3COOH})=n(\ch{C2H5-OH})=\SI{1}{\mole}-\SI{.8}{\mole} =
      \SI{.2}{\mole}$. Damit ergibt sich
      \[ K = \frac{0.8^2}{0.2^2} = 4 \,. \]
    \task Siehe Lösung zu Aufgabe \QuestionNumber{HI-Glgw}c.
    \task Die Stoffmengen der Reaktanden betragen im Gleichgewicht
      $n(\ch{H2O}]=[\ch{CH3-COO-C2H5})=x$, $n(\ch{CH3COOH})=1-x$ und
      $n(\ch{C2H5-OH})=4-x$ (jeweils in \si{\mole}).  Damit ergibt sich im
      MWG mit dem Ergebnis aus b und c \[ 4 = \frac{x^2}{(1-x)(4-x)}\,. \]
      Die Lösungen dieser quadratischen Gleichung betragen $x_1=0.93$ und
      $x_2=5.74$.  Die zweite Lösung ergibt chemisch keinen Sinn, da dann
      die Edukte negative Stoffmengen hätten.  Also werden \SI{0.93}{\mole}
      Ester gebildet.  Dieses Ergebnis bestätigt den Satz vom kleinsten
      Zwang, der sagt, dass sich bei der Konzentrationserhöhung eines der
      Edukte das Gleichgewicht nach rechts verschiebt.
    \task Bei vollem Reaktionsumsatz wären \SI{1}{\mole} Ester entstanden, also
      beträgt die Aubeute $\frac{0.93}{1}=\SI{93}{\percent}$.
    \task Setzen wir das Ergebnis aus d voraus, so betragen die
      Ausgangsstoffmengen $c(\ch{CH3COOH})=0.07$, $n(\ch{C2H5-OH})=3.07$
      und $n(\ch{H2O})=0.93$ (jeweils in \si{\mole}).  Durch Abdestillation
      des Esters wird sich das Gleichgewicht nach dem Prinzip des kleinsten
      Zwangs nach rechts verschieben.  Damit ergeben sich als Stoffmengen im
      Gleichgewicht (in \si{\mole}) $n(\ch{CH3COOH}) = 0.07-x$,
      $n(\ch{C2H5-OH}) = 3.07-x$, $n(\ch{H2O}) = 0.93+x$ und
      $n(\ch{CH3-COO-C2H5}) = 0.1$ (vorgegeben).  Das ergibt die Gleichung
      \[ 4 = \frac{0.1\cdot(0.93+x)}{(0.07-x)(3.07-x)} \]
      mit den Lösungen $x_1=0.06$ und $x_2=3.10$.  Nur die erste Lösung ergibt
      chemisch einen Sinn.  Damit sind von ursprünglich \SI{1}{\mole} Säure noch
      \SI{0.01}{\mole} übrig, was einer Ausbeute von $\frac{1-0.01}{1} =
      \SI{99}{\percent}$ entspricht.
  \end{tasks}
\end{solution}

\begin{question}[name=Bestimmung der Gleichgewichtskonstanten bei der Esterhydrolyse]
Bei der Reaktion von \SI{10}{\milli\mole} Ethansäuremethylester mit \SI{10}{\milli\liter}
\SI{10}{\Molar} Natronlauge nach der Gleichung
\begin{reaction*}
 CH3-COO-CH3 + OH- <=> CH3COO- + CH3OH
\end{reaction*}
wurden die folgenden Werte gemessen:
\begin{center}
 \begin{tabular}{ l*{7}{S[table-format=1.2]}}
  \toprule
   $t$ in Min
     & 1   & 2   & 3   & 4   & 5  & 10    & 15 \\
  \midrule
   $[\Hyd]$ in \si{\milli\mole\per\milli\liter}
     & 8.5 & 7.2 & 6.3 & 5.6 & 5. &  3.57 &  3.2 \\
   $[\ch{CH3COOCH3}]$ in \si{\milli\mole\per\milli\liter} \\
   $[\ch{CH3COO-}]$ in \si{\milli\mole\per\milli\liter} \\
   $[\ch{CH3OH}]$ in \si{\milli\mole\per\milli\liter} \\
  \bottomrule
 \end{tabular}
\end{center}
\begin{tasks}
  \task Formulieren Sie die Reaktionsgleichung mit Strukturformeln
  \task Ergänzen Sie die fehlenden Konzentrationen in der Tabelle.
  \task Beschreiben Sie den Verlauf der Reaktion in einem
    Konzentrations-Zeit-Diagramm über \num{15} Minuten.
  \task Wie hoch sind die Konzentrationen der Edukte und Produkte im
    Gleichgewicht
  \task Bestimmen Sie die Gleichgewichtskonstante $K$
\end{tasks}
\end{question}
\begin{solution}
  \begin{tasks}
    \task
      \schemestart
        \chemfig{H-C(-[2]H)(-[6]H)-C(=[2]\lewis{13,O})-\lewis{26,O}-C(-[2]H)(-[6]H)-H}
        \+ \ch{OH-}
        \arrow{<=>}
        \chemfig{H-C(-[2]H)(-[6]H)-C(=[2]\lewis{13,O})-{\Lewis{026,O}\,\mch}}
        \+
        \chemfig{H-C(-[2]H)(-[6]H)-\lewis{26,O}-H}
      \schemestop
    \task
      \begin{tabular}[t]{ l*{7}{S[table-format=1.2]}}
        \toprule
          $t$ in Min
            & 1   & 2   & 3   & 4   & 5  & 10    & 15 \\
        \midrule
          $[\Hyd]$ in \si{\milli\mole\per\milli\liter}
            & 8.5 & 7.2 & 6.3 & 5.6 & 5. &  3.57 &  3.2 \\
          $[\ch{CH3COOCH3}]$ in \si{\milli\mole\per\milli\liter}
            & 8.5 & 7.2 & 6.3 & 5.6 & 5. &  3.57 &  3.2 \\
          $[\ch{CH3COO-}]$ in \si{\milli\mole\per\milli\liter}
            & 1.5 & 2.8 & 3.7 & 4.4 & 5. &  6.43 &  6.8 \\
          $[\ch{CH3OH}]$ in \si{\milli\mole\per\milli\liter}
            & 1.5 & 2.8 & 3.7 & 4.4 & 5. &  6.43 &  6.8 \\
        \bottomrule
      \end{tabular}
    \task
      \begin{tikzpicture}[baseline]
        \begin{axis}[
          anchor = north ,
          xlabel = $t$ in \si\minute ,
          ylabel = $c$ in \si{\milli\mole\per\milli\liter}]
         \addplot[color=red,mark=x,draw=none] coordinates {
            (0,10)
            (1,8.5)
            (2,7.2)
            (3,6.3)
            (4,5.6)
            (5,5.0)
           (10,3.57)
           (15,3.2)
         };
         \addlegendentry{Edukt}
         \addplot[color=blue,mark=o,draw=none] coordinates {
            (0,0)
            (1,1.5)
            (2,2.8)
            (3,3.7)
            (4,4.4)
            (5,5.0)
           (10,6.43)
           (15,6.8)
         };
         \addlegendentry{Produkt}
        \end{axis}
      \end{tikzpicture}
    \task Die Konzentration im Gleichgewicht können anhand Tabelle und Grafik
      auf \SI{6.8}{\mole\per\liter} für die Produkte und
      \SI{3.2}{\mole\per\liter} für die Edukte geschätzt werden.
    \task Es gilt das Massenwirkungsgesetz
      \[ K = \frac{[\ch{CH3COO-}]\cdot[\ch{CH3OH}]}{[\ch{CH3-COO-CH3}]\cdot[\Hyd]} \,. \]
      Mit den Werten aus d berechnet sich damit $K$ wie folgt (Einheiten kürzen
      sich und werden daher ausgelassen):
      \[ K = \frac{6.8\cdot6.8}{3.2\cdot3.2} = 4.52 \,. \]
  \end{tasks}
\end{solution}

\begin{question}[name=Dimerisierung von Stickstoffdioxid]
Zwei Moleküle Stickstoffdioxid dimerisieren zu einem Molekül
Distickstofftetroxid:
\begin{reaction*}
 2 NO2 <=> N2O4
\end{reaction*}
\begin{tasks}
  \task Formulieren Sie das Massenwirkungsgesetz für diese Reaktion.
  \task Ein Kolben mit einem Volumen $V = \SI{1}{\liter}$ wurde mit
    \SI{0.26}{\mole} Stickstoffdioxid gefüllt.  Nach einiger Zeit sind
    \SI{0.08}{\mole} Distickstofftetroxid entstanden.  Berechnen Sie die
    Gleichgewichtskonstante $K$.
  \task Welche Einheit hat $K$?  In welchen Fällen hat $K$ keine Einheit?
  \task Ein Kolben mit einem Volumen $V = \SI{1}{\liter}$ wurde mit
    \SI{0.1}{\mole} Stickstoffdioxid gefüllt.  Wie viel Mole
    Distickstofftetroxid werden daraus gebildet?
  \task Wie viel Mole Stickstoffdioxid werden gebildet, wenn \SI{0.1}{\mole}
    Distickstofftetroxid eingefüllt wurde?
  \task Berechnen Sie die Ausbeute für b.
\end{tasks}
\end{question}
\begin{solution}
  \begin{tasks}
    \task \[ K=\frac{[\ch{N2O4}]}{[\ch{NO2}]^2} \]
    \task Die Gleichgewichtskonzentrationen betragen
      $[\ch{N2O4}]=\SI{.08}{\mole\per\liter}$ und
      $[\ch{NO2}]=\SI{.10}{\mole\per\liter}$.  Damit berechnet sich $K$ zu
      \[
        K = \frac{\SI{.08}{\mole\per\liter}}{(\SI{.10}{\mole\per\liter})^2}
          = \SI{8}{\liter\per\mole} \,.
      \]
    \task Die Einheit von $K$ wurde in b angegeben.  Keine Einheit hat $K$
      wenn  die Anzahl der Edukt-Teilchen mit der der Produkt-Teilchen
      übereinstimmt, das bedeutet, wenn die Summe der stöchiometrischen
      Faktoren der Edukte gleich der Summe der stöchiometrischen Faktoren der
      Produkte ist.  Dann kürzen sich die Einheiten im MWG.  In diesem Fall
      ist dann auch $K_c = K_n = K_x$, siehe roter Kasten.
    \task Die Konzentrationen im Gleichgewicht betragen $[\ch{NO2}]=0.1-2x$
      und $[\ch{N2O4}]=x$, jeweils in \si{\mole\per\liter}.  Damit ergibt sich
      die Gleichung \[ 8 = \frac{x}{(0.1-2x)^2} \] mit den Lösungen $x_1=0.023$
      und $x_2=0.108$. Nur die erste Lösung ergibt in diesem Kontext einen
      Sinn, also sind \SI{0.023}{\mole} \ch{N2O4} entstanden.  Beachten Sie,
      dass die Zahlen nur deshalb einfach so übertragbar sind, weil das
      Volumen des Kolbens genau ein Liter beträgt.
    \task Die Konzentrationen im Gleichgewicht betragen $[\ch{N2O4}]=0.1-0.5x$
      und $[\ch{NO}]=x$, jeweils in \si{\mole\per\liter}.  Damit ergibt sich
      die Gleichung \[ 8 = \frac{0.1-0.5x}{x^2} \] mit den Lösungen $x_1=-0.15$
      und $x_2=0.085$. Nur die zweite Lösung ergibt in diesem Kontext einen
      Sinn, also sind \SI{0.085}{\mole} \ch{NO2} entstanden.
    \task Aus \SI{0.26}{\mole} \ch{NO2} würden bei kompletter Dimerisierung
      \SI{0.13}{\mole} \ch{N2O4} entstehen, also beträgt die Ausbeute
      $\frac{0.08}{0.13} = \SI{61.5}{\percent}$.
  \end{tasks}
\end{solution}

\begin{question}[name=Ammoniaksynthese]
Stickstoff \ch{N2} und Wasserstoff \ch{H2} reagieren zu Ammoniak \ch{NH3}:
\begin{reaction*}
 N2 + 3 H2 <=> 2 NH3
\end{reaction*}
\begin{tasks}
  \task Formulieren Sie das Massenwirkungsgesetz für diese Reaktion.
  \task Ein Kolben mit einem Volumen $V = \SI{1}{\liter}$ wurde mit
    \SI{0.24}{\mole} Stickstoff und \SI{0.32}{\mole} Wasserstoff gefüllt.
    Nach einiger Zeit sind \SI{0.08}{\mole} Ammoniak entstanden.  Berechnen
    Sie die Gleichgewichtskonstante $K$.
  \task Ein Kolben mit einem Volumen $V = \SI{1}{\liter}$ wurde mit
    \SI{2}{\mole} Stickstoff und \SI{4}{\mole} Wasserstoff gefüllt.  Wie viel
    Mole Ammoniak werden daraus gebildet?
  \task Ein Kolben mit einem Volumen $V = \SI{1}{\liter}$ wurde mit
    \SI{0.1}{\mole} Ammoniak gefüllt.  Wie viel Mole Stickstoff und wie viel
    Mole Wasserstoff werden daraus gebildet?  (Nehmen Sie einen GTR zur
    Hilfe.)
  \task Berechnen Sie die Ausbeute für b und c.
\end{tasks}
\end{question}
\begin{solution}
  \begin{tasks}
    \task \[ K = \frac{[\ch{NH3}]^2}{[\ch{N2}]\cdot[\ch{H2}]^3} \]
    \task Die Konzentrationen im Gleichgewicht betragen
      $[\ch{NH3}]=0.08=2\cdot0.04$, $[\ch{N2}]=0.24-0.04=0.20$ und
      $[\ch{H2}]=0.32-3\cdot0.04=0.20$, jeweils in \si{\mole\per\liter}. Damit
      ergibt sich $K$ zu
      \[
        K = \frac
              {(\SI{0.08}{\mole\per\liter})^2}
              {\SI{0.20}{\mole\per\liter}\cdot(\SI{0.20}{\mole\per\liter})^3}
          = \SI{4}{\liter\squared\per\mole\squared} \,.
      \]
    \task Die Konzentrationen im Gleichgewicht betragen $[\ch{N2}]=2-x$,
      $[\ch{H2}]=4-3x$ und $[\ch{NH3}]=2x$, jeweils in \si{\mole\per\liter}.
      Damit ergibt sich folgende Gleichung:
      \[ 4 = \frac{(2x)^2}{(2-x)(4-3x)^3} \,. \]
      Ihre Lösungen betragen $x_1=1$ und $x_2=2.246$.  Die zweite Lösung gäbe
      für Stickstoff eine negative Konzentration, also ist die erste die
      richtige.  (Die Gleichung ist so gestellt, dass man die richtige Lösung
      auch erraten kann, da es sich hier um eine Gleichung vierten Grades
      handelt.)  Es werden also \SI{2}{\mole} Ammoniak gebildet.
    \task
      Die Konzentrationen im Gleichgewicht betragen $[\ch{N2}]=x$,
      $[\ch{H2}]=3x$ und $[\ch{NH3}]=0.1-2x$, jeweils in \si{\mole\per\liter}.
      Damit ergibt sich folgende Gleichung:
      \[ 4 = \frac{(0.1-2x)^2}{x\cdot(3x)^3} \,. \]
      Ihre Lösungen betragen $x_1=-0.234$ und $x_2=0.041$.  Nur die positive
      Lösung ergibt chemisch einen Sinn.  Es werden also \SI{0.041}{\mole}
      \ch{N2} und \SI{0.123}{\mole} \ch{H2} gebildet.
    \task Bei b gehen wir von der Reaktion \ch{N2 + 3 H2 -> 2 NH3} aus.  Bei
      einem vollständigen Umsatz wären \SI{4}{\mole} Ammoniak gebildet worden,
      die Ausbeute beträgt also $\frac{2}{4}=\SI{50}{\percent}$.

      Bei c gehen wir von der umgekehrten Reaktion \ch{2 NH3 -> N2 + 3 H2}
      aus.  Bei einem vollständigen Umsatz wären \SI{0.05}{\mole} Stickstoff
      gebildet worden, die Ausbeute beträgt also $\frac{0.041}{0.05} =
      \SI{82}{\percent}$.
  \end{tasks}
\end{solution}

\begin{question}[name=Schwefeltrioxidsynthese]
Zwei Moleküle Schwefeldioxid reagieren mit einem Molekül Sauerstoff \ch{O2} zu
zwei Molekülen Schwefeltrioxid:
\begin{reaction*}
 2 SO2 + O2 <=> 2 SO3
\end{reaction*}
\begin{tasks}
  \task Formulieren Sie das Massenwirkungsgesetz für diese Reaktion.
  \task Ein Kolben mit einem Volumen $V = \SI{1}{\liter}$ wurde mit
    \SI{0.4}{\mole} Schwefeldioxid und \SI{1.1}{\mole} Sauerstoff gefüllt.
    Nach einiger Zeit sind \SI{0.2}{\mole} Schwefeltrioxid entstanden.
    Berechnen Sie die Gleichgewichtskonstante.
  \task Ein Kolben mit einem Volumen $V = \SI{1}{\liter}$ wurde mit
    \SI{4}{\mole} Schwefeldioxid und \SI{2}{\mole} Sauerstoff gefüllt.  Wie
    viel Mole Schwefeltrioxid werden daraus gebildet?
  \task Wie viel Mole Schwefeldioxid werden gebildet, wenn \SI{0.1}{\mole}
    Schwefeltrioxid eingefüllt wurde? (Nehmen Sie den GTR zur Hilfe.)
  \task Berechnen Sie die Ausbeute für c.
\end{tasks}
\end{question}
\begin{solution}
  \begin{tasks}
    \task \[ K = \frac{[\ch{SO3}]^2}{[\ch{SO2}]^2\cdot[\ch{O2}]} \]
    \task Die Gleichgewichtskonzentrationen betragen $[\ch{SO3}]=0.1$,
      $[\ch{SO2}]=0.4-0.2=0.2$ und $[\ch{=2}]=1.1-\frac{1}{2}\cdot0.2=1.0$,
      jeweils in \si{\mole\per\liter}.  Aus dem MWG ergibt sich damit die
      Gleichgewichtskonstante
      \[
        K = \frac
              {(\SI{0.2}{\mole\per\liter})^2}
              {(\SI{0.2}{\mole\per\liter})^2\cdot\SI{1.0}{\mole\per\liter}}
          = \SI{1}{\liter\per\mole} \,.
      \]
    \task Im Gleichgewicht liegen folgende Konzentrationen vor: $[\ch{SO2}] =
      4-2x$, $[\ch{O2}]=2-x$ und $[\ch{SO3}]=2x$, jeweils in
      \si{\liter\per\mole}.  Mit dem MWG ergibt sich damit die Gleichung
      \[ 1 = \frac{(2x)^2}{(4-2x)^2(2-x)} \]
      mit der Lösung $x=1$.  (Die Aufgabe ist so gestellt, dass die richtige
      Lösung auch erraten werden kann, da es sich um eine Gleichung dritten
      Grades handelt.)  Es wurden also \SI{2}{\mole} \ch{SO3} gebildet.
      Denken Sie daran, dass $x$ eine Konzentration ist und wir die Zahl nur
      deshalb als Stoffmenge verwenden können, weil das Volumen des Kolbens
      genau ein Liter beträgt.
    \task Im Gleichgewicht liegen die Konzentrationen $[\ch{SO2}] =
      2x$, $[\ch{O2}]=x$ und $[\ch{SO3}]=0.1-2x$ vor, jeweils in
      \si{\liter\per\mole}. Daraus ergibt sich die Gleichung
      \[ 1 = \frac{(0.1-2x)^2}{(2x)^2\cdot x} \]
      mit der Lösung $x=0.042$.  Es wurden also \SI{0.084}{\mole} \ch{SO2}
      gebildet.  Denken Sie daran, dass $x$ eine Konzentration ist und wir die
      Zahl nur deshalb als Stoffmenge verwenden können, weil das Volumen des
      Kolbens genau ein Liter beträgt.
    \task Bei vollständigem Umsatz wären \SI{4}{\mole} Schwefeltrioxid
      gebildet worden.  Die Ausbeute beträgt also $\frac{2}{4} =
      \SI{50}{\percent}$.
  \end{tasks}
\end{solution}

\begin{question}[name=\pH-Wert einer schwachen Säure]
Phosphorsäure reagiert mit Wasser unter Bildung von Dihydrogenphosphat-Ionen
und Oxonium-Ionen:
\begin{reaction*}
 H3PO4 + H2O <=> H2PO4- + H3O+
\end{reaction*}
\begin{tasks}
  \task Formulieren Sie das Massenwirkungsgesetz für diese Reaktion.
  \task Wie viel Mole \ch{H2O} enthält ein Liter Wasser?  Wie groß ist
    $[\ch{H2O}]$ in einer verdünnten wäßrigen Lösung?
  \task Wie groß ist der \pH-Wert bei $[\ch{H3O+}] =
    \SI{104}{\mole\per\liter}$?  Wie groß ist $[\ch{H3O+}]$ bei $\pH = 8$?
  \task Löst man \SI{2}{\mole} Phosphorsäure in einem Liter Wasser, so erhält
    man $\pH = 1$.  Berechnen Sie die Gleichgewichtskonstante $K$ und runden
    Sie auf die 4. Stelle nach dem Komma.
  \task Wie viel Mole Phosphorsäure müssen in einem Liter Wasser gelöst
    werden, wenn man $\pH = 2$ erreichen will?
  \task Welchen \pH{} erhält man, wenn man \SI{0.5}{\mole} Phosphorsäure in
    einem Liter Wasser löst?
\end{tasks}
\end{question}
\begin{solution}
  \begin{tasks}
    \task \[ K = \frac{[\ch{H2PO4-}]\cdot[H3O+]}{[H3PO4]\cdot[H2O]} \]
    \task Die Konzentration von Wasser lässt sich mit Hilfe der Dichte
      berechnen.  Bei Raumtemperatur beträgt die Dichte von Wasser $\varrho =
      \SI{1.0}{\gram\per\cmc}$, ein Liter wiegt also \SI{1.0}{\kilo\gram}.  Da
      die molare Masse von Wasser \SI{18}{\gram\per\mole} beträgt, ergibt sich
      eine Konzentration von
      \[
        [\ch{H2O}]
         = \frac{\SI{1e3}{\gram}}{\SI{18}{\gram\per\mole}}/\SI{1}{\liter}
         = \SI{55.56}{\mole\per\liter} \,.
      \]
      Ein Liter Wasser enthält also \SI{55.56}{\mole} Moleküle pro Liter.
      Diese Konzentration wird ebenso auf verdünnte Lösungen zutreffen und
      die Verdünnung sich erst in kleineren Nach\-komma\-stellen bemerkbar machen.
    \task Die Definition des \pH-Werts ist $\pH = -\log[\ch{H3O+}]$, wobei mit
      $\log$ der dekadische Logarithmus gemeint ist und die Konzentration in
      \si{\mole\per\liter} angegeben wird.  Der \pH{} beträgt also $\pH =
      -\log(104) = -2.02$.  Die gesuchte Oxonium-Konzentration beträgt
      dementsprechend $[\ch{H3O+}] =
      \SI[scientific-notation=true]{1e-8}{\mole\per\liter} =
      \SI{1e-8}{\mole\per\liter}$. 
    \task Ein \pH{} von $1$ bedeutet eine Oxonium-Konzentration von
      \SI{.1}{\mole\per\liter}, was der Gleichgewichtskonzentration von
      Oxononium und Dihydrogenphosphat entspricht.  Die Konzentration der
      Phosphorsäure beträgt daher \SI{1.9}{\mole\per\liter}.  Damit errechnet
      sich $K$ zu \[ K = \frac{0.1\cdot0.1}{1.9\cdot55.5} = 0.0001 \,. \]
      (Die Einheiten kürzen sich im MWG.)
    \task Für diese Rechnung gehen wir von einer konstanten
      Wasserkonzentration aus. (Wie in b schon gesagt, ist die Angabe
      vernünftig, wie sie gerne an den Aufgaben d und f überprüfen können.)
      Ein $\pH = 2$ entspricht einer Oxonium-Konzentration von
      \SI{0.01}{\mole\per\liter}.  Eingesetzt in das MWG ergibt sich die
      Gleichung \[ 0.0001 = \frac{0.01^2}{x\cdot 55.6} \] mit der Lösung
      $x=0.018$. Man müsste also \SI{0.018}{\mole} Phosphorsäure lösen.
    \task Als Gleichgewichtskonzentrationen ergeben sich $[\ch{H2PO4-}] =
      [\ch{H3O+}] = x$, $[\ch{H3PO4}]=0.5-x$ und für Wasser $55.6-x$, jeweils
      in \si{\mole\per\liter}.  Das MWG liefert dann die Gleichung
      \[ 0.0001 = \frac{x\cdot x}{(0.5-x)(55.6-x)} \] mit den Lösungen
      $x_1=-0.056$ und $x_2=0.050$.  Da der negative Wert chemisch keinen Sinn
      ergibt, beträgt der gesuchte $\pH = -\log(0.050) = 1.3$.
  \end{tasks}
\end{solution}

\begin{question}[name=Prinzip vom kleinsten Zwang]
In welche Richtung verschieben sich die folgenden Gleichgewichte
\begin{tasks}[counter-format=tsk.]
 \task bei der angegebenen Konzentrationsänderung
 \task bei Druckerhöhung
 \task bei Temperaturerhöhung?
\end{tasks}
\begin{tasks}
  \task \ch{ N2\gas{} + 3 H2\gas{} <=> 2 NH3\gas{} }, $\Delta H < 0$\\
    $[\ch{NH3}]$ herauf- oder herabsetzen
  \task \ch{ C\sld{} + CO2\gas{} <=> 2 CO\gas{} }, $\Delta H > 0$\\
    $[\ch{CO2}]$ herauf- oder herabsetzen
  \task \ch{ Ca(HCO3)2\sld{} <=> CaCO3\sld{} + H2O\lqd{} + CO2\gas{} },
    $\Delta H > 0$\\
    $[\ch{CO2}]$ herauf- oder herabsetzen
  \task \ch{ H2O\gas{} + C_{(s)} <=> H2\gas{} + CO\gas{} }, $\Delta H > 0$\\
    $[\ch{CO}]$ herauf- oder herabsetzen
  \task \ch{ 2 SO2\gas{} + O2\gas{} <=> 2 SO3\gas{} }, $\Delta H < 0$\\
    $[\ch{O2}]$ herauf- oder herabsetzen
  \task \ch{ H3O+ \aq{} + \Hyd\aq{} <=> 2 H2O\lqd }, $\Delta H < 0$\\
    $[\ch{H3O+}]$ herauf- oder herabsetzen
\end{tasks}
\end{question}
\begin{solution}
  Es werden folgende Kürzel verwendet: $T\uparrow$ oder $p\uparrow$ bedeuten
  eine Erhöhung der Temperatur oder des Drucks. Mit $c\uparrow$ wird die
  Erhöhung der angegebenen Konzentration angezeigt.
  \begin{tasks}(2)
    \task $T\uparrow$: links;  $p\uparrow$: rechts; $c\uparrow$: links;
    \task $T\uparrow$: rechts; $p\uparrow$: links;  $c\uparrow$: rechts;
    \task $T\uparrow$: rechts; $p\uparrow$: links;  $c\uparrow$: links;
    \task $T\uparrow$: rechts; $p\uparrow$: links;  $c\uparrow$: links;
    \task $T\uparrow$: links;  $p\uparrow$: rechts; $c\uparrow$: rechts;
    \task $T\uparrow$: links;  $p\uparrow$: ---;    $c\uparrow$: rechts;
  \end{tasks}
\end{solution}

\begin{intermission}
  Bevor es an die nächsten Aufgaben geht, kommt hier noch etwas
  mathematisch"=physikalische Theorie zur Gleichgewichtskonstanten.
  Genauer  werden wir uns verschiedene Arten der Darstellung anschauen und wie
  die  verschiedenen Konstanten ineinander umgerechnet werden können.  Das
  wird bei den Lösungen der folgenden Aufgaben behilflich sein.

  Bei allen bisherigen Aufgaben wurde $K=K_c$ angenommen, das heißt die
  Berechnung der Gleichgewichtskonstanten aus den
  Gleichgewichtskonzentrationen.  Es gibt allerdings keinen Grund, die
  Konzentrationen zu verwenden, auch wenn das allgemein üblich ist und oft
  stillschweigend angenommen wird, wenn nichts anderes angegeben wird.
  Genauso gut könnte man die Stoffmengen verwenden ($K_n$), die Molenbrüche
  ($K_x$), die Partialdrücke ($K_p$) und so weiter.  Wir wollen uns nun mit
  $K_c$, $K_n$ und $K_x$ etwas genauer beschäftigen.

  Klären wir zunächst, was ein \emph{Molenbruch} ist.  Der Molenbruch $x_i$
  der Substanz $i$ ist definiert als ihre Stoffmenge $n_i$ geteilt durch die
  Gesamtstoffmenge $n$ und ist damit dimensionslos:
  \[
    x_i = \frac{n_i}{n} \qquad n_i = x_i \cdot n
  \]

  Nun führen wir eine kompakte Schreibweise für Summen und Produkte ein.
  Die Summe $n_1 + n_2 + \cdots + n_j$ schreiben wir als $\sum_{i=1}^j n_i$
  oder kurz $\sum_i n_i$:
  \[ \sum_i n_i = n_1 + n_2 + \ldots + n_i \]
  Analog verwenden wir $\prod_i n_i$ für Produkte:
  \[ \prod_i n_i = n_1 \cdot n_2 \cdot \ldots \cdot n_i \]

  Diese Schreibweise ermöglicht eine Kurzdarstellung der
  Gleichgewichtskonstanten.  Nehmen wir eine allgemeine Reaktion
  \begin{reaction*}
    $(-\nu_A)$ A + $(-\nu_B)$ B <=> $\nu_C$ C + $\nu_D$ D
  \end{reaction*}
  Hier habe ich eine weitere kleine Neuerung eingeführt: $\nu_i$ gibt den
  stöchiometrischen Faktor des Stoffes $i$ an.  $\nu_i$ ist negativ, wenn $i$
  ein Edukt ist, weswegen die Edukte in der Reaktionsgleichung ein
  Minuszeichen erhalten haben, um positive stöchiometrische Faktoren zu haben.
  Nun stellen wir das MWG auf:
  \[
    K_c = \frac
            { c_C^{\nu_C} \cdot c_D^{\nu_D} }
            { c_A^{-\nu_A } \cdot c_B^{-\nu_B} }
  \]
  Diese Gleichung lässt sich als Produkt schreiben:
  \[
    K_c =  c_C^{\nu_C} \cdot c_D^{\nu_D} \cdot c_A^{\nu_A } \cdot c_B^{\nu_B} \\
        = \prod_i c_i^{\nu_i}
  \]
  Ganz entsprechend lassen sich nun $K_n$ und $K_x$ notieren:
  \[
    K_n = \prod_i n_i^{\nu_i} \qquad
    K_x = \prod_i x_i^{\nu_i}
  \]
  Jetzt werden wir mit dieser neuen Schreibweise Zusammenhänge zwischen den
  drei genannten Gleichgewichtskonstanten herstellen.
  \[
    K_c = \prod_i c_i^{\nu_i}
        = \prod_i \biggl(\frac{n_i}{V}\biggr)^{\nu_i}
        = \prod_i n_i^{\nu_i} \cdot V^{-\nu_i}
        = K_n \cdot V^{-\sum_i\nu_i}
  \]
  $K_c$ lässt sich also aus $K_n$ berechnen, indem man es mit dem Volumen hoch
  der negativen Summe der stöchiometrischen Faktoren multipliziert.
  (Vergessen Sie dabei nicht, dass die stöchiometrischen Faktoren der Edukte
  negativ in die Summe eingehen.)  Einen ähnlichen Zusammenhang können wir zu
  $K_x$ herstellen.  Dafür beginnen wir mit einem Zwischenergebnis der obigen
  Gleichung:
  \[
    K_c = \prod_i n_i^{\nu_i} \cdot V^{-\nu_i}
        = \prod_i (x_i\cdot n)^{\nu_i} \cdot V^{-\nu_i}
        = \prod_i x_i \cdot \biggl(\frac{V}{n}\biggr)^{-\nu_i}
        = K_x \cdot V_m^{-\sum_i\nu_i}
  \]
  $K_c$ errechnet sich aus $K_x$ durch Multiplikation mit dem molaren Volumen
  $V_m = \frac{V}{n}$ hoch der negativen Summe der stöchiometrischen Faktoren.
  Dieses letzte Ergebnis wollen wir uns nun für zwei konkrete Beispiele
  anschauen, die Sie in den beiden verbleibenden Übungsaufgaben gebrauchen
  können.

  Für die Reaktion \ch{H2 + 3 N2 <=> 2 NH3} ist
  \[
    K_c = \frac{ c^2(\ch{NH3}) }{ c(\ch{N2}) \cdot c^3(\ch{H2}) }
    \quad \text{und} \quad
    K_x = \frac{ x^2(\ch{NH3}) }{ x(\ch{N2}) \cdot x^3(\ch{H2}) } \,.
  \]
  Die Summe der stöchiometrischen Faktoren ist $\sum_i\nu_i = -1 + (-3) +2 =
  -2$. Damit ergibt sich folgende Beziehung: $K_c=K_x\cdot V_m^2$.

  Für die Reaktion \ch{2 SO2 + O2 <=> 2 SO3} ist
  \[
    K_c = \frac{ c^2(\ch{SO3}) }{ c^2(\ch{SO2}) \cdot c(\ch{O2}) }
    \quad \text{und} \quad
    K_x = \frac{ x^2(\ch{SO3}) }{ x^2(\ch{SO2}) \cdot x(\ch{O2}) } \,.
  \]
  Die Summe der stöchiometrischen Faktoren ist $\sum_i\nu_i = -2 + (-1) +2 =
  -1$. Damit ergibt sich folgende Beziehung: $K_c=K_x\cdot V_m$.

  Und nun viel Spaß mit den weiteren Übungsaufgaben!
\end{intermission}

\bigskip

\begin{question}[name=Technische Ammoniaksynthese,ID=tech:NH3]
Die exotherme Ammoniaksynthese aus den Elementen wird technisch bei $p =
\SI{200}{\bar}$ und $T = \SI{500}{\celsius}$ unter Verwendung eines
Eisenkatalysators durchgeführt (Haber-Bosch-Verfahren).  In einem
geschlossenen Reaktionsbehälter erhält man einen Volumenanteil von
\SI{17.6}{\percent} \ch{NH3}, wenn vorher \ch{N2} und \ch{H2} im Verhältnis
$1:3$ eingefüllt wurden.  Um die Ausbeute zu erhöhen, werden die Gase nach
dem Verlassen des Ammoniakofens abgekühlt, wobei \ch{NH3} kondensiert und
abgetrennt werden kann.  Das restliche \ch{N2} und \ch{H2} wird wieder in den
Ammoniakofen zurückgepumpt.
\begin{tasks}
  \task Erklären Sie mit Hilfe des Prinzips vom kleinsten Zwang, wie sich hohe
    Drücke und Temperaturen auf die Ausbeute auswirken.
  \task Aus wirtschaftlichen Gründen wird eine ziemlich hohe
    Reaktionstemperatur gewählt, obwohl sich dadurch die Ausbeute verringert.
    Erklären Sie, warum sich die Produktionsleistung (= Ausbeute pro Zeit) des
    Ammoniakofens durch die Temperaturerhöhung trotzdem verbessert.
  \task Warum wird der Katalysator zugefügt?  Welchen Einfluss hat der
    Katalysator auf die Ausbeute?
  \task Berechnen Sie die Gleichgewichtskonstante $K$ bei $p = \SI{200}{\bar}$
    und $T = \SI{500}{\celsius}$.  Das Molvolumen bei $p = \SI{200}{\bar}$ und
    $T = \SI{500}{\celsius}$ ist $V_m = \SI{0.32}{\liter\per\mole}$. Der
    Molenbruch von Ammoniak im Gleichgewicht beträgt $x_{\ch{NH3}}=0.176$.
    
    Hinweis: Stellen Sie die Gleichgewichtsstoffmengen der beteiligten Stoffe
    in Abhängigkeit der Gesamtstoffmenge $n_0$ vor Beginn der
    Gleichgewichtseinstellung auf und bestimmen Sie damit die Molenbrüche im
    Gleichgewicht.  Daraus lässt sich dann die Gleichgewichtskonstante
    berechnen.
  \task Erklären Sie mit Hilfe des Prinzips vom kleinsten Zwang, warum die
    Abtrennung des \ch{NH3} und Rückführung des restlichen \ch{N2} und \ch{H2}
    die Ausbeute erhöhen.
\end{tasks}
\end{question}
\begin{solution}
  \begin{tasks}
    \task Ein hoher Druck wird die Ausbeute erhöhen, da eine Druckerhöhung das
      Gleichgewicht auf die Seite mit weniger Gasteilchen, also die
      Produktseite, verschiebt.  Da die Reaktion exotherm ist, verschiebt eine
      Temperaturerhöhung das Gleichgewicht in Richtung Ausgangsstoffe.  Eine
      hohe Temperatur ist also ungünstig für eine hohe Ausbeute.
    \task Eine Temperaturerhöhung verschiebt nicht nur das Gleichgewicht nach
      links, sondern erhöht auch die Reaktionsgeschwindigkeit
      (vgl. RGT-Regel).  Das bedeutet, dass sich trotz der ungünstigen
      Verschiebung des Gleichgewichts die Menge an Produkt in einem bestimmten
      Zeitraum erhöhen kann, sich also die Produktionsleistung erhöht.
    \task Ein Katalysator verringert die Aktivierungsenergie, beschleunigt die
      Reaktion also.  Das ist nicht nur günstig für eine möglichst hohe
      Produktionsleistung, sondern ermöglicht auch, die Reaktion bei
      niedrigeren Temperaturen durchzuführen, als es ohne Katalysator möglich
      wäre.  Das spart Energie und damit Kosten.  Ein Katalysator ändert die
      Lage des Gleichgewichts nicht, hat also auch auf die Ausbeute keinen
      Einfluss.
    \task Die eingefüllte Gesamtstoffmenge $n_0$ verteilt sich im Verhältnis
      $3:1$ auf Stickstoff und Wasserstoff: $n_{\ch{N3}} = 0.25n_0$ und
      $n_{\ch{H2}} = 0.75n_0$.  Damit ergeben sich im Gleichgewicht folgende
      Stoffmengen: $n_{\ch{N2}} = 0.25n_0 - x$, $n_{\ch{H2}} = 0.75n_0 - 3x$ und
      $n_{\ch{NH3}} = 2x$. Die Gesamtstoffmenge ist dann die Summe daraus,
      also $n=n_0-2x$.  Aus dem Molenbruch für \ch{NH3} lässt sich nun $x$ in
      Abhängigkeit von $n_0$ bestimmen.
      \[
        x_{\ch{NH3}} = 0.176 = \frac{2x}{n_0-2x}
        \quad \Leftrightarrow \quad
        x = 0.0748n_0
      \]
      Daraus berechnen sich die Stoffmengen zu $n_{\ch{N2}} = 0.175n_0$,
      $n_{\ch{H2}} = 0.526n_0$, $n = 0.850n_0$.  Als Molenbrüche ergeben sich
      mit $x_i=\frac{n_i}{n}$ die Werte $x_{\ch{N2}} = 0.206$ und $x_{\ch{H2}}
      = 0.619$.  Nun endlich lässt sich auch die Gleichgewichtskonstante
      berechnen.  Dafür verwenden wir das Ergebnis $K_c=K_x\cdot V_m^2$ aus
      dem roten Kasten:
      \[
        K_c = K_x\cdot V_m^2
            = \frac{0.176^2}{0.206\cdot0.619^3}\cdot
              (\SI{0.32}{\liter\per\mole})^2
            = \SI{0.065}{\liter\squared\per\mole\squared}
      \]
    \task Durch die Entfernung des Ammoniaks aus dem Gleichgewicht, also durch
      Verringerung seiner Konzentration, bewirkt man eine Verschiebung des
      Gleichgewichts nach rechts.  Eine Rückführung von \ch{N2} und \ch{H2}
      verschiebt ihrerseits durch die Erhöhung der Edukt"=Konzentrationen
      eine Verschiebung nach rechts.  Dadurch wird die Ausbeute weiter maximiert.
  \end{tasks}
\end{solution}

\begin{question}[name=Technische Schwefelsäuresynthese]
\begin{enumerate}[label=\arabic*.]
  \item Rösten von Pyrit \ch{FeS2} an der Luft:
    \begin{reaction*}
      4 FeS2 + 11 O2 <=> 2 Fe2O3 + 8 SO2 \qquad{} "mit~\Enthalpy{-3309}"
    \end{reaction*}
  \item Kontaktverfahren: Schwefeldioxid wird an der Luft bei $T =
    \SI{600}{\celsius}$ und $p = \SI{1}{\bar}$ über Vanadiumpentoxid \ch{V2O5}
    geleitet, das als Katalysator dient:
    \begin{itemize}
      \item 1. Schritt: Oxidation des \ch{SO3} durch \ch{V2O5}:
        \begin{reaction*}
          2 V2O5 + 2 SO2 <=> V2O4 + 2 SO3
        \end{reaction*}
      \item 2. Schritt: Regeneration des Katalysators durch Luft-\ch{O2}:
        \begin{reaction*}
          2 V2O4 + O2 <=> 2 V2O5
        \end{reaction*}
      \item Gesamtbilanz:
        \begin{reaction*}
          2 SO2 + O2 <=> 2 SO3 \qquad{} "mit~\Enthalpy{-198}"
        \end{reaction*}
    \end{itemize}
  \item \ch{SO3} löst sich nur schwer in Wasser, relativ leicht und schnell
    dagegen in ca. \SI{20}{\percent}-iger Schwefelsäure, wobei sich zunächst
    Dischwefelsäure \ch{H2S2O7} bildet.  In diesem Fall dienen die schon vor
    der Reaktion vorhandenen Schwefelsäuremoleküle als Katalysator:
    \begin{itemize}
     \item 1. Schritt: Bildung der Dischwefelsäure:
       \begin{reaction*}
         SO3 + H2SO4 <=> H2S2O7
       \end{reaction*}
     \item 2. Schritt: Regeneration des Katalysators durch Wasser:
       \begin{reaction*}
         H2S2O7 + H2O <=> 2 H2SO4
       \end{reaction*}
     \item Gesamtbilanz:
       \begin{reaction*}
         SO3 + H2O <=> H2SO4
       \end{reaction*}
    \end{itemize}
\end{enumerate}
\begin{tasks}
  \task Geben Sie die Oxidationszahlen für Eisen und Schwefel beim Röstvorgang
    an und kennzeichnen Sie den Elektronenübergang durch einen Pfeil.
  \task In einem geschlossenen Reaktionsgefäß bei $T = \SI{600}{\celsius}$ und
    $p = \SI{1}{\bar}$ wurden \ch{SO2} und \ch{O2} im Verhältnis $2:1$
    eingepumpt und zur Reaktion gebracht.  Nach Einstellung des
    Gleichgewichtes \ch{2 SO2 + O2 <=> 2 SO3} wurde ein Volumenanteil von
    \SI{76}{\percent} \ch{SO3} gemessen.  Das Molvolumen bei diesen
    Bedingungen ist \SI{72.5}{\liter\per\mole}.  Berechnen Sie die
    Gleichgewichtskonstante.  (Der Molenbruch von \ch{SO3} im Gleichgewicht
    beträgt entsprechend dem prozentualen Anteil $x_{\ch{SO3}}=0.76$.)
    
    Hinweis: Stellen Sie die Gleichgewichtsstoffmengen der beteiligten Stoffe
    in Abhängigkeit der Gesamtstoffmenge $n_0$ vor Beginn der
    Gleichgewichtseinstellung auf und bestimmen Sie damit die Molenbrüche im
    Gleichgewicht.  Daraus lässt sich dann die Gleichgewichtskonstante
    berechnen.
  \task Erklären Sie, warum die Ausbeute an \ch{SO3} bei steigenden Drücken
    und sinkenden Temperaturen wächst.
  \task Warum arbeitet man im großtechnischen Verfahren trotzdem mit einer
    relativ hohen Temperatur von \SI{600}{\celsius}?
  \task \ch{SO3} entsteht auch ohne Katalysator bei der Verbrennung von
    Schwefel oder Schwefeldioxid.  Welchen Zweck hat der Katalysator?
  \task Erklären Sie, warum die Abtrennung des \ch{SO3} und Rückführung des
    restlichen \ch{SO2} und \ch{O2} die Ausbeute erhöhen.
\end{tasks}
\end{question}
\begin{solution}
  \begin{tasks}
    \task Die Oxidationsstufe von Eisen erhöht sich von II auf III, die von
      Schwefel von $-$I auf IV.  Damit werden insgesamt $44$ Elektronen
      übertragen.  Die $22$ Sauerstoff Atome ändern ihre Oxidationsstufe von
      $0$ auf $-$II.
      \vspace*{2\bigskipamount}
      \begin{center}
        \ch{
          4 "\OX{Fe-r,\ox{2,Fe}}\OX{S-r,\ox{-1,S}}" {}2 + 11 O2
          <=> 2 "\OX{Fe-o,\ox{3,Fe}}" 2O3 + 8 "\OX{S-o,\ox{4,S}}" O2 
        }
      \end{center}
      \vspace*{2\bigskipamount}
      \redox(Fe-r,Fe-o)[draw,->]{$-4\cdot1\,\el$}%
      \redox(S-r,S-o)[->][-1]{$-8\cdot5\,\el$}%
      Der Sauerstoff hat zwei verschiedene Reaktiosnpartner, einmal Eisen und
      einmal Schwefel.  Der EInfachheit halber wurde daher hier auf die
      bildhafte Darstellung seiner Elektronenübertragung verzichtet.
    \task Die eingefüllte Gesamtstoffmenge $n_0$ verteilt sich im Verhältnis
      $2:1$ auf Schwefeldioxid und Sauerstoff: $n_{\ch{SO2}} = \frac{2}{3}n_0$
      und $n_{\ch{O2}} = \frac{1}{3}n_0$.  Damit ergeben sich im Gleichgewicht
      folgende Stoffmengen: $n_{\ch{SO2}} = \frac{1}{3}n_0 - 2x$, $n_{\ch{O2}}
      = \frac{1}{3}n_0 - x$ und $n_{\ch{SO3}} = 2x$. Die Gesamtstoffmenge ist
      dann die Summe daraus, also $n=n_0-x$.  Aus dem Molenbruch für \ch{SO3}
      lässt sich nun $x$ in Abhängigkeit von $n_0$ bestimmen.
      \[
        x_{\ch{SO3}} = 0.76 = \frac{2x}{n_0-x}
        \quad \Leftrightarrow \quad
        x = 0.275n_0
      \]
      Daraus berechnen sich die Stoffmengen zu $n_{\ch{SO2}} = 0.117n_0$,
      $n_{\ch{O2}} = 0.058n_0$, $n = 0.725n_0$.  Als Molenbrüche ergeben sich
      mit $x_i=\frac{n_i}{n}$ die Werte $x_{\ch{SO2}} = 0.161$ und
      $x_{\ch{O2}} = 0.080$.  Nun endlich lässt sich auch die
      Gleichgewichtskonstante berechnen.  Dafür verwenden wir das Ergebnis
      $K_c=K_x\cdot V_m$ aus dem roten Kasten:
      \[
        K_c = K_x\cdot V_m
            = \frac{0.76^2}{0.161^2\cdot0.08}\cdot
              (\SI{72.5}{\liter\per\mole})
            = \SI[scientific-notation=true]{20e3}{\liter\squared\per\mole\squared}
      \]
    \task Da auf der rechten Seite weniger Gasteilchen vorliegen, verschiebt
      eine Druckerhöhung das Gleichgewicht nach rechts, zugunsten der
      Ausbeute.  Eine Temperaturerniedrigung begünstigt ebenfalls die rechte
      Seite, da die Reaktion exotherm ist.
    \task Hohe Temperaturen beschleunigen die Reaktionsgeschwindigkeit, was
      die Produktionsleistung günstig beeinflusst, siehe auch die entsprechende
      Antwort zu Übung~\QuestionNumber{tech:NH3}  Der eigentliche Grund ist
      jedoch ein anderer. Die Temperatur muss etwa zwischen \SI{420}{\degree}
      und \SI{620}{\degree} liegen. Bei tieferen Temperaturen wird der
      Katalysator inaktiv, bei höheren Temperaturen zersetzt er sich.
    \task Ohne Katalysator ist die Reaktion zu langsam  und bei hohen
      Temperaturen ist das Gleichgewicht zudem zu weit zu den Edukten
      verschoben, um eine wirtschaftliche Ausbeute zu bringen.
    \task Sowohl eine Abtrennung von \ch{SO3} als auch eine Rückführung der
      Edukte sorgen nach dem Prinzip des kleinsten Zwangs eine Verschiebung
      zugunsten des Produkts und damit für eine höhere Ausbeute.
  \end{tasks}
\end{solution}

\newpage
\addsec{Lösungen}

Die Lösungen gehen davon aus, dass die benötigte Mathematik beherrscht
wird.  Das bedeutet zum Beispiel, dass die Umrechnung in Prozentwerte und das
Lösen quadratischer Gleichungen nicht vorgerechnet wird.

\printsolutions

\end{document}
