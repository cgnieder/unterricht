% arara: pdflatex
% arara: pdflatex
\documentclass{scrartcl}

\usepackage[T1]{fontenc}
\usepackage[utf8]{inputenc}
\usepackage[supstfm=libertinesups]{superiors}
\usepackage{libertine}
\usepackage{microtype}

\usepackage[greek,ngerman]{babel}
\usepackage{scrpage2}
\clearscrheadfoot
\pagestyle{scrheadings}
\chead{Seite \thepage}
\cfoot{\small C.\,Niederberger -- aktualisiert am \today}

\usepackage{upgreek}
\usepackage{chemmacros}
\chemsetup{
  option/language         = german,
  chemformula/name-format = \centering ,
%  phases/pos              = sub
}
\renewrobustcmd*\mch[1][]{\ch{^{#1}-}}
\renewrobustcmd*\pch[1][]{\ch{^{#1}+}}

\DeclareChemReaction[star]{gthreactions}{gather}
\newcommand*\rctref[1]{\{\ref{#1}\}}

\usepackage{siunitx}
\sisetup{
  detect-all ,
%  per-mode = fraction ,
  per-mode = symbol ,
  fixed-exponent = 0 ,
%  scientific-notation = true ,
  exponent-product = \cdot ,
  list-final-separator = { und } ,
  list-pair-separator = { und } ,
  range-phrase = { bis }
}
\DeclareSIUnit{\graddh}{grad\,dH}

\usepackage{chemfig}
\renewcommand*\printatom[1]{#1}
\setatomsep{1.8em}

\usepackage{pgfplots,tikz}
\pgfplotsset{
  compat = 1.7 ,
  scaled ticks = false
}

\usepackage[load-headings]{exsheets}
\SetupExSheets{
  totoc ,
  headings = block-rev
}

\usepackage{fnpct}

\usepackage{booktabs}
\usepackage{collcell}
\newcolumntype{C}{>{\collectcell\ch}l<{\endcollectcell}}

\usepackage{mdframed}
\newmdenv[
  backgroundcolor = red!20,
  hidealllines    = true,
  leftmargin      = 2em ,
  rightmargin     = 2em ,
  skipabove       = \baselineskip ,
  skipbelow       = \baselineskip ,
  splittopskip    = .5\baselineskip ,
  splitbottomskip = .5\baselineskip ,
  frametitle      = Definition ,
  frametitlefont  = \large\sffamily\scshape ,
  frametitlebackgroundcolor = red!40
]{definition}

\newmdenv[
  backgroundcolor = black!5,
  hidealllines    = true,
  leftmargin      = 2em ,
  rightmargin     = 2em ,
  skipabove       = \baselineskip ,
  skipbelow       = \baselineskip ,
  splittopskip    = .5\baselineskip ,
  splitbottomskip = .5\baselineskip ,
  frametitle      = Beispiel ,
  frametitlefont  = \large\sffamily\scshape ,
  frametitlebackgroundcolor = black!10
]{beispiel}


\usepackage{multicol}
\usepackage[colorlinks]{hyperref}

\begin{document}

\begin{center}
  \Huge\sffamily Fette und Seifen
\end{center}

\begin{question}[name=Iodzahl und Verseifungszahl]
  Tabelle \ref{tab:fett_eigenschaften} zeigt die Zusammensetzung von drei
  pflanzlichen Fetten in Prozent.
  \begin{tasks}
    \task Ordnen Sie den drei Fetten jeweils die drei Verseifungszahlen $170$,
      $200$ und $255$ sowie die drei Iodzahlen $8$, $50$ und $100$ zu und
      begründen Sie.
    \task Formulieren Sie an einem Beispiel die Reaktionsgleichung für die
      Bestimmung der Iodzahl und den Reaktionsmechanismus.
    \task Formulieren Sie an einem Beispiel die Reaktionsgleichung für die
      Bestimmung der Verseifungszahl und den Reaktionsmechanismus.
  \end{tasks}
\end{question}

\begin{table}[hbp]
  \centering
  \caption{Zusammensetzung einiger Fette}\label{tab:fett_eigenschaften}
  \begin{tabular}{lCSSS}
    \toprule
      \bfseries Fettsäure & &
      {\bfseries Kokosfett} & {\bfseries Rapsöl} & {\bfseries Palmöl} \\
    \midrule
      Capronsäure    & C5H11COOH  & 1  & 0  & 0 \\
      Caprylsäure    & C7H15COOH  & 9  & 0  & 0 \\
      Caprinsäure    & C9H19COOH  & 7  & 0  & 0 \\
      Laurinsäure    & C11H23COOH & 47 & 0  & 0 \\
      Myristinsäure  & C13H27COOH & 16 & 0  & 2 \\
      Palmitinsäure  & C15H31COOH & 10 & 4  & 39 \\
      Stearinsäure   & C17H35COOH & 2  & 2  & 5 \\
      Ölsäure        & C17H33COOH & 6  & 57 & 45 \\
      Linolsäure     & C17H31COOH & 2  & 24 & 9 \\
      Linolensäure   & C17H29COOH & 0  & 13 & 0 \\
    \bottomrule
  \end{tabular}
\end{table}

\begin{question}[name=Verseifung und deutsche Härte]
  \begin{tasks}
    \task Formulieren Sie für die Bildung von Seifen aus Fetten eine
      Reaktionsgleichung.
    \task Erklären Sie die Reinigungswirkung der Seife unter Verwendung der
      Begriffe:
      \begin{itemize}
        \item Grenzflächenaktivität
        \item Dispergiervermögen
        \item Emulgiervermögen
     \end{itemize}
    \task In hartem Wasser steht ein Teil der zugegebenen Seife nicht für die
      Reinigungswirkung zur Verfügung.  Berechnen Sie, wie viel Gramm
      Natriumhexadecanoat beim Waschen mit Wasser von \SI{20}{\graddh}
      (deutsche Härte) nicht für den Waschvorgang genutzt werden können.
      Wasserverbrauch für Hauptwaschgang: \SI{20}{\liter}.  $\SI{1}{\graddh} =
      \SI{7.15}{\milli\gram}$ \ch{Ca^2+} pro Liter Wasser.
  \end{tasks}
\end{question}

\end{document}
