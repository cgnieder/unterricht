% arara: pdflatex
% arara: pdflatex
\documentclass[DIV11]{scrartcl}

\usepackage[T1]{fontenc}
\usepackage[utf8]{inputenc}
\usepackage[supstfm=libertinesups]{superiors}
\usepackage{libertine}
\usepackage{microtype}

\usepackage[ngerman]{babel}
\usepackage{scrpage2}
\clearscrheadfoot
\pagestyle{scrheadings}
\chead{Seite \thepage}
\cfoot{\small C.\,Niederberger -- aktualisiert am \today}

\usepackage{upgreek}
\usepackage{chemmacros}
\chemsetup{
  option/language         = german,
  chemformula/name-format = \centering ,
%  phases/pos              = sub
}
\renewrobustcmd*\mch[1][]{\ch{^{#1}-}}
\renewrobustcmd*\pch[1][]{\ch{^{#1}+}}

\DeclareChemReaction[star]{gthreactions}{gather}
\newcommand*\rctref[1]{\{\ref{#1}\}}

\usepackage{siunitx}
\sisetup{
%  per-mode = fraction ,
  per-mode = symbol ,
  fixed-exponent = 0 ,
%  scientific-notation = true ,
  exponent-product = \cdot ,
  list-final-separator = { und } ,
  list-pair-separator = { und } ,
  range-phrase = { bis }
}

\usepackage{chemfig}
\renewcommand*\printatom[1]{#1}
\setatomsep{1.8em}

\usepackage{pgfplots,tikz}
\pgfplotsset{
  compat = 1.7 ,
  scaled ticks = false
}

\usepackage[load-headings]{exsheets}
\SetupExSheets{
  totoc ,
  headings = block-rev
}
\usepackage{enumitem}

\usepackage{fnpct}
\usepackage{booktabs}
\usepackage{collcell}
\newcolumntype{C}{>{\collectcell\ch}l<{\endcollectcell}}

\usepackage[colorlinks]{hyperref}

\begin{document}

\begin{center}
  \Huge\sffamily Puffer
\end{center}

\addsec{Übungen}
Auf Seite~\pageref{tab:pKs-werte} finden Sie eine Tabelle mit den \pKa-Werten
einiger Säuren, die Sie zur Lösung der Aufgaben verwenden können.  Wenn nicht
anders angegeben, bedeutet die Notation $[\ch{A}]$ die Konzentration des
Stoffes \ch{A}, also $c(\ch{A})$.  Mit $\log$ ist, wenn nicht anders
angegeben, der dekadische Logarithmus gemeint.

\bigskip

\begin{question}
\begin{tasks}
  \task Wie muß ein Puffer für den \pH-Wert $7.3$ zusammengesetzt sein, wenn
    eine Konzentration für das Dihydrogenphosphation von
    \SI{0.30}{\mole\per\liter} vorgegeben wird? 
  \task Puffert dieser Puffer besser gegen Säuren oder Basen?
  \task Zu \SI{100}{\milli\liter} dieses Puffers gibt man
   \SI{10}{\milli\liter} \SI{1}{\Molar}  Salzsäure. Welchen \pH-Wert zeigt die
   Lösung jetzt?
  \task Welchen \pH-Wert zeigt eine ungepufferte Lösung mit dem gleichen
    \pH-wert wie der Puffer bei der gleichen Säurezugabe?
  \task Geben Sie ein Beispiel für ein wichtiges Puffersystem in Natur oder
    Technik. 
\end{tasks}
\end{question}
\begin{solution}
  \begin{tasks}
    \task Einsetzen der gegebenen Werte in die
      Henderson"=Hasselbalch"=Gleichung ergibt die Gleichung
      \[ 7.3 = 7.21 - \log \frac{0.30}{x} \]
      mit der Lösung $x=0.37$.  Damit ist der Puffer aus \SI{0.30}{\Molar}
      Dihydrogenphosphat und \SI{0.37}{\Molar} Hydrogenphosphat
      zusammengesetzt.
    \task Der Puffer puffert etwa gleich gut gegen Säuren und Basen, durch den
      leichten Überschuß Hydrogenphosphat etwas besser gegen Säuren.
    \task Nach Zugabe der Säure betragen die neuen Konzentrationen
      $[\ch{H2PO4-}] = (0.3\cdot0.1+0.01/0.11) = 0.364$ und $[\ch{HPO4^2-}] =
      (0.37\cdot0.1-0.01)/0.11 = 0.245$, jeweils in \si{\mole\per\liter}.
      Damit ergibt sich der $\pH = 7.04$, der \pH{} sank also um $0.26$
      Einheiten.
    \task In einer ungepufferten Lösung mit $\pH = 7.3$ beträgt die
      Oxonium"=Konzentration $[\ch{H3O+}] = \SI{5.0e-8}{\mole\per\liter}$.
      Damit ergibt sich nach Zugabe der Säure eine Konzentration von
      $[\ch{H3O+}] = (\num{5.0e-8}\cdot0.1+0.01=0.01$, die vor Zugabe der
      Säure vorhandene Stoffmenge ist also vernachlässigbar.  Die Lösung hat
      dann einen $\pH = 2$, was einer Änderung um $5.3$ Einheiten entspricht.
    \task Der in dieser Aufgabe verwendete Dihydrogenphosphat\slash
      Hydrogenphosphat"=Puffer ist Bestandteil des wichtigen Puffersystems
      Blut.
  \end{tasks}
\end{solution}

\begin{question}
In \SI{500}{\milli\liter} einer Lösung sind \SI{10}{\gram} Essigsäure und
\SI{15}{\gram} Natriumacetat gelöst.
\begin{tasks}
  \task Welchen \pH-Wert zeigt diese Lösung?
  \task Welchen \pH-Wert zeigt diese Lösung nach Zugabe von \SI{2}{\gram}
    \ch{NaOH} (Volumenänderung vernachlässigen)?
  \task Welchen \pH-Wert zeigt diese Lösung nach Zugabe von \SI{10}{\gram}
    \ch{NaOH}?
\end{tasks}
\end{question}
\begin{solution}
  \begin{tasks}
    \task Die molaren Massen von Essigsäure ist \SI{60.0}{\gram\per\mole}, die
      von Natriumacetat \SI{82.0}{\gram\per\mole}.  Die Konzentrationen betragen
      also $[\ch{CH3COOH}] = (10/60)/0.5 = 0.333$ und $[\ch{CH3COO-}] =
      (15/82)/0.5 = 0.366$, jeweils in \si{\mole\per\liter}.  Eingesetzt in die
      Henderson"=Hasselbalch"=Gleichung ergibt sich der $\pH = 4.80$.
    \task Die molare Masse von \ch{NaOH} beträgt \SI{40.0}{\gram\per\mole}.
      Es werden also \SI{0.05}{\mole} Natriumhydroxid zugegeben.  Damit
      ergeben sich die neuen Konzentrationen $[\ch{CH3COOH}] =
      (0.333\cdot0.5-0.05)/0.5 = 0.233$ und $[\ch{CH3COO-}] =
      (0.366\cdot0.5+0.05)/0.5 = 0.466$, jeweils in \si{\mole\per\liter}.  Der
      neue \pH{} ergibt sich dann zu $\pH = 5.06$.
    \task \SI{10}{\gram} Natriumhydroxid entsprechen \SI{0.25}{\mole}.  In der
      Lösung befanden sich $0.333\cdot0.5 = \SI{0.167}{\mole}$ Essigsäure.
      Diese werden vollständig deprotoniert.  Es verbleiben also
      \SI{0.083}{\mole} Hydroxid.  Damit ergibt sich ein $\pOH = 1.1$ und ein
      $\pH = 12.9$.
  \end{tasks}
\end{solution}

\begin{question}
Hergestellt werden soll ein Puffer mit $\pH = 7$.
\begin{tasks}
  \task Erläuteren und begründen Sie Zusammensetzung und Wirkungsweise eines
    Puffers.
  \task Wählen Sie ein Puffersystem aus und berechnen Sie seine
    Zusammensetzung für den gewünschten \pH-Bereich.
  \task Puffert die berechnete Lösung besser gegen Säure- oder Basenzugabe?
  \task Setzen Sie die Konzentration der Säure des Puffers gleich
    \SI{0.1}{\Molar} und berechnen Sie, welchen \pH-Wert \SI{20}{\milli\liter}
    der Lösung nach Zugabe von \SI{5}{\milli\liter} \SI{0.01}{\Molar}
    Salzsäure hat.
  \task Welchen \pH-Wert hätten \SI{20}{\milli\liter} einer ungepufferten
    Lösung mit \pH = $7$ nach Zugabe der gleichen Säuremenge wie in d?
\end{tasks}
\end{question}
\begin{solution}
  \begin{tasks}
    \task Ein Puffer ist eine wässrige Lösung einer schwachen Säure und ihrer
      korrespondierenden Base.  Die Säure kann zugebenene Hydroxid-Ionen
      abfangen und reagiert dabei zu ihrer korrespondierenden Base.  Die Base
      kann zugegebene Oxonium-Ionen abfangen und reagiert dabei zur Säure.
      Wenn die Konzentrationen der Pufferbestandteile groß im Vergleich zu den
      zugegebenen Mengen Hodroxid oder Oxonium sind, verändern sich dabei die
      Konzentrationen der Pufferbestandteile nicht wesentlich und ihr
      Verhältnis bleibt weitestgehend gleich, was den \pH-Wert stabil hält.

      Am besten puffert ein Puffer bei gleichen Bestandteilen Säure und
      korrespondierender Base, also bei dem \pH-Wert, der dem \pKa-Wert der
      Säure entspricht.  Die Stoffmengen von Säure und Base bestimmen die
      Pufferkapazität.  Wird eine von beiden vollständig umgesetzt, ist der
      Puffer zerstört.
    \task Für $\pH=7$ bieten sich entweder der Dihydrogenphosphat\slash
      Hydrogenphosphat"=Puffer ($\pKa=7.21$) oder der Kohlensäure\slash
      Hydrogencarbonat"=Puffer ($\pKa=6.46$) an.  Das benötigte Verhältnis
      $x=\text{Säure} : \text{Base}$ ergibt sich aus der
      Henderson"=Hasselbalch"=Gleichung
      \[
        x = 10^{\pKa-\pH}
      \]
      und ergibt für den Dihydrogenphosphat\slash Hydrogenphosphat"=Puffer $x
      = 1.6 = 8:5$ und für den Kohlensäure\slash Hydrogencarbonat"=Puffer $x =
      0.29 \approx 3:10$.
    \task Der Dihydrogenphosphat\slash Hydrogenphosphat"=Puffer puffert besser
      gegen zugegebene Base, der Kohlensäure\slash Hydrogencarbonat"=Puffer
      besser gegen zugegebene Säure.
    \task Die Konzentration der Säure beträgt nach Zugabe der Salzsäure
      $(0.1\cdot0.02 + 0.01\cdot0.05)/0.025 = 0.1$ (in
      \si{\mole\per\liter}). Im Fall des Dihydrogenphosphat\slash
      Hydrogenphosphat"=Puffers beträgt die Konztentration der Base
      $(0.1\cdot0.02/1.6 - 0.01\cdot0.05)/0.025 = 0.03$ (in
      \si{\mole\per\liter}).  Damit ergibt sich $\pH = 6.7$.  Im Fall des
      Kohlensäure\slash Hydrogencarbonat"=Puffers beträgt die Konzentration
      der Base $(0.1\cdot0.02/0.29 - 0.01\cdot0.05)/0.025 = 0.26$ (in
      \si{\mole\per\liter}).  Damit ergibt sich $\pH = 6.9$.  Wie in c
      festgestellt puffert der Kohlensäure\slash Hydrogencarbonat"=Puffer
      besser gegen die Säurezugabe.
    \task Ohne Puffer hätte sich eine Oxonium"=Konzentration von
      \SI{0.02}{\mole\per\liter} eingestellt, was $\pH = 1.7$ entspricht.
  \end{tasks}
\end{solution}

\begin{question}[ID=HCl/NaCl]
Warum kann eine Mischung aus Salzsäure und Kochsalz nicht als Puffer wirken?
\end{question}
\begin{solution}
  Salzsäure ist eine starke Säure ($\pKa=-6$).  Dementsprechend ist das
  Chlorid \ch{Cl-} eine so schwache Base ($\pKb = 20$), dass es quasi gar
  nicht basisch reagiert.  Anders gesagt: das Gleichgewicht der Reaktion
  \ch[arrow-ratio=.1,arrow-min-length=4.5em]{HCl + H2O <=>> H3O+ + Cl-} liegt
  (fast) vollständig auf der Seite der dissoziierten Säure, so dass sich kein
  Verhältnis zwischen Säure und korrespondierender Base einstellen kann, das
  als Puffer wirken kann.
\end{solution}

\begin{question}
\begin{tasks}
  \task Eine Lösung von \SI{100}{\milli\liter} enthält \SI{0.2}{\Molar}
    \ch{HOAc} und \SI{0.4}{\Molar} \ch{^-OAc} (\ch{^-OAc} ist die
    Kurzschreibweise für \ch{CH3COO-}).  Erläutern Sie die Vorgänge, die das
    Protolysegleichgewicht in dieser Lösung bestimmen und leiten Sie aus dem
    Massenwirkungsgesetz ab, wie man den \pH-Wert dieser Lösung berechnet.
    Berechnen Sie den \pH-Wert.
  \task Zu dieser Lösung gibt man \SI{10}{\milli\liter} \SI{0.2}{\Molar}
    \ch{HCl}.  Berechnen Sie  den \pH-Wert, den die Lösung jetzt hat.
\end{tasks}
\end{question}
\begin{solution}
  \begin{tasks}
    \task Es lassen sich mehrere Protolysereaktionen aufschreiben
      \begin{gthreactions}
        HOAc + H2O <=> ^-OAc + H3O+ \label{rct:HOAc} \\
        ^-OAc + H2O <=> HOAc + OH- \label{rct:OAc-} \\
        2 H2O <=> H3O+ + OH- \label{rct:autoprotolyse}
      \end{gthreactions}
      Reaktion~\rctref{rct:HOAc} beschreibt die Säureprotolyse der Essigsäure,
      Reaktion~\rctref{rct:OAc-} die entsprechende basische Reaktion des
      Acetats.  Beide hängen über den Satz von Heß mit
      Reaktion~\rctref{rct:autoprotolyse}, der Autoprotolyse des Wassers,
      miteinander zusammen.  (Damit ließe sich zeigen, weshalb \pKa{} und
      \pKb{} korrespondierender Base und Säure zusammen $14$ ergeben.)  Das
      Massenwirkungsgesetz für Reaktion~\rctref{rct:HOAc} ist
      \begin{gather}
        K   = \frac{[\ch{^-OAc}][\ch{H3O+}]}{[\ch{HOAc}][\ch{H2O}]} \label{eq:MWG} \\
        \Ka = \frac{[\ch{^-OAc}][\ch{H3O+}]}{[\ch{HOAc}]} \label{eq:Ks}
      \end{gather}
      \Ka{} in Gleichung~\eqref{eq:Ks} ist definiert als $K\cdot[\ch{H2O}]$,
      weshalb sich Gleichung~\eqref{eq:Ks} aus Gleichung~\eqref{eq:MWG} durch
      Multiplikation mit $[\ch{H2O}]$ ergibt.  Logarithmieren und
      anschließende Multiplikation mit $-1$ von Gleichung~\eqref{eq:Ks} führt
      mit den Definitionen $\pH=-\log[\ch{H3O+}]$ und $\pKa=-\log\Ka$ auf die
      Henderson"=Hasselbalch"=Gleichung für den Essigsäure\slash Acetat"=Puffer:
      \begin{gather}
        \pKa = -\log\biggl(\frac{[\ch{^-OAc}]}{[\ch{HOAc}]}\biggl) + \pH \\
        \pH = \pKa -\log\biggl(\frac{[\ch{HOAc}]}{[\ch{^-OAc}]}\biggl)
      \end{gather}
      Damit ergibt sich nun ein \pH{} von $5.1$.
    \task Die Konzentrationen nach Zugabe der Salzsäure betragen (in
      \si{\mole\per\liter}) $[\ch{HOAc}] = (0.2\cdot0.1 + 0.2\cdot0.01)/0.11 =
      0.2$ und $[\ch{^-OAc}] = (0.4\cdot0.1 - 0.2\cdot0.01)/0.11 = 0.34$.  Der
      \pH{} beträgt nun also $5.0$.
  \end{tasks}
\end{solution}

\begin{question}[ID=HCOOH]
Gegeben ist eine Lösung  von  \SI{0.2}{\Molar} \ch{HCOOH} und \SI{0.3}{\Molar}
Natriumformiat (\SI{100}{\milli\liter}).
\begin{tasks}
  \task Berechnen Sie den \pH-Wert der Lösung.
  \task Ist diese Lösung ein Puffer? Wenn ja, puffert sie besser gegen Säuren
    oder Basen?
  \task Welchen \pH-Wert zeigt diese Lösung nach Zugabe von
    \SI{10}{\milli\liter} \SI{0.1}{\Molar} \ch{HCl}?
  \task Welchen \pH-Wert würde eine Lösung, die ungepuffert ist und den
    gleichen \pH-Wert hat, nach der gleichen Säurezugabe zeigen?
\end{tasks}
\end{question}
\begin{solution}
  \begin{tasks}
    \task Nach der Henderson"=Hasselbalch"=Gleichung ergibt sich ein \pH{} von
      $3.9$.
    \task Da Ameisensäure eine schwache Säure ist, ist die Mischung
      \ch{HCOOH}\slash\ch{HCOO-} ein Puffer.  Die Base ist in leichtem
      Überschuß vorhanden, also puffert die Lösung besser gegen Säuren.
    \task Die neuen Konzentrationen betragen $[\ch{HCOOH}] =
      (0.2\cdot0.1+0.01\cdot0.1)/0.11 = 0.19$ und $[\ch{HCOO-}] =
      (0.3\cdot0.1-0.01\cdot0.1)/0.11 = 0.26$, jeweils in
      \si{\mole\per\liter}.  Damit ergibt sich ein \pH{} von $3.9$.
    \task Ein \pH{} von $3.9$ enstpricht einer Oxonium"=Konzentration von
      \SI{1.3e-4}{\mole\per\liter} und im vorgegebenen Volumen einer
      Stoffmenge von \SI{1.3e-5}{\mole}.  Es werden \SI{0.001}{\mole}
      Oxonium"=Ionen hinzugegeben, was ein deutlicher Überschuß ist.  Damit
      kann die AUsgangskonzentration vernachlässigt werden und mit der
      Konzentration von \SI{0.009}{\mole\per\liter} ergibt sich ein \pH{} von
      $2.0$.
  \end{tasks}
\end{solution}

\begin{question}
Gegeben ist eine Lösung von \SI{1}{\gram} \ch{NH3} und \SI{5}{\gram}
\ch{NH4Cl} in \SI{500}{\milli\liter} Lösung.
\begin{tasks}
  \task Berechnen Sie den \pH-Wert der Lösung.
  \task Welchen \pH-Wert zeigt diese Lösung nach Zugabe von \SI{1}{\gram}
    \ch{NaOH} (keine Volumenänderung)?
\end{tasks}
\end{question}
\begin{solution}
  \begin{tasks}
    \task \SI{1}{\gram} Ammoniak entspricht einer Stoffmenge von
      \SI{0.059}{\mole} und damit einer Konzentration von
      \SI{0.12}{\mole\per\liter}.  \SI{5}{\gram} Ammoniumchlorid entsprechen
      einer Stoffmenge von \SI{0.019}{\mole} und einer Konzentration von
      \SI{0.037}{\mole\per\liter}.  Damit ergibt sich ein \pH{} von $9.75$.
    \task \SI{1}{\gram} Natriumhydroxid entspricht \SI{0.025}{\mole}.  Die
      neuen Konzentrationen betragen also $[\ch{NH4+}] = (0.037-0.025)/0.5 =
      0.024$ und $[\ch{NH3}] = (0.12+0.025)/0.5 = 0.29$, jeweils in
      \si{\mole\per\liter}.  Der \pH{} beträgt dann $10.3$.
  \end{tasks}
\end{solution}

\begin{question}
Gegeben ist eine Lösung von \SI{1}{\gram} \ch{HCl} und \SI{1}{\gram} \ch{NaCl}
in \SI{500}{\milli\liter} Lösung.  Welchen \pH-Wert zeigt diese Lösung nach
Zugabe von \SI{10}{\gram} \ch{NaOH} (keine Volumenänderung)?
\end{question}
\begin{solution}
  Auch wenn die Zusammensetzung der Lösung nach einem Puffer aussieht, ist sie
  keiner (siehe auch die Lösung zu Übung~\QuestionNumber{HCl/NaCl}).  Lediglich
  die Salzsäure muss zur Berechnung des \pH{} berücksichtigt werden.
  \SI{1}{\gram} \ch{HCl} entsprechen \SI{0.027}{\mole}.  Das ergibt eine
  Konzentration von \SI{0.056}{\mole\per\liter}, also beträgt der $\pH = 1.3$.
  \SI{10}{\gram} \ch{NaOH} entsprechen \SI{0.250}{\mole}.  Nach Neutralisation
  verbleiben also \SI{0.223}{\mole} \ch{NaOH}, also eine Konzentration von
  \SI{0.446}{\mole\per\liter}.  Der \pOH{} ist dann $0.4$, der \pH{} beträgt
  $13.6$.
\end{solution}

\begin{question}
Gegeben ist eine Lösung von \SI{500}{\milli\liter} aus \SI{0.1}{\Molar}
Natriumdihydrogenphosphat und \SI{0.1}{\Molar} Dinatriumhydrogenphosphat.
\begin{tasks}
  \task Berechnen Sie den \pH-Wert der Lösung.
  \task Ist diese Lösung ein Puffer? Wenn ja, puffert sie besser gegen Säuren
    oder Basen?
  \task Welchen \pH-Wert zeigt diese Lösung nach Zugabe von
    \SI{10}{\milli\liter} \SI{0.1}{\Molar} \ch{HCl}?
  \task Welchen \pH-Wert würde eine Lösung, die ungepuffert ist und den
    gleichen \pH-Wert hat, nach der gleichen Säurezugabe zeigen?
\end{tasks}
\end{question}
\begin{solution}
  \begin{tasks}
    \task Da die Konzentrationen im Verhältnis $1:1$ vorliegen, entspricht der
      \pH{} dem \pKa{} der Säure, es ist also $\pH = 7.2$.
    \task Die Mischung aus Dihydrogenphosphat und Hydrogenphosphat ist ein
      Puffer.  Da die Konzentrationen im Verhältnis $1:1$ vorliegen, puffert
      sie gleichermaßen gegen Säuren und Basen.
    \task Nach Säurezugabe ergeben sich die Konzentrationen $[\ch{H2PO4-}] =
      (0.1\cdot0.5 + 0.01\cdot0.1)/0.51 = 0.1$ und $[\ch{HPO4^2-}] =
      (0.1\cdot0.5 - 0.01\cdot0.1)/0.51 = 0.096$, also hat die Lösung den $\pH
      = 7.2$.
    \task Mit der gleichen Argumentation wie in der Lösung zu
      Übung~\QuestionNumber{HCOOH}d kann die zu Beginn vorliegende Menge an
      Oxonium"=Ionen vernachlässigt werden.  Mit der Konzentration
      $[\ch{H3O+}] = \SI{0.002}{\mole\per\liter}$ ergibt sich also ein \pH{}
      von $2.7$.
  \end{tasks}
\end{solution}

\begin{question}
Gegeben sind \SI{10}{\milli\liter} einer \SI{0.1}{\Molar} \ch{HF}.  Diese wird
mit \SI{0.1}{\Molar} \ch{NaOH} titriert.
\begin{tasks}
  \task Berechnen Sie den \pH-Wert nach Zugabe von
      \SIlist{0;1;5;9;10;10.5;12;14}{\milli\liter} Zugabe der \ch{NaOH}.
  \task Skizzieren Sie den Verlauf der \pH-Kurve.
\end{tasks}
\end{question}
\begin{solution}
  \begin{tasks}
    \task Zur Lösung der Aufgabe werden wir folgende Annahmen machen: 1. \ch{HF}
      ist eine schwache Säure, es gilt $\pH=0.5(\pKa-\log[\ch{HF}])$ für die
      reine Säurelösung.  Die analoge Annahme soll für die reine
      Fluorid-Lösung gelten  2. Die Zugabe von Natronlauge führt vollständig
      zur Bildung von \ch{F-} und solange \ch{HF} nicht vollständig
      neutralisiert wurde, gilt die Henderson"=Hasselbalch"=Gleichung.
      3. Nach der vollständigen Neutralisierung kann \ch{F-} vernachlässigt
      werden und nur noch die überschüssige \ch{NaOH} trägt zum \pH-Wert bei.
      Damit ergeben sich folgende Werte:
      \begin{center}
        \begin{tabular}{l*{8}{S}}
          \toprule
            $V(\ch{NaOH})$ in \si{\milli\liter}
                &   0 &   1 &   5 &   9 &  10 & 10.5 &   12 &   14 \\
            \pH & 2.1 & 2.2 & 3.1 & 4.1 & 7.9 & 11.4 & 12.0 & 12.2 \\
          \bottomrule
        \end{tabular}
      \end{center}
    \task
      \begin{tikzpicture}[baseline]
        \begin{axis}[
          anchor = north ,
          xlabel = {$V$ zugegebene \ch{NaOH} in \si{\milli\liter}} ,
          ylabel = {\pH}]
         \addplot[color=red,mark=x,smooth] coordinates {
            (0,2.1)
            (1,2.2)
            (5,3.1)
            (9,4.1)
            (10,7.9)
           (10.5,11.4)
           (12,12.0)
           (14,12.2)
         };
        \end{axis}
       \end{tikzpicture}
  \end{tasks}
\end{solution}

\begin{question}
Gegeben sind \SI{10}{\milli\liter} einer \SI{0.1}{\Molar} \ch{NH3}.  Diese
wird mit \SI{0.1}{\Molar} \ch{HCl} titriert.
\begin{tasks}
  \task Berechnen Sie den \pH-Wert nach Zugabe von
    \SIlist{0;1;5;9;10;10.5;12;14}{\milli\liter} Zugabe der \ch{HCl}.
  \task Skizzieren Sie den Verlauf der Titrationskurve.
\end{tasks}
\end{question}
\begin{solution}
  \begin{tasks}
    \task Zur Lösung der Aufgabe werden wir folgende Annahmen machen: 1. \ch{NH3}
      ist eine schwache Base, es gilt $\pOH=0.5(\pKb-\log[\ch{NH3}])$ für die
      reine Basenlösung.  Die analoge Annahme soll für die reine
      Ammonium-Lösung gelten  2. Die Zugabe von Salzsäure führt vollständig
      zur Bildung von \ch{NH4+} und solange \ch{NH3} nicht vollständig
      neutralisiert wurde, gilt die Henderson"=Hasselbalch"=Gleichung.
      3. Nach der vollständigen Neutralisierung kann \ch{NH4+} vernachlässigt
      werden und nur noch die überschüssige \ch{HCl} trägt zum \pH-Wert bei.
      Damit ergeben sich folgende Werte:
      \begin{center}
        \begin{tabular}{l*{8}{S}}
          \toprule
            $V(\ch{HCl})$ in \si{\milli\liter}
                &    0 &    1 &   5 &   9 &  10 & 10.5 &  12 &  14 \\
            \pH & 11.1 & 10.2 & 9.2 & 8.3 & 5.3 &  2.6 & 2.0 & 1.8 \\
          \bottomrule
        \end{tabular}
      \end{center}
    \task
      \begin{tikzpicture}[baseline]
        \begin{axis}[
          anchor = north ,
          xlabel = {$V$ zugegebene \ch{HCl} in \si{\milli\liter}} ,
          ylabel = {\pH}]
         \addplot[color=red,mark=x,smooth] coordinates {
            (0,11.1)
            (1,10.2)
            (5,9.2)
            (9,8.3)
            (10,5.3)
           (10.5,2.6)
           (12,2.0)
           (14,1.8)
         };
        \end{axis}
       \end{tikzpicture}
  \end{tasks}
\end{solution}

\begin{table}
  \centering
  \begin{tabular}{lCCSS}
    \toprule
      & {\bfseries{}Säure} & {\bfseries{}Base} & {\bfseries\Ka} & {\bfseries\pKa} \\
    \midrule
      Chlorwasserstoff   & HCl     & Cl-     &    1e6   & -6 \\
      Oxonium            & H3O+    & H2O     &   55     & -1.74 \\
      Schwefelsäure      & H2SO4   & HSO4-   &    1e3   & -3 \\
      Salpetersäure      & HNO3    & NO3-    &   21     & -1.32 \\
      Hydrogensulfat     & HSO4-   & SO4^2-  &  1.3e-2  &  1.92 \\
      Phosphorsäure      & H3PO4   & H2PO4-  &  1.1e-2  &  1.96 \\
      Fluorwasserstoff   & HF      & F-      & 7.25e-4  &  3.14 \\
      Methansäure        & HCOOH   & HCOO-   &  1.7e-4  &  3.77 \\
      Ethansäure         & CH3COOH & CH3COO- & 1.75e-5  &  4.76 \\
      Kohlensäure        & H2CO3   & HCO3-   &  3.5e-7  &  6.46 \\
      Dihydrogenphosphat & H2PO4-  & HPO4^2- &  6.2e-8  &  7.21 \\
      Ammonium           & NH4+    & NH3     &  5.8e-10 &  9.24 \\
      Hydrogencarbonat   & HCO3-   & CO3^2-  &    4e-11 & 10.4 \\
      Hydrogenphosphat   & HPO4^2- & PO4^3-  &  4.8e-13 & 12.32 \\
      Wasser             & H2O     & OH-     &  1.8e-16 & 15.74 \\
      Ammoniak           & NH3     & NH2-    &    1e-23 & 23 \\
      Hydroxid           & OH-     & O^2-    &    1e-24 & 24 \\
    \bottomrule
  \end{tabular}
  \caption{\pKa-Werte einiger Säuren}
  \label{tab:pKs-werte}
\end{table}

\newpage
\addsec{Lösungen}
In den Lösungen ist die Kenntnis der Henderson"=Hasselbalch"=Gleichung
vorausgesetzt und sie wird nicht explizit hingeschrieben.  Zur Erinnerung:
\[ \pH = \pKa - \log\biggl(\frac{[\text{Säure}]}{[\text{Base}]}\biggr) \]
Desweiteren wird angenommen, dass Sie die Definition des \pH-Wertes kennen und
anwenden können ($\pH = -\log[\ch{H3O+}]$).  Außerdem wird die entsprechende
Definition des \pOH{} ($\pOH = -\log[\ch{OH-}]$) und die Beziehung $\pH + \pOH
= 14$ als bekannt angenommen.  Ebenso wird angenommen, dass Sie zwischen
schwachen und starken Säuren unterscheiden können und wissen, wie man im
jeweiligen Fall den \pH-Wert berechnen kann, nämlich mit $\pH =
\frac{1}{2}\cdot(\pKa - \log[\ch{HA}])$ bzw. $\pH = -\log[\ch{HA}]$.   Und zu
guter Letzt wird vorausgesetzt, dass Sie mit Stoffmengen, Massen, Dichten und
Konzentrationen umgehen können, also die Gleichungen $\varrho = \frac{m}{V}$,
$M = \frac{m}{n}$ und $c = \frac{n}{V}$ (mit $\varrho$~= Dichte, $m$~= Masse,
$M$~= molare Masse, $n$~= Stoffmenge, $V$~= Volumen und $c$~= Konzentration)
kennen und anwenden können.

Da das hier chemische Übungen sind, wird die benötigte Mathematik
stillschweigend vorausgesetzt.  Wenn Sie Schwierigkeiten mit dem Umstellen von
Gleichungen und der Funktionsweise von Logarithmen haben, sollten Sie Sich
dazu Übungen besorgen.  Sonst fällt Ihnen das nachvollziehen des einen oder
anderen Schritts in der Lösung vermutlich schwerer als nötig.

\bigskip

\printsolutions

\end{document}
