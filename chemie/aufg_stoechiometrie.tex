\documentclass[DIV11,bib=totoc]{scrartcl}

\usepackage[T1]{fontenc}
\usepackage[utf8]{inputenc}
\usepackage[supstfm=libertinesups]{superiors}
\usepackage{libertine}
\usepackage{microtype}
\usepackage[ngerman]{babel}
\usepackage{scrpage2}
\clearscrheadfoot
\pagestyle{scrheadings}
\chead{Seite \thepage}
\cfoot{\small C.\,Niederberger -- aktualisiert am \today}

\usepackage{upgreek}
\usepackage{chemmacros}
\chemsetup{
  option/language         = german,
  chemformula/name-format = \centering ,
%  phases/pos              = sub
}
\renewrobustcmd*\mch[1][]{\ch{^{#1}-}}
\renewrobustcmd*\pch[1][]{\ch{^{#1}+}}

\usepackage{siunitx}
\sisetup{
%  per-mode = fraction ,
  per-mode            = symbol ,
  fixed-exponent      = 0 ,
  scientific-notation = fixed ,
  exponent-product    = \cdot ,
  group-digits        = false
}

\usepackage{chemfig}
\renewcommand*\printatom[1]{#1}
\setatomsep{1.8em}

\usepackage{biblatex}
\addbibresource{\jobname.bib}
\usepackage{filecontents}

\begin{filecontents}{\jobname.bib}
@book{pse_viley,
  author    = {Fluck and Heumann},
  title     = {Das Periodensystem der Elemente},
  edition   = {3. Auflage},
  year      = {2002},
  publisher = { VILEY-VCH Verlag}
}
\end{filecontents}

\usepackage[load-headings]{exsheets}
\SetupExSheets{
  totoc ,
  headings = block-rev
}
\usepackage{enumitem}

\usepackage{fnpct}
\usepackage{booktabs}

\usepackage[colorlinks]{hyperref}

\begin{document}

\begin{center}
  \Huge\sffamily Aufgaben Stöchiometrie
\end{center}

\addsec{Übungen}

\begin{question}
Wieviel Gramm Natriumamid (\ch{NaNH2}) und Distickstoffoxid (\ch{N2O}) werden
benötigt, um \SI{50.0}{g} Natriumazid (\ch{NaN3}) herzustellen bei der Annahme
eines vollständigen Stoffumsatzes gemäß
\begin{reaction*}
  2 NaNH2 + N2O -> NaN3 + NaOH + NH3 \,?
\end{reaction*}
\end{question}
\begin{solution}
\begin{tasks}[label=\empty](2)
  \task
  \begin{align*}
    m_{\ch{NaNH2}} &= m_{\ch{NaN3}} \cdot\frac{M_{\ch{NaNH2}}}{M_{\ch{NaN3}}} \\
      &= \SI{50.0}{g}\cdot\frac{\SI{39.0}{\gram\per\mole}}{\SI{65.0}{\gram\per\mole}} \\
      &= \SI{30.0}{g}
  \end{align*}
  \task
  \begin{align*}
    m_{\ch{N2O}} &= m_{\ch{NaN3}} \cdot\frac{M_{\ch{N2O}}}{M_{\ch{NaN3}}} \\
      &= \SI{50.0}{g}\cdot\frac{\SI{46.0}{\gram\per\mole}}{\SI{65.0}{\gram\per\mole}} \\
      &= \SI{35.4}{g}
  \end{align*}
\end{tasks}
\end{solution}

\begin{question}
Bei der Umsetzung von \ch{P4O10} mit \ch{PCl5} entsteht \ch{POCl3} als
einziges Produkt.
\begin{tasks}
  \task Formulieren Sie die Reaktionsgleichung.
  \task Wieviel Mol \ch{POCl3} kann man aus \SI{1.00}{mol} \ch{PCl5} erhalten?
  \task Welche Masse \ch{PCl5} braucht man, um \SI{12.0}{g} \ch{POCl3}
    herzustellen?
  \task Welche Masse \ch{P4O10} braucht man zur Umsetzung mit \SI{7.50}{g}
    \ch{PCl5}?
\end{tasks}
\end{question}
\begin{solution}
  \begin{tasks}
    \task \ch{P4O10 + 6 PCl5 -> 10 POCl3}
    \task Aus \SI{1.00}{mol} \ch{PCl5} erhält man
      $\SI{1.00}{mol}\cdot\frac{10}{6} = \SI{1.67}{mol}$ \ch{POCl3}.
    \task \SI{12.0}{g} \ch{POCl3} entsprechen
      $\frac{\SI{12.0}{g}}{\SI{153.5}{g/mol}} = \SI{0.078}{mol}$.  Also
      benötigt man $\SI{0.078}{mol}\cdot\frac{6}{10}=\SI{0.047}{mol}$
      \ch{PCl5}.  Das entspricht $\SI{0.047}{mol}\cdot\SI{208.5}{g/mol} =
      \SI{9.8}{g}$.
    \task \SI{7.50}{g} \ch{PCl5} entsprechen
      $\frac{\SI{7.50}{g}}{\SI{208.5}{g/mol}} = \SI{0.0360}{mol}$.  Man
      benötigt dann $\SI{0.0360}{mol}\cdot\frac{1}{6}=\SI{6.00e-3}{mol}$
      \ch{P4O10}.  Das entspricht $\SI{6.00e-3}{mol}\cdot\SI{284}{g/mol} =
      \SI{1.70}{g}$.
  \end{tasks}
\end{solution}

\begin{question}
Wieviel Gramm Iodwasserstoff (\ch{HI}) entstehen aus \SI{5.00}{g} \ch{PI3} bei
der vollständigen Umsetzung gemäß
\begin{reaction*}
  PI3 + 3 H2O -> 3 HI + H3PO3 \,?
\end{reaction*}
\end{question}
\begin{solution}
\SI{5.00}{g} \ch{PI3} enstsprechen $\frac{\SI{5.00}{g}}{\SI{412}{g/mol}} =
\SI{12.1}{\milli\mole}$.  Also entstehen $\SI{12.1}{\milli\mole}\cdot3 =
\SI{36.4}{\milli\mole}$ \ch{HI}.  Das sind
$\SI{36.4e-3}{\mole}\cdot\SI{128}{\gram\per\mole} = \SI{4.66}{\gram}$.
\end{solution}

\begin{question}
Wieviel Gramm des \textcolor{blue}{blau}gedruckten Produktes können maximal bei
der Umsetzung folgender Mengen erhalten werden?
\begin{tasks}
 \task
   \begin{reaction*}
     !(\SI{9.00}{\gram})( CS2 ) + !(\SI{3.00}{\gram})( 2 NH3 ) ->
       @{format=\color{blue}} NH4NCS + H2S
    \end{reaction*}
 \task
   \begin{reaction*}
     !(\SI{2.50}{\gram})( 2 F2 ) + !(\SI{2.50}{\gram})( 2 NaOH ) ->
       @{format=\color{blue}} OF2 + 2 NaF + H2O
    \end{reaction*}
 \task
   \begin{reaction*}
     !(\SI{6.00}{\gram})( 3 SCl2 ) + !(\SI{3.50}{\gram})( 4 NaF ) ->
       @{format=\color{blue}} SF4 + S2Cl2 + 4 NaCl
    \end{reaction*}
 \task
   \begin{reaction*}
     !(\SI{2.650}{\gram})( 3 NaBH4 ) + !(\SI{4.560}{\gram})( 4 BF3 ) ->
       3 NaBF4 + 2 @{format=\color{blue}} B2H6
    \end{reaction*}
\end{tasks}
\end{question}
\begin{solution}
Es müssen die Stoffmengen, berücksichtigt um die stöchiometrischen Faktoren,
verglichen werden.
\begin{tasks}
  \task Das Gemisch enthält $\frac{m}{M} =
    \frac{\SI{9.00}{\gram}}{\SI{76.15}{\gram\per\mole}} = \SI{0.118}{\mole}$
    \ch{CS2} und $\frac{\SI{3.00}{\gram}}{\SI{17.04}{\gram\per\mole}} =
    \SI{0.176}{\mole} = 2\cdot\SI{0.0880}{\mole}$ \ch{NH3}.  Also wird die
    maximale Menge durch die Menge des \ch{NH3} bestimmt.  Es enstehen $M\cdot
    n = \SI{76.14}{\gram\per\mole}\cdot\SI{0.0880}{\mole} = \SI{6.70}{\gram}$
    \ch{NH4NCS}.
  \task Das Gemisch enthält
    $\frac{\SI{2.50}{\gram}}{\SI{38.00}{\gram\per\mole}} = \SI{0.0658}{\mole}
    = 2\cdot\SI{0.0329}{\mole}$ \ch{F2} und
    $\frac{\SI{2.50}{\gram}}{\SI{40.00}{\gram\per\mole}} = \SI{0.0625}{\mole}
    = 3\cdot\SI{0.0208}{\mole}$ \ch{NaOH}.  Also wird die maximale Menge durch
    die Menge des \ch{NaOH} bestimmt.  Es enstehen
    $\SI{54.00}{\gram\per\mole}\cdot\SI{0.0208}{\mole} = \SI{1.12}{\gram}$
    \ch{OF2}.
  \task Das Gemisch enthält
    $\frac{\SI{6.00}{\gram}}{\SI{102.97}{\gram\per\mole}} = \SI{0.0583}{\mole} =
    3\cdot\SI{0.0194}{\mole}$ \ch{SCl2} und
    $\frac{\SI{3.50}{\gram}}{\SI{41.99}{\gram\per\mole}} = \SI{0.0834}{\mole}
    = 4\cdot\SI{0.0208}{\mole}$ \ch{NaF}.  Also wird die maximale Menge durch
    die Menge des \ch{SCl2} bestimmt.  Es enstehen
    $\SI{108.07}{\gram\per\mole}\cdot\SI{0.0194}{\mole} = \SI{2.10}{\gram}$
    \ch{SF4}.
  \task Das Gemisch enthält
    $\frac{\SI{2.650}{\gram}}{\SI{37.833}{\gram\per\mole}} =
    \SI{0.07004}{\mole} = 3\cdot\SI{0.02335}{\mole}$ \ch{NaBH4} und
    $\frac{\SI{4.650}{\gram}}{\SI{67.805}{\gram\per\mole}} =
    \SI{0.06858}{\mole} = 4\cdot\SI{0.01714}{\mole}$ \ch{BF3}.  Also wird die
    maximale Menge durch die Menge des \ch{BF3} bestimmt.  Es enstehen
    $\SI{27.670}{\gram\per\mole}\cdot2\cdot\SI{0.01714}{\mole} =
    \SI{0.9485}{\gram}$ \ch{B2H6}.
\end{tasks}
\end{solution}

\begin{question}
Berechnen Sie die prozentuale Ausbeute des des \textcolor{blue}{blau}gedruckten
Produkts.  Der Reaktand ohne Mengenangabe ist im Überschuss vorhanden.
\begin{tasks}
 \task
   \begin{reaction*}
     !(\SI{5.00}{\gram})( 3 LiBH4 ) + 3 NH4Cl ->
       !(\SI{2.16}{\gram})( @{format=\color{blue}} B3N3H6 ) + 9 H2 + 3 LiCl
    \end{reaction*}
 \task
   \begin{reaction*}
     !(\SI{6.00}{\gram})( Ca3P2 ) + 6 H2O ->
       !(\SI{1.40}{\gram})( 2 @{format=\color{blue}} PH3 ) + 3 Ca(OH)2
    \end{reaction*}
\end{tasks}
\end{question}
\begin{solution}
\begin{tasks}
  \task \SI{5.00}{\gram} \ch{LiBH4} enstprechen $\frac{m}{M} =
    \frac{\SI{5.00}{\gram}}{\SI{21.784}{\gram\per\mole}} = \SI{0.2295}{\mole} =
    4\cdot\SI{0.0574}{\mole}$.  Also können maximal \SI{0.0574}{\mole}
    \ch{B3N3H6} erhalten werden.  Das entspricht $M\cdot n =
    \SI{80.502}{\gram\per\mole}\cdot\SI{0.0574}{\mole} = \SI{4.62}{\gram}$.
    Damit beträgt die Ausbeute $\frac{\SI{2.16}{\gram}}{\SI{4.62}{\gram}} =
    \num{0.468} = \SI{46.8}{\percent}$.
  \task \SI{6.00}{\gram} \ch{Ca3P2} enstprechen
    $\frac{\SI{6.00}{\gram}}{\SI{182.182}{\gram\per\mole}} =
    \SI{0.0329}{\mole}$.  Also können maximal $2\cdot\SI{0.0329}{\mole} =
    \SI{0.0658}{\mole}$ \ch{PH3} erhalten werden.  Das entspricht $M\cdot n =
    \SI{33.998}{\gram\per\mole}\cdot\SI{0.0658}{\mole} = \SI{2.24}{\gram}$.
    Damit beträgt die Ausbeute $\frac{\SI{1.40}{\gram}}{\SI{2.24}{\gram}} =
    \num{0.654} = \SI{65.4}{\percent}$.
\end{tasks}
\end{solution}

\begin{question}
\SI{7.69}{\gram} eines Gemisches von Calciumcarbid (\ch{CaC2}) und Calciumoxid
(\ch{CaO}) reagieren mit Wasser gemäß den folgenden Reaktionen:
\begin{reactions*}
  CaC2\sld{} + 2 H2O\lqd{} &-> Ca(OH)2\aq{} + C2H2\gas{} \\
  CaO\sld{} + H2O\lqd{} &-> Ca(OH)2\aq{}
\end{reactions*}
Nehmen Sie vollständigen Stoffumsatz an. Wieviel Prozent des Gemisches
bestehen aus \ch{CaC2}, wenn \SI{2.34}{\gram} \ch{C2H2} (Ethin) erhalten
werden?
\end{question}
\begin{solution}
\SI{2.34}{\gram} Ethin sind
$\frac{\SI{2.34}{\gram}}{\SI{26.04}{\gram\per\mole}} = \SI{90}{\milli\mole}$
Ethin.  Also wurden \SI{90}{\milli\mole} \ch{CaC2} eingesetzt.  Das sind
$\SI{90e-3}{\mole}\cdot\SI{64.10}{\gram\per\mole} = \SI{5.77}{\gram}$.  Das
Gemisch besteht damit zu $\frac{5.77}{7.69}=0.75=\SI{75}{\percent}$ aus
Calciumcarbid.
\end{solution}

\begin{question}
Welche Stoffmengenkonzentrationen haben folgende Lösungen?
\begin{tasks}(2)
  \task \SI{4.00}{\gram} \ch{NaOH} in \SI{250}{\milli\liter} Lösung
  \task \SI{13.0}{\gram} \ch{NaCl} in \SI{1.50}{\liter} Lösung
  \task \SI{10.0}{\gram} \ch{AgNO3} in \SI{350}{\milli\liter} Lösung
  \task \SI{94.5}{\gram} \ch{HNO3} in \SI{250}{\milli\liter} Lösung
  \task \SI{6.500}{\gram} \ch{KMnO4} in \SI{2.000}{\liter} Lösung
\end{tasks}
\end{question}
\begin{solution}
\begin{tasks}
  \task $n_{\ch{NaOH}} = \frac{\SI{4.00}{\gram}}{\SI{40.00}{\gram\per\mole}} =
    \SI{0.100}{\mole}$, $c =\frac{\SI{0.100}{\mole}}{\SI{0.250}{\liter}} =
    \SI{0.400}{\mole\per\liter}$
  \task $n_{\ch{NaCl}} = \frac{\SI{13.0}{\gram}}{\SI{58.44}{\gram\per\mole}} =
    \SI{0.222}{\mole}$, $c = \frac{\SI{0.222}{\mole}}{\SI{1.50}{\liter}} =
    \SI{0.148}{\mole\per\liter}$
  \task $n_{\ch{AgNO3}} = \frac{\SI{10.0}{\gram}}{\SI{169.88}{\gram\per\mole}} =
    \SI{58.9}{\milli\mole}$, $c = \frac{\SI{58.9e-3}{\mole}}{\SI{0.350}{\liter}}
    = \SI{0.168}{\mole\per\liter}$
  \task $n_{\ch{HNO3}} = \frac{\SI{94.5}{\gram}}{\SI{63.02}{\gram\per\mole}} =
    \SI{1.50}{\mole}$, $ c =\frac{\SI{1.50}\mole}{\SI{0.250}{\liter}} =
    \SI{6.00}{\mole\per\liter}$
  \task $n_{\ch{KMnO4}} = \frac{\SI{6.500}{\gram}}{\SI{158.032}{\gram\per\mole}}
    = \SI{41.13}{\milli\mole}$, $c =
    \frac{\SI{41.13e-3}{\mole}}{\SI{2.000}{\liter}} =
    \SI{0.02057}{\mole\per\liter}$
\end{tasks}
\end{solution}

\begin{question}
Wieviele Mole Substanz sind in folgenden Lösungen enthalten?
\begin{tasks}(2)
  \task \SI{1.20}{\liter} mit $c(\ch{Ba(OH)2}) = \SI{0.0500}{\mole\per\liter}$
  \task \SI{25.0}{\milli\liter} mit $c(\ch{H2SO4}) = \SI{6.00}{\mole\per\liter}$
  \task \SI{0.250}{\liter} mit $c(\ch{NaCl}) = \SI{0.100}{\mole\per\liter}$
\end{tasks}
\end{question}
\begin{solution}
\begin{tasks}
  \task $n = c\cdot V = \SI{0.0500}{\mole\per\liter}\cdot\SI{1.20}{\liter} =
    \SI{0.0600}\mole$
  \task $n = \SI{6.00}{\mole\per\liter}\cdot\SI{25.0e-3}{\liter} =
    \SI{0.150}{\mole}$
  \task $n = \SI{0.100}{\mole\per\liter}\cdot\SI{0.250}{\liter} =
    \SI{0.0250}{\mole}$
\end{tasks}
\end{solution}

\begin{question}
Welche Masse muss man einwiegen, um folgende Lösungen herzustellen?
\begin{tasks}(2)
  \task \SI{500.0}{\milli\liter} mit $c(\ch{KMnO4}) = \SI{0.02000}{\mole\per\liter}$
  \task \SI{2.000}{\liter} mit $c(\ch{KOH}) = \SI{1.500}{\mole\per\liter}$
  \task \SI{25.00}{\milli\liter} mit $c(\ch{BaCl2}) = \SI{0.2000}{\mole\per\liter}$
\end{tasks}
\end{question}
\begin{solution}
\begin{tasks}
  \task $n = c\cdot V = \SI{0.02000}{\mole\per\liter}\cdot\SI{0.5000}{\liter}
    = \SI{0.01000}{\mole}$,\\
    $m = M\cdot n = \SI{158.032}{\gram\per\mole}\cdot\SI{0.01000}{\mole} =
    \SI{1.5810}{\gram}$
  \task $n = \SI{1.500}{\mole\per\liter}\cdot\SI{2.000}{\liter} =
    \SI{3.000}{\mole}$,\\
    $m = \SI{56.105}{\gram\per\mole}\cdot\SI{3.000}{\mole} =
    \SI{168.32}{\gram}$
  \task $n = \SI{0.2000}{\mole\per\liter}\cdot\SI{25.00e-3}{\liter} =
    \SI{5.000e-3}{\mole}$,\\
    $m = \SI{208.233}{\gram\per\mole}\cdot\SI{5.000e-3}{\mole} =
    \SI{1.0412}{\gram}$
\end{tasks}
\end{solution}

\begin{question}
Wieviele Milliliter einer Lösung mit $c(\ch{KOH}) =
\SI{0.250}{\mole\per\liter}$ reagieren mit \SI{15.0}{\milli\liter} einer
Lösung mit $c(\ch{H2SO4}) = \SI{0.350}{\mole\per\liter}$ gemäß der Gleichung
\begin{reaction*}
  2 KOH\aq{} + H2SO4\aq{} -> K2SO4\aq{} + 2 H2O\lqd{}
\end{reaction*}
bei Annahme vollständigen Stoffumsatzes?
\end{question}
\begin{solution}
\begin{align*}
  n_{\ch{H2SO4}}
    &= c\cdot V = \SI{0.350}{\mole\per\liter}\cdot\SI{15.0e-3}{\liter}\\
    &= \SI{5.25e-3}{\mole}\\
  \Rightarrow\quad n_{\ch{KOH}}
    &= 2\cdot n_{\ch{H2SO4}} = 2\cdot\SI{5.25e-3}{\mole}\\
    &= \SI{10.5e-3}{\mole}\\
  V_{\ch{KOH}}
    &= \frac{n}{c} = \frac{\SI{10.5e-3}{\mole}}{\SI{0.250}{\mole\per\liter}}\\
    &= \SI{42.0}{\milli\liter}
\end{align*}
\end{solution}

\begin{question}
Wenn Phosphorsäure, \ch{H3PO4\aq{}}, im Überschuß zu \SI{125}{\milli\liter}
einer Lösung von Bariumchlorid, \ch{BaCl2\aq{}}, gegeben wird, scheiden sich
\SI{3.26}{\gram} \ch{Ba3(PO4)2\sld} aus.  Welche Stoffmengenkonzentration hat
die \ch{BaCl2}-Lösung?
\end{question}
\begin{solution}
\SI{3.26}{\gram} \ch{Ba3(PO4)2} entsprechen
$\frac{\SI{3.26}{\gram}}{\SI{601.93}{\gram\per\mole}} =
\SI{5.416}{\milli\mole}$.  Die Lösung enthielt damit
$2\cdot\SI{5.416}{\milli\mole} = \SI{10.83}{\milli\mole}$ \ch{BaCl2}.  Also
beträgt die Konzentration $c =
\frac{\SI{10.83}{\milli\mole}}{\SI{125}{\milli\liter}} =
\SI{0.0866}{\mole\per\liter}$.
\end{solution}

\begin{question}
Wenn Eisen-Pulver zu einer Silbersalz-Lösung gegeben wird, geht das Eisen in
Lösung und Silber scheidet sich aus:
\begin{reaction*}
 Fe\sld{} + 2 Ag+ \aq{} -> Fe^2+ \aq{} + 2 Ag\sld{} v
\end{reaction*}
Welche Masse \ch{Fe\sld} benötigt man mindestens, um alles Silber aus
\SI{2.00}{\liter} einer Lösung mit $c(\ch{Ag+}) =
\SI{0.0650}{\mole\per\liter}$ auszuscheiden?
\end{question}
\begin{solution}
\begin{align*}
  n_{\ch{Ag}}
    &= c\cdot V = \SI{0.0650}{\mole\per\liter}\cdot\SI{2.00}{\liter}\\
    &= \SI{0.130}{\mole}\\
  \Rightarrow\quad n_{\ch{Fe}}
    &= \frac{1}{2}\cdot n_{\ch{Ag}} = \frac{1}{2}\cdot\SI{0.130}{\mole}\\
    &= \SI{0.0650}{\mole}\\
  m_{\ch{Fe}}
    &= M\cdot n = \SI{55.85}{\gram\per\mole}\cdot\SI{0.0650}{\mole}\\
    &= \SI{3.63}{\gram}
\end{align*}
\end{solution}

\addsec{Lösungen}
Bei allen Aufgaben wurden die molaren Massen mit den Angaben des PSE des
WILEY-VCH-Verlages~\cite{pse_viley} berechnet.

Es bedeuten $M$:~molare Masse, $m$:~Masse, $n$:~Stoffmenge, $V$:~Volumen,
$c$:~Stoffmengenkonzentration.

\printsolutions

\printbibliography

\end{document}
