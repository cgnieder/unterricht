\documentclass[DIV11]{scrartcl}

\usepackage[T1]{fontenc}
\usepackage[utf8]{inputenc}
\usepackage[supstfm=libertinesups]{superiors}
\usepackage[utopia]{mathdesign}
\usepackage[oldstyle]{libertine}
\usepackage{microtype}

\usepackage[ngerman]{babel}
\usepackage{scrpage2}
\clearscrheadfoot
\pagestyle{scrheadings}
\chead{Seite \thepage}
\cfoot{\small C.\,Niederberger -- aktualisiert am \today}

\usepackage{upgreek}
\usepackage{chemmacros}
\chemsetup{
  option/language         = german,
  chemformula/name-format = \centering ,
  chemformula/format      = \libertineLF ,
%  phases/pos              = sub
}
\renewrobustcmd*\mch[1][]{\ch{^{#1}-}}
\renewrobustcmd*\pch[1][]{\ch{^{#1}+}}

\usepackage{siunitx}
\sisetup{
%  per-mode = fraction ,
  per-mode = symbol ,
  fixed-exponent = 0 ,
  scientific-notation = fixed ,
  exponent-product = \cdot
}

\usepackage{chemfig}
\makeatletter
\def\CF@node@content{%
  \expandafter\expandafter\expandafter
    \printatom\expandafter\expandafter\expandafter
      {\csname atom@\number\CF@cnt@atomnumber\endcsname}%
    \ensuremath{\CF@node@strut}%
}
\makeatother
\renewcommand*\printatom[1]{\ch{#1}}
\setatomsep{1.8em}

\usepackage{pgfplots,tikz}
\pgfplotsset{
  compat = 1.7 ,
  y tick label style = {
    /pgf/number format/.cd,
    fixed,
    fixed zerofill,
    precision = 2,
    /tikz/.cd} ,
  scaled ticks = false
}

\usepackage[load-headings]{exsheets}
\SetupExSheets{
  totoc ,
  headings = block-rev
}
\usepackage{enumitem}

\usepackage{fnpct}
\usepackage{booktabs}

\usepackage{graphicx}
\usepackage{wrapfig}

\usepackage{mdframed}
\newmdenv[linecolor=red!20, linewidth=1ex]{intermission}

\usepackage[colorlinks]{hyperref}

\hyphenation{Salz-säu-re}
\begin{document}
\begin{center}
  \Huge\sffamily Chemie Querbeet -- Klasse 10/11
\end{center}

\addsec{Übungen}

\begin{question}[name=Reaktionsgleichungen]
Vervollständigen Sie die folgenden Reaktionsgleichungen.
\begin{tasks}(2)
  \task \ch{Na + O2 -> Na2O}
  \task \ch{Al + Br2 -> AlBr3}
  \task \ch{H2 + O2 -> H2O}
  \task \ch{Mg(OH)2\sld{} ->[ H2O ] Mg^2+ \aq{} + OH- \aq{} }
  \task \ch{K2O + H2O -> KOH}
  \task \ch{CaCl2 + AgNO3 -> AgCl v + Ca(NO3)2}
  \task \ch{Fe + H2O -> Fe3O4 + H2}
  \task \ch{TiCl4 + H2O -> TiO2 + HCl}
  \task \ch{B2O3 + C + Cl2 -> BCl3 + CO}
\end{tasks}
\end{question}
\begin{solution}[name=Reaktionsgleichungen]
\begin{tasks}(2)
  \task \ch{4 Na + O2 -> 2 Na2O}
  \task \ch{2 Al + 3 Br2 -> 2 AlBr3}
  \task \ch{2 H2 + O2 -> 2 H2O}
  \task \ch{Mg(OH)2\sld{} ->[ H2O ] Mg^2+ \aq{} + 2 OH- \aq }
  \task \ch{K2O + H2O -> 2 KOH}
  \task \ch{CaCl2 + 2 AgNO3 -> 2  AgCl v + Ca(NO3)2}
  \task \ch{3 Fe + 4 H2O -> Fe3O4 + 4 H2}
  \task \ch{TiCl4 + 2 H2O -> TiO2 + 4 HCl}
  \task \ch{B2O3 + 3 C + 3 Cl2 -> 2 BCl3 + 3 CO}
\end{tasks}
\end{solution}

\begin{question}[name=Summenformeln]
Korrigieren Sie folgende Summenformeln. Benennen Sie die Substanz.
\begin{tasks}(6)
  \task \ch{AlO3}
  \task \ch{BaOH}
  \task \ch{NH2}
  \task \ch{Ca2CO3}
  \task \ch{CH5}
  \task \ch{Al(OH)2}
\end{tasks}
\end{question}
\begin{solution}[name=Summenformeln]
\begin{tasks}(3)
  \task Aluminiumoxid \ch{Al2O3}
  \task Bariumhydroxid \ch{Ba(OH)2}
  \task Ammoniak \ch{NH3}
  \task Calciumcarbonat \ch{CaCO3}
  \task Methan \ch{CH4}
  \task Aluminiumhydroxid \ch{Al(OH)3}
\end{tasks}
\end{solution}


\begin{question}[name=Methanol-Metabolismus]
Der Körper baut Methanol ab, indem er es oxidiert.  Dabei entsteht zunächst
giftiges Formaldehyd (= Methanal) und in einem weiteren Schritt die
hochgiftige Ameisensäure.
\begin{tasks}
  \task Geben Sie die Strukturformeln von Methanol. Formaldehyd und
    Ameisensäure an.  Wie lautet der systematische Name der Ameisensäure?
  \task Geben Sie eine Reaktionsgleichung für die Dehydrierung (= Entzug von
    Wasserstoff) von Methanol zu Methanal an.
  \task Geben Sie eine Reaktionsgleichung für die Oxidation von Methanol zu
    Ameisensäure an.
  \task \ghspic{skull} Ca. \SI{50}{g}\footnotemark{} Ameisensäure sind
    tödlich. Wieviel Gramm Methanol muss man dafür zu sich nehmen?
  \task \SI{1}{mL} Methanol wiegt \SI{0.79}{g}. Wieviel \si{mL} Methanol sind
    tödlich?   Ein Viertel dieser Menge kann bereits zur Erblindung führen.
\end{tasks}
\end{question}
\footnotetext{Wert bezieht sich auf einen \SI{70}{kg}  schweren Menschen. Die
  LD$_{50}$ beträgt \SI{730}{mg/kg}, getestet an Ratten.}

\begin{solution}[name=Methanol-Metabolismus]
\begin{tasks}
  \task Der Reihe nach: Methanol, Formaldehyd (Methanal) und Ameisensäure
    (Methansäure)\par
    \chemfig{H-C(-[2]H)(-[6]H)-OH} \quad \chemfig{H-[:30]C(=[2]O)-[:-30]H}
    \quad \chemfig{H-[:30]C(=[2]O)-[:-30]OH}
  \task \ch{CH3OH + 1/2 O2 ->[\text{Ag,~\SI{400}{\celsius}}] CH2O + H2O}
  \task Formal: \ch{CH3OH + O2 -> HCOOH + H2O}
  \task
    \begin{gather*}
      n(\ch{HCOOH})
        = \frac{m}{M}
        = \frac{\SI{50}{\gram}}{\SI{46}{\gram\per\mole}}
        = \SI{1.09}{\mole} \\
      m(\ch{CH3OH})
        = n\cdot M
        = \SI{1.09}{\mole}\cdot\SI{32}{\gram\per\mole}
        = \SI{34.9}{\gram}
    \end{gather*}
  \task
    \[
      V = \frac{m}{\varrho}
        = \frac{\SI{34.9}{g}}{\SI{0.79}{\gram\per\cmc}}
        = \SI{44.2}{\milli\liter}
    \]
\end{tasks}
\end{solution}

\begin{question}[name=Ester I]
Eines der Aromen der Ananas ist Butansäureethylester.
\begin{tasks}(2)
  \task Aus welchen beiden Stoffen könnte man diesen Ester im Labor
    herstellen?
  \task Geben Sie die Strukturformeln der Ausgangsstoffe und des Esters an.
\end{tasks}
\end{question}
\begin{solution}[name=Ester I]
\begin{tasks}
  \task Buttersäure und Ethanol (konz. Schwefelsäure als Katalysator)
  \task Buttersäure \chemfig{CH_3-CH_2-CH_2-C(=[2]O)-OH}\par
    Ethanol \chemfig{CH_3-CH_2-OH}\par
    Butansäureethylester \chemfig{CH_3-CH_2-CH_2-C(=[2]O)-O-CH_2-CH_3}
\end{tasks}
\end{solution}

\begin{question}[name=Ester II]
Als Rumaroma wird Methansäureethylester verwendet.
\begin{tasks}
  \task Geben Sie die Reaktionsgleichung für die Herstellung des Esters an.
  \task Wieviel Gramm Ethanol benötigt man zur Herstellung von \SI{500}{\gram}
    dieses Esters? Wieviel \si{\gram} Methansäure werden gebraucht?
\end{tasks}
\end{question}
\begin{solution}[name=Ester II]
\begin{tasks}
  \task \ch{C2H5OH + HCOOH ->[ \[H2SO4\] ] HCOOC2H5 + H2O}
  \task
    \begin{gather*}
      n(\text{Ester})
        = \frac{m}{M}
        = \frac{\SI{500}{\gram}}{\SI{74.09}{\gram\per\mole}}
        = \SI{6.75}{\mole} \\
      m(\ch{C2H5OH})
        = n \cdot M
        = \SI{6.75}{\mole}\cdot\SI{46.08}{\gram\per\mole}
        = \SI{311.0}{\gram} \\
      m(\ch{HCOOH})
        = n \cdot M
        = \SI{6.75}{\mole}\cdot\SI{46.03}{\gram\per\mole}
        = \SI{310.7}{\gram}
    \end{gather*}
\end{tasks}
\end{solution}

\begin{question}[name=Wasserstoffdarstellung]
Wenn man im Labor kleine Mengen Wasserstoff benötigt, setzt man ein Metall mit
Säure um und fängt das entstehende Gas auf.
\begin{tasks}
  \task Geben Sie eine Reaktionsgleichung für die Reaktion von Schwefelsäure
    mit Zink an.
  \task Wenn man \SI{100}{\gram} konzentrierte (= \SI{98}{\percent}-ige)
    Schwefelsäure mit einem Überschuß Zink versetzt, wird die Schwefelsäure
    nahezu vollständig reagieren. Wieviel Liter Wasserstoff entstehen dabei?

    Hinweis: \SI{1}{\mole} entspricht \SI{22.4}{\liter} (bei \SI{0}{\celsius}
    und \SI{1}{\atm}).
\end{tasks}
\end{question}
\begin{solution}[name=Wasserstoffdarstellung]
\begin{tasks}
  \task \ch{Zn + H2SO4 -> H2 ^ + ZnSO4}
  \task
    \[
      n = \frac{\SI{100}{\gram}\cdot 0.98}{\SI{98}{\gram\per\mole}}
      = \SI{1.0}{\mole}
      \quad \Rightarrow\quad
      V = \SI{22.4}{\liter}
    \]
\end{tasks}
\end{solution}

\begin{question}[name=Neutralisation]
Die Reaktion einer Säure mit einer Lauge nennt man Neutralisation.
\begin{tasks}
  \task Geben Sie die Neutralisationsreaktion von Essigsäure mit
    Natriumhydroxid an.
  \task Wieviel \si{\gram} Chlorwasserstoff benötigt man zur Neutralisation
    von \SI{120}{\gram} Natriumhydroxid?
  \task In Wasser gelöstes Chlorwasserstoffgas nennt man Salzsäure.
    Konzentrierte Salzsäure enthält etwa \SI{38}{\percent}
    Chlorwasserstoff. Wieviel \si{\milli\liter} Salzsäure benötigt man für b,
    wenn \SI{1}{\milli\liter} Salzsäure etwa \SI{1.2}{\gram} wiegt, wenn also
    die Dichte $\varrho = \SI{1.2}{\gram\per\milli\liter}$ beträgt? 
\end{tasks}
\end{question}
\begin{solution}[name=Neutralisation]
\begin{tasks}
  \task \ch{CH3COOH + NaOH <=> CH3COONa + H2O}
  \task
    \begin{gather*}
      n(\ch{NaOH}) = \frac{m}{M}
                   = \frac{\SI{120}{\gram}}{\SI{40}{\gram\per\mole}}
                   = \SI{3.0}{\mole} \\
      m(\ch{HCl})  = n \cdot M
                   = \SI{3.0}{\mole} \cdot \SI{36.45}{\gram\per\mole}
                   = \SI{109.4}{\gram}
    \end{gather*}
  \task
    \[
      V = \frac{m}{\varrho\cdot 0.38}
        = \frac{\SI{109.4}{\gram}}{\SI{1.2}{\gram\per\milli\liter}\cdot0.38}
        = \SI{240}{\milli\liter}
    \]
\end{tasks}
\end{solution}

\begin{question}[name=Verbrennungsreaktion]
Stellen Sie die Reaktionsgleichungen für die vollständige Verbrennung
folgender Stoffe auf:
\begin{tasks}(4)
  \task Cyclohexan
  \task Toluol (\ch{C7H8})
  \task Octan
  \task Propan
  \task Thiophen (\ch{C4H4S})
  \task Pyridin (\ch{C5H5N})
  \task Anilin (\ch{C6H7N})
  \task Butandisäure
\end{tasks}
\end{question}
\begin{solution}[name=Verbrennungsreaktion]
Verbrennung ist die Reaktion mit Sauerstoff.  Dabei entstehen bei vollständiger
Verbrennung hauptsächlich \ch{CO2} und \ch{H2O}.
\begin{tasks}
  \task \ch{C6H12 + 9 O2 -> 6 CO2 + 6 H2O}
  \task \ch{C7H8 + 9 O2 -> 7 CO2 + 4 H2O}
  \task \ch{2 C8H18 + 25 O2 -> 16 CO2 + 18 H2O}
  \task \ch{C3H8 + 7 O2 -> 3 CO2 + 4 H2O}
  \task \ch{C4H4S + 6 O2 -> 4 CO2 + 2 H2O + SO2}
  \task \ch{4 C5H5N + 29 O2 -> 20 CO2 + 10 H2O + 4 NO2}
  \task \ch{4 C6H7N + 35 O2 -> 24 CO2 + 14 H2O + 4 NO2}
  \task \ch{2 C4H6O4 + 7 O2 -> 8 CO2 + 6 H2O}
\end{tasks}
\end{solution}
\newpage

\begin{question}[name=Oxidation von Alkoholen]
\begin{wrapfigure}[3]{r}{38mm}
  \vskip-\baselineskip\relax
  \includegraphics{./170px-Potassium-chromate-sample.jpg}\hfil
  \includegraphics{./180px-Chromium(III)-oxide-sample.jpg}
\end{wrapfigure}
Versetzt man gelbes Chromat, das \ox[pos=side]{6,Cr} enthält, mit Alkohol,
wird das Chrom zu grünem \ox[pos=side]{3,Cr} reduziert.  Diese Reaktion
verwendete man früher in den "`Pusteröhrchen"' der Polizei zur
Alkoholkontrolle.
\begin{tasks}
  \task Was entsteht bei dieser Reaktion aus dem Trinkalkohol (= Ethanol)?
  \task Wieso würde diese Reaktion mit \iupac{\tert\-Butanol} (=
    \iupac{1,1\-Di\|methyl\|ethanol}) nicht funktionieren?
  \task Was würde aus Isopropanol (= \iupac{Propan\-2\-ol}) entstehen?
\end{tasks}
\end{question}
\begin{solution}[name=Oxidation von Alkoholen]
\begin{tasks}
  \task Im ersten Schritt wirk Ethanol zu Acetaldehyd (Ethanal) oxidiert, das
    schließlich zu Essigsäure (Ethansäure) oxidiert wird.
  \task \iupac{\tert\-Butanol} (\ch{\textit{t} BuOH}) hat kein
    \Chemalpha-\ch{H}-Atom und lässt sich daher nicht dehydrieren.
  \task Isopropanol würde zu Aceton (Propanon) oxidiert, das nicht weiter
    oxidiert werden kann, ohne dass die Struktur zerstört würde.
\end{tasks}
\end{solution}

\begin{question}[name=Bindungen I]
Entscheiden Sie, ob bei den folgenden Bindungen eine Elektronenpaarbindung
oder eine ionische Bindung vorliegt.  Benennen Sie die Stoffe.  Sortieren Sie
die Stoffe mit Elektronenpaarbindung nach steigender Polarität.
\begin{tasks}(8)
  \task \ch{NaCl}
  \task \ch{HCl}
  \task \ch{H2O}
  \task \ch{K2O}
  \task \ch{CaH2}
  \task \ch{BH3}
  \task \ch{NH3}
  \task \ch{MgBr2}
\end{tasks}
\end{question}
\begin{solution}[name=Bindungen I]
Es gilt als Faustregel: Bindungen mit $\Delta EN > 1.7$ sind ionisch, sonst
sind es Elektronenpaarbindungen. Weitere Faustregel: Metalle ragieren mit
Nichtmetallen zu Ionenverbindungen, Nichtmetalle mit Nichtmetallen zu
Elektronenpaarbindungen.
\begin{tasks}(2)
  \task \ch{NaCl}: ionische Bindung $\Delta EN=1.8$
  \task \ch{HCl}: Elektronenpaarbindung $\Delta EN=0.6$
  \task \ch{H2O}: Elektronenpaarbindung $\Delta EN=1.3$
  \task \ch{K2O}: ionische Bindung $\Delta EN=2.6$
  \task \ch{CaH2}: ionische Bindung $\Delta EN=1.2$ (Faustregel greift nicht,
    hier gilt: Metall"=Nichtmetall $\Rightarrow$ Salz)
  \task \ch{BH3}: Elektronenpaarbindung $\Delta EN=0.2$
  \task \ch{NH3}: Elektronenpaarbindung $\Delta EN=0.9$
  \task \ch{MgBr2}: ionische Bindung $\Delta EN=1.5$ (Faustregel greift nicht,
    hier gilt: Metall"=Nichtmetall $\Rightarrow$ Salz)
\end{tasks}
Moleküle sortiert nach steigender Polarität: $ \ch{BH3} < \ch{HCl} < \ch{NH3}
< \ch{H2O}$.
\end{solution}

\begin{question}[name=Bindungen II]
Welche Arten chemischer Bindung kennen Sie (drei
intramolekulare\footnote{innerhalb von Molekülen}, drei
intermolekulare\footnote{zwischen Molekülen})?  Beschreiben Sie sie und geben
Sie jeweils ein Beispiel an.
\end{question}
\begin{solution}[name=Bindungen II]
\begin{itemize}
  \item intramolekulare Bindungen
    \begin{itemize}
      \item Elektronenpaarbindung: gerichtete lokalisierte Bindung zwischen
        zwei Atomen, Atome bilden Moleküle; Beispiel: \ch{HCl}.
      \item Ionenbindung: ungerichtete Bindung durch elektrostatische
        Anziehung, Atome im Kristallgitter angeordnet; Beispiel: \ch{KBr}.
      \item metallische Bindung: Metallatome bilden Kugelpackung, Elektronen
        delokalisiert; Beispiel: \ch{Fe}.
    \end{itemize}
  \item intermolekulare Bindungen
    \begin{itemize}
      \item Van-der-Waals-Bindungen: induzierte Dipole, Anziehung zwischen
        unpolaren Stoffen; Beispiel: \ch{CH4}.
      \item Dipol-Dipol-Wechselwirkungen: permanente Dipole; Beispiel:
        \ch{H2S}.
      \item Wasserstoff-Brücken-Bindungen: der starke Dipol zwischen der
        \ch{O-H}- oder \ch{N-H}-Bindung und einem elektronegativen Atom,
        in der Regel \ch{O}; Beispiel: \ch{H2O}.
     \end{itemize}
\end{itemize}
\end{solution}

\begin{question}[name=Bindungen III]
Bindungen in der organischen Chemie.
\begin{tasks}
  \task Erläutern Sie die Begriffe Einfachbindung, Doppelbindung und
    Dreifachbindung.  Ordnen Sie sie den gesättigten und ungesättigten
    Kohlenwasserstoffen zu.
  \task Um welche Bindungsart handelt es sich bei den \ch{C-C}-Einfach- und
    den \ch{C-C}"=Mehrfachbindungen?  Begründen Sie.
\end{tasks}
\end{question}
\begin{solution}[name=Bindungen III]
\begin{tasks}
  \task Einfachbindung: eine kovalente Bindung zwischen zwei Atomen;
    gesättigte Kohlenwasserstoffe (eine \Chemsigma-Bindung). \par
    Doppelbindung: zwei kovalente Bindungen zwischen denselben Atomen;
    ungesättigte Kohlenwasserstoffe (eine \Chemsigma-Bindung und eine
    \Chempi-Bindung). \par
    Dreifachbindung: drei kovalente Bindungen zwischen denselben Atomen;
    ungesättigte Kohlenwasserstoffe (eine \Chemsigma-Bindung und zwei
    \Chempi-Bindungen).
  \task Elektronenpaarbindungen, eine \Chemsigma-Bindung, zusätzliche
    Bindungen sind \Chempi"=Bindungen.
\end{tasks}
\end{solution}

\begin{question}[name={Säuren, Basen und Salze I}]
Benennen Sie folgende Salze und ordnen Sie ihnen die Säure zu, aus denen sie
sich darstellen lassen.  Geben Sie die Säure mit Namen und Summenformel
an.  Geben Sie für drei Salze Ihrer Wahl je drei Darstellungsmethoden mit
Reaktionsgleichung an.
\begin{tasks}(8)
  \task \ch{Na2SO4}
  \task \ch{CaCO3}
  \task \ch{K3PO4}
  \task \ch{NaNO3}
  \task \ch{LiCl}
  \task \ch{Na2SO3}
  \task \ch{CaS}
  \task \ch{KNO2}
\end{tasks}
\end{question}
\begin{solution}[name={Säuren, Basen und Salze I}]
\begin{tasks}
  \task Natriumsulfat, Salz der Schwefelsäure \ch{H2SO4}.\\
    Bildungs-Methode Metall und Säure: \ch{2 Na + H2SO4 -> Na2SO4 + H2 ^}.\\
    Bildungs-Methode Metallhydroxid und Säure: \ch{2 NaOH + H2SO4 -> 2 H2O +
      Na2SO4}.\\
    Bildungs-Methode Metalloxid und Säure: \ch{Na2O + H2SO4 -> Na2SO4 + H2O}.
  \task Calciumcarbonat, Salz der Kohlensäure \ch{H2CO3}.\\
    Bildungsmethode Metall und Säure: \ch{Ca + H2CO3 -> CaCO3 + H2 ^}.\\
    Bildungsmethode Metallhydroxis und Säure: \ch{Ca(OH)2 + H2CO3 -> CaCO3 + 2
      H2O}.\\
    Bildungsmethode Metalloxid und Säure: \ch{CaO + H2CO3 -> CaCO3 + H2O}.
  \task Kaliumphosphat, Salz der Phosphorsäure \ch{H3PO4}.\\
    Bildungsmethode Metall und Säure: \ch{6 K + 2 H3PO4 -> 2 K2PO4 + 3 H2 ^}.\\
    Bildungsmethode Metallhydroxid und Säure: \ch{3 KOH + H3PO4 -> K3PO4 + 3
      H2O}.\\
    Bildungsmethode Metalloxid und Säure: \ch{3 K2O + 2 H3PO4 -> 2 K3PO4 + 3
      H2O}.
  \task Natriumnitrat, Salz der Salpetersäure \ch{HNO3}.\\
    Bildungsmethode Metall und Säure: \ch{2 Na + 2 HNO3 -> 2 NaNO3 + H2 ^}.\\
    Bildungsmethode Metallhydroxid und Säure: \ch{NaOH + HNO3 -> NaNO3 +
      H2O}.\\
    Bildungsmethode Metalloxid und Säure: \ch{Na2O + 2 HNO3 -> 2 NaNO3 +
      H2O}.
  \task Lithiumchlorid, Salz der Salzsäure \ch{HCl}.\\
    Bildungsmethode Metall und Säure: \ch{2 Li + 2 HCl -> 2 LiCl + H2 ^}.\\
    Bildungsmethode Metallhydroxid und Säure: \ch{LiOH + HCl -> LiCl + H2O}.\\
    Bildungsmethode Metalloxid und Säure: \ch{Li2O + 2 HCl -> 2 LiCl + H2O}.
  \task Natriumsulfit, Salz der schwefligen Säure \ch{H2SO3}.\\
    Bildungsmethode Metall und Säure: \ch{2 Na + H2SO3 -> Na2SO3 + H2 ^}.\\
    Bildungsmethode Metallhydroxid und Säure: \ch{2 NaOH + H2SO3 -> Na2SO3 + 2
      H2O}.\\
    Bildungsmethode Metalloxid und Säure: \ch{Na2O + H2SO3 -> Na2SO3 + H2O}.
  \task Calciumsulfid, Salz der Schwefelwasserstoffsäure \ch{H2S}.\\
    Bildungsmethode Metall und Säure: \ch{Ca + H2S -> CaS + H2 ^}.\\
    Bildungsmethode Metallhydroxid und Säure: \ch{Ca(OH)2 + H2S -> CaS + 2
      H2O}.\\
    Bildungsmethode Metalloxid und Säure: \ch{CaO + H2S -> CaS + H2O}.
  \task Kaliumnitrit, salz der salpetrigen Säure \ch{HNO2}.\\
    Bildungsmethode Metall und Säure: \ch{2 K + 2 HNO2 -> 2 KNO2 + H2 ^}.\\
    Bildungsmethode Metallhydroxid und Säure: \ch{KOH + HNO2 -> KNO2 + H2O}.\\
    Bildungsmethode Metalloxid und Säure: \ch{K2O + 2 HNO2 -> 2 KNO2 + H2O}.
\end{tasks}
\end{solution}

\begin{question}[name={Säuren, Basen und Salze II}]
Stellen Sie die Dissoziationsgleichungen für folgende Stoffe auf:
\begin{tasks}(4)
  \task Salzsäure
  \task Schwefelsäure
  \task Natriumhydroxid
  \task Magnesiumsulfat
  \task Kaliumchlorid
  \task Natriumsulfat
\end{tasks}
\end{question}
\begin{solution}[name={Säuren, Basen und Salze II}]
\begin{tasks}
  \task \ch{HCl\aq{} <=> H+ \aq{} + Cl- \aq}
  \task \ch{H2SO4\aq{} <=> H+ \aq{} + HSO4- \aq{} <=> 2 H+ \aq{} + SO4^2- \aq}
  \task \ch{NaOH\aq{} <=> OH- \aq{} + Na+ \aq}
  \task \ch{MgSO4\aq{} <=> Mg^2+ \aq{} + SO4^2- \aq}
  \task \ch{KCl\aq{} <=> K+ \aq{} + Cl- \aq}
  \task \ch{Na2SO4\aq{} <=> 2 Na+ \aq{} + SO4^2- \aq}
\end{tasks}
\end{solution}

\begin{question}[name=Strukturformeln I]
Geben Sie die Strukturformeln folgender Stoffe an:
\begin{tasks}(4)
  \task Hexan
  \task \iupac{3\-Methyl\|pentan}
  \task \iupac{2\-Chlor\|heptan}
  \task \iupac{2,3,3\-Tri\|brom\|octan}
\end{tasks}
\end{question}
\begin{solution}[name=Strukturformeln I]
\begin{tasks}
  \task \chemfig{CH_3-CH_2-CH_2-CH_2-CH_2-CH_3}
  \task \chemfig{CH_3-CH_2-CH(-[2]CH_3)-CH_2-CH_3}
  \task \chemfig{CH_3-CH(-[2]Cl)-CH_2-CH_2-CH_2-CH_2-CH_3}
  \task \chemfig{CH_3-CH(-[2]Br)-C(-[2]Br)(-[6]Br)-CH_2-CH_2-CH_2-CH_2-CH_3}
\end{tasks}
\end{solution}

\begin{question}[name=Strukturformeln II]
Geben Sie die Strukturformeln aller isomeren Pentanole\footnote{Ohne Beachten
  der R/S-Isomerie.} an.  Benennen Sie sie mit ihrem systematischen Namen und
teilen Sie die Isomere in primäre, sekundäre und tertiäre Alkohole ein.
\end{question}
\begin{solution}[name=Strukturformeln II]
Zunächst die Strukturformeln aller Isomere:

\chemfig{[:30]--[:-30]--[:-30]-OH}\quad
\chemfig{[:30]--[:-30]-(-[2])-[:-30]OH}\quad
\chemfig{[:30]--[:-30](-[6]-[:-30])-OH}\quad
\chemfig{[:30]--[:-30](-[6])--[:-30]OH}

\chemfig{[:30]-(-[2])-[:-30]--[:-30]OH}\quad
\chemfig{[:30]--[:-30](-[:-70])(-[:-110])-OH}\quad
\chemfig{[:30]-(-[2])-[:-30](-[6])-OH}\quad
\chemfig{[:30]-(-[:70])(-[:110])-[:-30]-OH}

\noindent
Nun der Reihe nach die Namen und die Klassifizierung in primär, sekundär und
tertiär:

\iupac{1\-Pentanol} (primär); \iupac{2\-Pentanol} (sekundär);
\iupac{3\-Pentanol} (sekundär); \iupac{2\-Methyl\-1\-butanol} (primär);

\iupac{3\-Methyl\-1\-butanol} (primär); \iupac{2\-Methyl\-2\-butanol}
(tertiär); \iupac{3\-Methyl\-2\-butanol} (sekundär);
\iupac{2,2\-Di\|methyl\|propanol} (primär); 
\end{solution}

\begin{question}[name=Strukturformeln III -- Nomenklatur]
Geben Sie die systematischen Namen\footnote{Ohne Beachten der absoluten
  Konfiguration (R/S-Nomenklatur)} -- gegebenenfalls auch den Trivialnamen --
der folgenden Verbindungen an.
\begin{tasks}(2)
  \task \chemfig{[:30]--[:-30]([6])-}
  \task \chemfig{[:30]O=-[:-30]-}
  \task \chemfig{CH_3-CH(-[2]OH)-CH_2-CH_2-CH_3}
  \task \chemfig{[:30]HO--[:-30]-OH}
  \task \chemfig{[:30]--[:-30]-(=[2]O)-[:-30]OH}
  \task \chemfig{CH_3-CH_2-O-CH_3}
  \task \chemfig{CH_3-CH_2-COO-C_4H_9}
  \task \chemfig{[:30]-(=[2]O)-[:-30]}
  \task \chemfig{[:30]--[:-30]-NH_2}
  \task \chemfig{[:30]--[:-30]--[:-30](-[6]SH)--[:-30]}
\end{tasks}
\end{question}
\begin{solution}[name=Strukturformeln III - Nomenklatur]
\begin{tasks}(3)
  \task \iupac{2\-Methyl\|butan}
  \task Propanal (Propionaldehyd)
  \task \iupac{Pentan\-2\-ol}
  \task \iupac{Ethan\|di\|ol} (Glykol)
  \task Butansäure (Buttersäure)
  \task \iupac{Ethyl\|methyl\|ether}
  \task \iupac{Propansäure\|butyl\|ester}
  \task Propanon (Aceton)
  \task \iupac{1\-Amino\|propan}
  \task \iupac{Heptan\-3\-thiol}
\end{tasks}
\end{solution}

\begin{question}[name=Organische Reaktionen]
Entscheiden Sie, welches der folgenden Gemische reagieren kann.  Geben Sie
gegebenenfalls die Reaktionsgleichung und den Reaktionstyp an:
\begin{tasks}(3)
  \task Butan/Wasserstoff
  \task \iupac{2\-Buten}/Chlor
  \task Butan/Chlor
  \task \iupac{1\-Buten}/Chlorwasserstoff
  \task \iupac{2\-Butin}/Wasserstoff
  \task \iupac{1\-Butin}/\iupac{1\-Butin}
\end{tasks}
\end{question}
\begin{solution}[name=Organische Reaktionen]
\begin{tasks}
  \task keine Reaktion
  \task Addition: \ch{CH3-CH=CH-CH3 + Cl2 -> CH3-CHCl-CHCl-CH3}
  \task (radikalische) Substitution:
    \[
      \ch{C4H10 + Cl2 -> !(Nebenprodukt)( C2H5-CH2-CH2Cl )
        + !(Hauptprodukt)( C2H5-CHCl-CH3 ) + HCl}
    \]
  \task Addition:
    \[
      \ch{C2H5-CH=CH2 + HCl -> !(Nebenprodukt)( C2H5-CH2-CH2Cl )
        + !(Hauptprodukt)( C2H5-CHCl-CH3 )}
    \]
  \task Addition:
    \begin{gather*}
      \ch{CH3-C+C-CH3 + H2 -> CH3-CH=CH-CH3}\\
      \ch{CH3-C+C-CH3 + 2 H2 -> CH3-CH3-CH3-CH3}
    \end{gather*}
\end{tasks}
\end{solution}

\begin{question}[name=Siedepunkte und Struktur]
Ordnen Sie folgende Siedepunkte den Verbindungen Ethanal, Ethansäure, Ethanol,
Propan und Propansäure zu:

\SI{78}{\celsius}, \SI{97}{\celsius}, \SI{-42}{\celsius}, \SI{20}{\celsius},
\SI{118}{\celsius}

Begründen Sie Ihre Entscheidung.
\end{question}
\begin{solution}[name=Siedepunkte und Struktur]
\begin{itemize}
  \item \SI{-42}{\celsius}: Propan (kurzes unpolares Molekül, nur
    v.d.W.-Kräfte);
  \item \SI{20}{\celsius}: Ethanal (Carbonylgruppe ist polar,
    Dipol-Dipol-Wechselwirkungen);
  \item \SI{78}{\celsius}: Ethanol (Hydroxy-Gruppe kann \ch{H}-Brücken
    bilden);
  \item \SI{97}{\celsius}: Ethansäure (Carboxy-Gruppe ist polar, kann
    \ch{H}-Brücken bilden, Dimerisierung möglich);
  \item \SI{118}{\celsius}: Propansäure (Carboxy-Gruppe ist polar, kann
    \ch{H}-Brücken bilden, Dimerisierung möglich, Molekül größer als
    Ethansäure);
\end{itemize}
\end{solution}

\newpage
\addsec{Lösungen}
\printsolutions

\end{document}
