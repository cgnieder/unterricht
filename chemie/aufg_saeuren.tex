% arara: pdflatex
% arara: pdflatex
\documentclass{scrartcl}

\usepackage[T1]{fontenc}
\usepackage[utf8]{inputenc}
\usepackage[supstfm=libertinesups]{superiors}
\usepackage{libertine}
\usepackage{microtype}

\usepackage[greek,ngerman]{babel}
\usepackage{scrpage2}
\clearscrheadfoot
\pagestyle{scrheadings}
\chead{Seite \thepage}
\cfoot{\small C.\,Niederberger -- aktualisiert am \today}

\usepackage{upgreek}
\usepackage{chemmacros}
\chemsetup{
  option/language         = german,
%  chemformula/name-format = \centering ,
%  phases/pos              = sub
}

\usepackage{siunitx}
\sisetup{
  per-mode = symbol ,
  fixed-exponent = 0 ,
  exponent-product = \cdot ,
  list-final-separator = { und } ,
  list-pair-separator = { und } ,
  range-phrase = { bis }
}

\usepackage[load-headings]{exsheets}
\SetupExSheets{
  totoc ,
  headings = block-rev
}

\usepackage{fnpct}

\usepackage{booktabs}

\usepackage{mdframed}
\newmdenv[
  backgroundcolor = red!20,
  hidealllines    = true,
  leftmargin      = 2em ,
  rightmargin     = 2em ,
  skipabove       = \baselineskip ,
  skipbelow       = \baselineskip ,
  frametitle      = Definition ,
  frametitlefont  = \large\sffamily\scshape ,
  frametitlebackgroundcolor = red!40
]{definition}

\usepackage[colorlinks]{hyperref}

\begin{document}

\begin{center}
  \Huge\sffamily Säuren und \pH-Wert
\end{center}

\addsec{Ein bisschen Theorie}
Eine Säure ist ein Stoff, der bei der Reaktion mit Wasser ein Proton
(\Hpl)\footnote{Gibt das Wasserstoff-Atom, das aus einem Proton im Kern und
  einem Elektron in der Hülle besteht, sein Elektron ab, bleibt nur ein Proton
  übrig. Daher nennt man das Wasserstoff"=Ion \Hpl\ auch Proton.} abspalten
kann.  Das Proton verbindet sich mit Wasser (\ch{H2O}).
\begin{definition}
  Säurereaktion\slash Protolyse:
  \begin{reaction*}
    H-Rest + H2O -> Rest- + H3O+
  \end{reaction*}
\end{definition}
Bei dieser Reaktion entsteht immer ein negativ geladenes Säurerest-Ion.

Das verbindende aller Säuren ist das bei der Reaktion mit Wasser entstehende
\ch{H3O+}"=Teilchen, das \emph{Oxonium"=Ion}\footnote{früher: Hydronium"=Ion}.
Je mehr Oxonium"=Ionen in der Säurelösung schwimmen, desto saurer ist die
Lösung.

In neutralem, reinem Wasser sind ebenfalls Oxonium"=Ionen vorhanden, etwa $1$
Ion auf $10$ Millionen Wassermoleküle.  Das entspricht der Konzentration von
\begin{equation}
  c = \SI{0.0000001}{\mole\per\liter} = \SI{e-7}{\mole\per\liter} \,.
  \label{eq:aquadest}
\end{equation}

Der \emph{\pH-Wert} einer Lösung gibt die Konzentration der Oxonium"=Ionen in
\si{\mole\per\liter} an.  Da aber Zahlen wie \num{0.0000001} unhandliche
Zahlen sind, ist der \pH-Wert durch den \emph{negativen dekadischen
  Logarithmus} dieser Konzentration gegeben.  Bei reinem Wasser entspricht das
\begin{equation}
  \pH = -\log\num{0.0000001} = -\log 10^{-7} = -(-7) = 7 \,.
\end{equation}
Je \emph{höher} die Konzentration ist (je mehr Oxonium"=Ionen sich in der
Lösung befinden, je saurer die Lösung also ist), desto \emph{niedriger} wird
der \pH-Wert (siehe Tabelle \ref{tab:pH-Werte}).  Eine Konzentration von
$\num{1}:\num{1000} = \SI{0.0001}{\mole\per\liter}$ entspricht dann
\begin{equation}
  \pH = -\log\num{0.0001} = -\log 10^{-3} = 3 \,.
\end{equation}

\begin{table}
  \centering
  \begin{tabular}{l*{2}{>{$}l<{$}}S[retain-zero-exponent]}
    \toprule
      \textbf{\pH-Wert} & \text{\textbf{Konzentration}} &
      \text{\textbf{als Dezimale}} & {\textbf{als Zehnerpotenz}} \\
    \midrule
      0             & 1:1        & 1         & e0 \\
      1             & 1:10       & 0.1       & e-1 \\
      2             & 1:100      & 0.01      & e-2 \\
      3             & 1:1000     & 0.001     & e-3 \\
      4             & 1:10000    & 0.0001    & e-4 \\
      5             & 1:100000   & 0.00001   & e-5 \\
      6             & 1:1000000  & 0.000001  & e-6 \\
      7 (= neutral) & 1:10000000 & 0.0000001 & e-7 \\
    \bottomrule
  \end{tabular}
  \caption{Zusammenhang zwischen \pH-Wert und der Konzentration von
    Oxonium"=Ionen in Wasser in \si{\mole\per\liter}.}
  \label{tab:pH-Werte}
\end{table}

\newpage
\addsec{Aufgaben zum \pH-Wert}

\begin{question}
  In ein Planschbecken von \SI{1000}{\liter} wird \SI{1}{\liter} Essigsäure
  ($\pH=3$) gegossen.  Welchen \pH-Wert hat dann das Wasser?
\end{question}
\begin{solution}
  Die Säure hat den \pH-Wert $\pH = 3$, also eine Konzentration von $\num{e-3} =
  \SI{0.001}{\mole\per\liter}$.   Das entspricht $\num{1}:\num{1000}$.  Nun
  wird die Säure auf das tausendfache\footnote{Eigentlich auf das
    \num{1001}"=fache, das kann man aber vernachlässigen.} verdünnt und hat
  damit eine Konzentration von $1:\num{1000000} =\num{0.000001}$.  Das
  entspricht einem \pH-Wert von
\begin{equation*}
  \pH = -\log\num{0.000001} = -\log 10^{-6} = 6 \,.
\end{equation*}
Das Wasser hat also einen \pH-Wert von 6.
\end{solution}

\begin{question}[ID=1literhcl]
  Wieviel Wasser brauche ich, um \SI{1}{\liter} Salzsäure ($\pH = 1$) zu
  neutralisieren?
\end{question}
\begin{solution}
  Der \pH-Wert von 1 bedeutet, dass die Säure eine Konzentration (in
  \si{\mole\per\liter}) von $10^{-1} = 0.1 = 1:10$ hat.  Sie muss nun auf
  $\num{1}:\num{10000000}$ (siehe Gleichung~\eqref{eq:aquadest}) verdünnt
  werden, also um eine Million.  Das heißt, man benötigt \SI{1000000}{\liter}
  Wasser.
\end{solution}

\begin{question}
  Wieviel Wasser benötigt man, um \SI{0.3}{\liter} Salzsäure zu neutralisieren?
\end{question}
\begin{solution}
  Da man eine Million Liter Wasser für \SI{1}{\liter} Salzsäure benötigte
  (siehe Aufgabe~\QuestionNumber{1literhcl}), benötigen wir hier das
  \num{0.3}"=fache, also \SI{300000}{\liter} Wasser.
\end{solution}

\begin{question}
  Du möchtest aus Salpetersäure ($\pH = 1$) \SI{500}{\milli\liter} eine
  Salpetersäure mit $\pH = 5$ herstellen.  Wie sieht das Mischungsverhältnis
  aus?
\end{question}
\begin{solution}
  Um die Salpetersäure mit $\pH = 1$ (Konzentration
  $1:10\,\si{\mole\per\liter}$) auf $\pH = 5$ (Konzentration
  $\num{1}:\num{100000}\,\si{\mole\per\liter}$) zu verdünnen, müsste ich
  \SI{1}{\liter} auf \SI{10000}{\liter} verdünnen.  Da ich aber nur
  $\SI{500}{\milli\liter} = \SI{0.5}{\liter}$ und nicht \SI{10000}{\liter}
  herstellen möchte, benötige ich (Dreisatz):
  \begin{center}
    \begin{tabular}{ll}
        Säure mit $\pH = 1$  & Säure mit $\pH = 5$ \\
      \midrule
        \SI{1}{\liter}       & \SI{10000}{\liter} \\
        \SI{0.0001}{\liter}  & \SI{1}{\liter} \\
        \SI{0.00005}{\liter} & \SI{0.5}{\liter}
    \end{tabular}
  \end{center}
  \SI{0.00005}{\liter} sind \SI{0.05}{\milli\liter}.  Ich muss also
  \SI{0.05}{\milli\liter} der Säure mit $\pH = 1$ auf einen halben Liter
  verdünnen.
\end{solution}

\begin{question}
  Wieviel Wasser muss ich zu Essigsäure ($\pH = 3$) geben, um
  \SI{300}{\milli\liter} einer Essigsäure mit $\pH = 5$ zu erhalten?
\end{question}
\begin{solution}
  $\pH = 3$ entspricht einer Konzentration (in \si{\mole\per\liter}) von
  $\num{1}:\num{1000}$, $\pH = 5$ einer Konz. von $\num{1}:\num{100000}$.  Ich
  muss die Säure also um das \num{100}"=fache verdünnen. Das bedeutet also,
  ich muss \SI{3}{\milli\liter} der Säure mit $\pH = 1$ auf
  \SI{300}{\milli\liter} verdünnen.
\end{solution}

\begin{question}
  Ich verdünne \SI{30}{\milli\liter} Salzsäure-Lösung ($\pH = 2$) mit Wasser
  zu einem \pH{} von 4.  Welches Volumen hat die Lösung dann?
\end{question}
\begin{solution}
  $\pH = 2$ bedeutet eine Konzentration (in \si{\mole\per\liter}
  $\num{1}:\num{10}$, $\pH = 4$ bedeutet $\num{1}:\num{10000}$.  Ich muss um
  das \num{1000}"=fache verdünnen.  Also muss ich die \SI{30}{\milli\liter}
  auf $\SI{30000}{\milli\liter} = \SI{30}{\liter}$ verdünnen.
\end{solution}

\begin{question}
  \SI{1}{\milli\liter} Schwefelsäure ($\pH = 0.5$) wird auf $\pH = 6$
  verdünnt.  Wieviel Wasser brauche ich?
\end{question}
\begin{solution}
  Der $\pH = 0.5$ entspricht einer Konzentration (in \si{\mole\per\liter}) von
  $10^{-0.5} = \frac{1}{\sqrt{10}} \approx \frac{1}{3.162}$, also etwa
  $1:\num{3.162}$.  Um diese Lösung auf $\pH = 6$, also
  $\num{1}:\num{1000000}$ zu verdünnen, muss ich die Lösung um das
  $\frac{\num{1000000}}{\num{3.162}} = \num{316200}$"=fache verdünnen.  Ich
  muss die Lösung also auf $\SI{316200}{\milli\liter} = \SI{316.2}{\liter}$
  verdünnen.
\end{solution}

\newpage
\addsec{Lösungen}
\printsolutions

\end{document}
