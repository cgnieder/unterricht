\documentclass[DIV11]{scrartcl}

\usepackage[T1]{fontenc}
\usepackage[utf8]{inputenc}
\usepackage[supstfm=libertinesups]{superiors}
\usepackage{libertine}
\usepackage{microtype}

\usepackage[ngerman]{babel}
\usepackage{scrpage2}
\clearscrheadfoot
\pagestyle{scrheadings}
\chead{Seite \thepage}
\cfoot{\small C.\,Niederberger -- aktualisiert am \today}

\usepackage{upgreek}
\usepackage{chemmacros}
\chemsetup{
  option/language         = german,
  chemformula/name-format = \centering ,
%  phases/pos              = sub
}
\renewrobustcmd*\mch[1][]{\ch{^{#1}-}}
\renewrobustcmd*\pch[1][]{\ch{^{#1}+}}

\usepackage{siunitx}
\sisetup{
%  per-mode = fraction ,
  per-mode = symbol ,
  fixed-exponent = 0 ,
%  scientific-notation = true ,
  exponent-product = \cdot
}

\usepackage{chemfig}
\renewcommand*\printatom[1]{#1}
\setatomsep{1.8em}

\usepackage{pgfplots,tikz}
\pgfplotsset{
  compat = 1.7 ,
  y tick label style = {
    /pgf/number format/.cd,
    fixed,
    fixed zerofill,
    precision = 2,
    /tikz/.cd} ,
  scaled ticks = false
}

\usepackage[load-headings]{exsheets}
\SetupExSheets{
  totoc ,
  headings = block-rev
}
\usepackage{enumitem}

\usepackage{fnpct}
\usepackage{booktabs}

\usepackage{mdframed}
\newmdenv[linecolor=red!20, linewidth=1ex]{intermission}

\usepackage[colorlinks]{hyperref}

\begin{document}

\begin{center}
  \Huge\sffamily Gasgleichung
\end{center}

\addsec{Übungen}

\begin{question}[name=Ideales Gas]
\begin{tasks}
  \task Wie groß ist das molare Volumen $V_m$ von Stickstoff bei den heute
    üblichen Standardbedingungen (SATP) $T_0 = \SI{298.15}{\kelvin}$, $p_0 =
    \SI{1.000e5}{\pascal}$ ($\SI{1.000}{\bar}$) bzw.\ bei den in früherer Zeit
    üblichen Normalbedingungen (STP) $T = \SI{273.15}{\kelvin}$, $p =
    \SI{1.013e5}{\pascal}$ (\SI{1.013}{\bar}, \SI{1.000}{\atm})?
  \task Wieviele Moleküle Stickstoff befinden sich bei \SI{20}{\celsius} und
  \SI{25}{\bar} in einer \SI{10}{\liter}"=Stahlflasche?
\end{tasks}
\end{question}
\begin{solution}[name=Ideales Gas]
\begin{tasks}
  \task
  \begin{gather*}
    V_{m,\text{SATP}}
      = \frac{V}{n}
      = \frac{RT}{p}
      = \frac
          {
            \SI[sticky-per]{8.314}{\joule\per\kelvin\mole} \cdot
            \SI{298.15}{\kelvin}
          }
          {\SI{1.000e5}{\pascal}}
      = \SI{24.79}{\liter\per\mole} \\
   V_{m,\text{STP}}
      = \frac
          {
            \SI[sticky-per]{8.314}{\joule\per\kelvin\mole} \cdot
            \SI{273.15}{\kelvin}
          }
          {\SI{1.013e5}{\pascal}}
     = \SI{22.42}{\liter\per\mole}
  \end{gather*}
  \task
    \[
      N = \frac{pV}{kT}
        = \frac
            {\SI{25e5}{\pascal}\cdot\SI{0.010}{\cubic\metre}}
            {\SI{1.381e-23}{\joule\per\kelvin}\cdot\SI{293}{\kelvin}}
        = \num{6.18e24}
    \]
\end{tasks}
\end{solution}

\begin{question}[name=Virial-Ansatz]
Beim sogenannten Virial-Ansatz für reale Gase wird die Gasgleichung durch eine
Reihenentwicklung an ein reales Gas angepasst: \[\frac{pV}{n} = RT + Bp + Cp^2
+ Dp^3 + \cdots\]  Die Virial-Koeffizienten $B,C,D\ldots$ sind druckunabhängig
und müssen experimentell bestimmt werden. Bricht man die Entwicklung nach dem
zweiten Glied ab, erhält man \[\frac{pV}{n} = RT + Bp \,.\]  Zeigen Sie, dass
für $B=b-\frac{a}{RT}$ der Virialansatz in die van"=der"=Waalssche Gleichung
übergeht.

Tipp: stellen Sie beide Gleichungen auf den Kompressibilitätsfaktor $Z$ um und
vergleichen Sie dann die Gleichungen.
\end{question}
\begin{solution}[name=Virial-Ansatz]
  Wir gehen die Lösung von zwei Seiten an. Zunächst stellen wir die verkürzte
  Virialgleichung in etwas anderer Form dar:
  \begin{equation}
    pV_m = RT + Bp
    \quad\Leftrightarrow\quad
    \frac{pV_m}{RT} = 1 + \frac{Bp}{RT}
    \qquad\Leftrightarrow\quad
    \frac{pV_m}{RT} = 1 + \frac{B}{V_m} \label{eq:virial}
  \end{equation}
  Nun versuchen wir, die van"=der"=Waals"=Gleichung in eine vergleichbare Form
  zu bringen:
  \begin{equation}
    p = \frac{RT}{V_m-b} - \frac{1}{V_m^2}
    \quad\Leftrightarrow\quad
    \frac{pV_m}{RT} = \frac{V_m}{V_m-b} - \frac{a}{RTV_m} \label{eq:vdW}
  \end{equation}
  Der erste Bruch auf der rechten Seite in Gleichung~\eqref{eq:vdW} lässt sich
  durch Polynomdivision als unendliche Summe darstellen:
  \begin{equation}
    \frac{V_m}{V_m-b}
      = 1 + \frac{b}{V_m} + \frac{b^2}{V_m^2} + \frac{b^3}{V_m^3} + \cdots
      \label{eq:inv-sum}
  \end{equation}
  Diese Summe brechen wir nun nach dem zweiten Glied ab und fügen Sie wieder
  in Gleichung~\eqref{eq:vdW} ein.
  \begin{equation}
    \frac{pV_m}{RT}
      = 1 + \frac{b}{V_m} - \frac{a}{RTV_m}
      = 1 + \biggl(b-\frac{a}{RT}\biggr)\cdot\frac{1}{V_m} \label{eq:vdW2}
  \end{equation}
  Ein Vergleich von Gleichung~\eqref{eq:vdW2} mit Gleichung~\eqref{eq:virial}
  zeigt nun, dass die Gleichungen übereinstimmen, wenn $B = b - a/(RT)$ ist.
\end{solution}

\begin{question}[name=Druck als Zustandsfunktion]
Zeigen sie mit Hilfe des sogenannten Schwarzschen Satzes, dass der Druck eine
Zustandsfunktion ist, und zwar im Falle eines idealen Gases als auch eines
Gases, welches der van"=der"=Waalsschen Zustandsgleichung gehorcht.
\end{question}
\begin{solution}[name=Druck als Zustandsfunktion]
Der Satz von Schwarz besagt, dass für eine Zustandsfunktion die Reihenfolge
des Ableitens keine Rolle spielt. Mit anderen Worten: der Zustand ist
unabhängig davon, wie er erreicht wird:
\[
  \frac{\partial p^2}{\partial T \partial V}
  = \frac{\partial p^2}{\partial V \partial T}
\]
Für ein ideales Gas ergibt sich also:
\begin{align*}
 p &= \frac{nRT}{V}\\
 \frac{\partial p}{\partial T} &= \frac{nR}{V} & \frac{\partial p}{\partial V}
   &= -\frac{nRT}{V^2}\\
 \frac{\partial p^2}{\partial V\partial T}
   &= -\frac{nR}{V^2} & \frac{\partial p^2}{\partial T\partial V}
   &= -\frac{nR}{V^2}
\end{align*}
Für ein van"=der"=Waals-Gas erhalten wir:
\begin{align*}
 p &= \frac{RT}{V_m-b} - \frac{a}{V_m^2}\\
 \frac{\partial p}{\partial T}
   &= \frac{R}{V_m-b} & \frac{\partial p}{\partial V_m}
   &= -\frac{RT}{(V_m-b)^2} + \frac{2a}{V_m^3}\\
 \frac{\partial p^2}{\partial V_m\partial T}
   &= -\frac{R}{(V_m-b)^2} & \frac{\partial p^2}{\partial T\partial V_m}
   &= -\frac{R}{(V_m-b)^2}
\end{align*}
\end{solution}

\begin{question}[name=Totales Differential]
Überprüfen Sie, ob der folgende Ausdruck ein totales Differential ist:
\[ dQ = \frac{RT}{p}dp - RdT \]
\end{question}
\begin{solution}[name=Totales Differential]
Wenn $dQ$ ein totales Differential ist, dann muss der Schwarzsche Satz erfüllt
sein, es muss also gelten:
\[
  \frac{\partial Q^2}{\partial T\partial p} = \frac{\partial Q^2}{\partial p\partial T}
\]
Ein Vergleich der beiden Gleichungen
\[
  dQ = \frac{RT}{p}dp - RdT
  \quad \text{und} \quad
  dQ = \frac{\partial Q}{\partial p}dp + \frac{\partial Q}{\partial T}dT
\]
liefert uns
\[
  \frac{\partial Q}{\partial p} = \frac{RT}{p}
  \quad \Rightarrow \qquad
  \frac{\partial Q^2}{\partial T\partial p} = \frac{R}{p}
\]
und
\[
  \frac{\partial Q}{\partial T} = -R
  \quad \Rightarrow \qquad
  \frac{\partial Q^2}{\partial p\partial T} = 0 \,.
\]
Also ist $dQ$ kein totales Differential.
\end{solution}

\begin{question}[name=Van-der-Waalssche Zustandsgleichung I,ID=vdW:1]
Der kritische Druck und die kritische Temperatur von Stickstoff betragen
\[p_k = \SI{3.3944e6}{\pascal} \text{ und } T_k = \SI{126.15}{\kelvin}.\]
Wie lautet die van"=der"=Waalssche Zustandsgleichung für Stickstoff?  Wie groß
ist das kritische Molvolumen von \ch{N2}, $V_{m,k}$?  Geben Sie die
van"=der"=Waals-Konstante $a$ sowohl in
\si{\pascal(\metre\cubed\per\mole)\squared} als auch in
\si{\atm(\liter\per\mole)\squared} an!
\end{question}
\begin{solution}[name=Van-der-Waalssche Zustandsgleichung I]
  Es gilt $p_k = a/(27b^2)$ und $T_k = 8a/(27Rb)$ (siehe
  Übung~\QuestionNumber{vdW:3}). Durch Teilen der kritischen Temperatur durch
  den kritischen Druck kann man also $b$ bestimmen:
  \[
    \frac{T_k}{p_k} = \frac{8b}{R}
    \quad\Leftrightarrow\quad
    b = \frac{RT_k}{8p_k}
      = \SI{3.8625e-5}{\metre\cubed\per\mole}
  \]
  Mit jeder der beiden kritischen Größen lässt sich dann $a$ berechnen:
  \[
    a = p_k \cdot 27b^2
      = \SI{0.13673}{\pascal\metre\tothe{6}\per\mole\squared}
      = \SI{1.3494}{\atm\square\liter\per\mole\squared}
  \]
\end{solution}

\begin{question}[name=Van-der-Waalssche Zustandsgleichung II,ID=vdW:2]
Ein Gas gehorcht näherungsweise der van"=der"=Waals"=Gleichung ($a =
\SI{0.50}{\metre\tothe{6}\pascal\per\mol\squared}$).  Bei \SI{273}{\kelvin} und
\SI{3}{\mega\pascal} beträgt sein Volumen $\SI{5e-4}{\metre\cubed\per\mole}$.
\begin{tasks}
  \task Berechnen Sie die van"=der"=Waals"=Konstante $b$.
  \task Welche Bedeutung besitzen die van"=der"=Waals"=Konstanten $a$ und $b$?
  \task Wie hängt das Kovolumen mit dem Radius $r$ eines "`kugelförmigen"'
    Moleküls (z.\,B. Methan) zusammen?
\end{tasks}
\end{question}
\begin{solution}[name=Van-der-Waalssche Zustandsgleichung II]
  \begin{figure}
    \centering
    \begin{tikzpicture}
      \draw (0,0) circle (1cm) ;
      \draw (2,0) circle (1cm) ;
      \draw[dotted] (1,0) circle (2cm) ;
    \end{tikzpicture}
    \caption{Schematische Darstellung des Kovolumens zweier kugelförmiger
      Atome}\label{fig:kovolumen}
  \end{figure}
  \begin{tasks}
    \task Umgestellt auf $b$ lässt sich $b$ durch einfaches Einsetzen
      berechnen:
      \[
        b = V_m - \frac{RT}{p-a/V_m^2}
          = \SI{-1.77e-3}{\metre\cubed\per\mole}
      \]
    \task Die van"=der"=Waals"=Konstante $b$ wird oft als das \emph{Kovolumen}
      oder \emph{Ausschließungsvolumen} betrachtet, also als das Volumen, dass
      sich die Teilchen gegenseitig wegnehmen, was etwas größer als das
      Eigenvolumen der Teilchen ist.

      Den Quotienten $a/V_m^2$ nennt man in der Regel den \emph{Binnendruck}
      des Gases.  Durch die gegenseitigen anziehenden/abstoßenden
      Wechselwirkung ist der Druck eines realen Gases in der Regel geringer
      als er für ein ideales Gases wäre, was durch den Binnendruck in der
      van"=der"=Waals"=Gleichung berücksichtigt wird.
    \task Das Ko-Volumen kann man sich durch folgende Überlegung klarmachen:
      das Volumen, das zwei Kugeln gegenseitig nicht zur Verfügung haben
      entspricht einer Kugel mit doppeltem Radius (siehe
      Abbildung~\ref{fig:kovolumen}) also dem achtfachen des Volumens eines
      Atoms.  Bezogen auf eines der Atome entspricht  das dem vierfachen des
      Eigenvolumens.  Rechnet man das auf ein Mol hoch, ergibt sich  $b =
      4\cdot V_{m,\text{Eigen}}$.
  \end{tasks}
\end{solution}

\begin{question}[name=Van-der-Waalssche Zustandsgleichung III,ID=vdW:3]
\SI{10}{\mole} Methan nehmen bei \SI{0}{\celsius} ein Volumen von
\SI{1.756}{\liter} ein.
\begin{tasks}
  \task Berechnen Sie den Druck von Methan mit Hilfe der idealen Gasgleichung.
  \task Berechnen Sie den Druck mit Hilfe der Van"=der"=Waalsschen
    Zu\-stands\-glei\-chung.  Die kritische Temperatur und der kritische Druck
    von Methan betragen $T_k = \SI{190.7}{\kelvin}$ bzw. $p_k =
    \SI{4.63}{\mega\pascal}$.
  \task Ein Manometer zeigt Ihnen einen Wert von \SI{10.13}{\mega\pascal} an.
    Erläutern Sie in diesem Zusammenhang den Unterschied zwischen den von
    Ihnen vorhergesagten Werten.
  \task Schätzen Sie aus dem Kovolumen $b$ den Radius des Methanmoleküls ab.
\end{tasks}
Tipp: Der Zusammenhang zwischen kritischen Größen und den
Van"=der"=Waals"=Konstanten $a$ und $b$ ergibt sich aus der Tatsache, dass am
kritischen Punkt die Isotherme im $p$-$V$-Diagramm einen Sattelpunkt besitzt,
für den gilt: $\frac{\partial p}{\partial V_m} = 0$; $\frac{\partial
  ^2p}{\partial V_m^2} = 0$.  Nach Lösen der Gleichungen erhält man $p_k =
\frac{a}{27b^2}$ und $T_k = \frac{8a}{27Rb}$.
\end{question}
\begin{solution}[name=Van-der-Waalssche Zustandsgleichung III]
\begin{tasks}
  \task
    \[
      p = \frac{nRT}{V}
        =
          \frac
            {
              \SI{10}{\mole} \cdot \SI{8.3145}{\joule\per\kelvin\per\mole}
              \cdot \SI{273}{\kelvin}
            }
            {\SI{1.756e-3}{\metre^3}}
        = \SI{12.9}{\mega\pascal}
    \]
  \task kritische Größen liefern
    \[
      b = \frac{RT_k}{8 p_k}
        = \frac
            {\SI{8.3145}{\joule\per\kelvin\per\mole}\cdot\SI{190.7}{\kelvin}}
            {8 \cdot\SI{4.63e6}{\pascal}}
        = \SI{42.8e-6}{\metre^3\per\mole}
    \]
    \[
      a = \frac{27 R^2 T_k^2}{64 p_k}
        = \frac
            {
              27 \cdot
              \left(
                \SI{8.3145}{\joule\per\kelvin\per\mole}
              \right)^2
              \cdot \left( \SI{190.7}{\kelvin} \right)^2
            }
             {64 \cdot \SI{4.63e6}{\pascal}}
        = \SI{0.229}{\pascal\metre^6\per\mole^2}
    \]
    \begin{align*}
     p &= \frac{nRT}{V - nb} - \frac{an^2}{V^2} \\
       &= \frac
            {
              \SI{10}{\mole} \cdot \SI{8.3145}{\joule\per\kelvin\per\mole} \cdot
              \SI{273}{\kelvin}
            }
            {\SI{1.756e-3}{\metre^3} - \SI{10}{\mole} \cdot
          \SI{42.8e-6}{\metre^3\per\mole}}
         -
          \frac
            {
              \SI{0.229}{\pascal\metre^6\per\mole^2} \cdot
              \left( \SI{10}{\mole} \right)^2
            }
            {\left( \SI{1.756e-3}{\metre^3} \right)^2} \\
       &= \SI{9.67}{\mega\pascal}
    \end{align*}
  \task Die ideale Gasgleichung, die Eigenvolumen und Wechselwirkungen zwischen
    den Gasteilchen vernachlässigt, kann nur einen Näherungswert liefern.  Die
    van"=der"=Waalssche Gleichung nähert zwar mithilfe empirischer Konstanten
    etwas besser, ist aber immer noch eine Näherungsgleichung.
  \item
    \begin{gather*}
      b = 4 \cdot V_{m,\text{eigen}}
        = 4 \cdot N_A \cdot \frac{4}{3} \pi r^3 \\
      r = \sqrt[3]{\frac{3 b}{16 N_A \pi}}
        = \sqrt[3]
            {
              \frac{3 \cdot \SI{42.8e-6}{\metre^3\per\mole}}{16 \cdot
              \SI{6.0221e23}{\per\mole} \cdot \pi}
            }
        = \SI{1.62e-10}{\metre} = \SI{1.62}{\angstrom}
    \end{gather*}
\end{tasks}
\end{solution}

\begin{question}[name=Ideales vs. reales Gas]
\SI{10}{\mole} Ethan nehmen bei \SI{27}{\celsius} ein Volumen von
\SI{4.86}{\liter} ein.
\begin{tasks}
  \task Berechnen Sie den Druck des Gases nach dem idealen Gasgesetz und
  \task nach der van"=der"=Waals-Gleichung ($a =
    \SI{5.507}{\liter\squared\bar\per\mole\squared}$, $b =
    \SI{0.0651}{\liter\per\mole}$).
  \task Berechnen Sie die prozentuale Abweichung und beurteilen Sie, ob
    anziehende oder abstoßende Kräfte vorherrschen.
\end{tasks}
\end{question}
\begin{solution}[name=Ideales vs. reales Gas]
\begin{tasks}
  \task
    \[
      p = \frac{nRT}{V}
        = \frac
            {
              \SI{10}{\mole} \cdot \SI{8.3145}{\joule\per\kelvin\per\mole}
              \cdot \SI{300}{\kelvin}
            }
            {\SI{4.86e-3}{\metre\cubed}}
        = \SI{5.13}{\mega\pascal}
    \]
  \task
    \begin{align*}
      p &= \frac{nRT}{V-nb}-\frac{an^2}{V^2} \\
        &= \frac
             {
               \SI{10}{\mole} \cdot \SI{8.3145}{\joule\per\kelvin\per\mole}
               \cdot \SI{300}{\kelvin}
             }
             {\SI{4.86e-3}{\metre\cubed} -
           \SI{10}{\mole}\cdot\SI{0.0651e-3}{\metre\cubed\per\mole} }
         -
           \frac
             {
               \SI{0.5507}{\metre^6\pascal\per\mole^2} \cdot
               (\SI{10}{\mole})^2
             }
             { (\SI{4.86e-3}{\metre\cubed})^2 } \\
        &= \SI{3.59}{\mega\pascal}
    \end{align*}
  \task Es herrschen anziehende  Kräfte vor, da $p_{\text{vdW}} <
    p_{\text{id}}$: $ \Delta p/p_\text{id} = (5.13 - 3.59)/5.13 =
    \SI{30}{\percent}$.
\end{tasks}
\end{solution}

\begin{question}[name=Differentialquotient]
Es wird angenommen, dass für ein bestimmtes Gas zwar $(\partial U/\partial
V)_T = 0 $ ansonsten aber die Zustandsgleichung $ p(V - b) = RT $ gilt, wobei
$b$ die van"=der"=Waals"=Konstante ist.  Berechnen Sie den partiellen
Differentialquotienten $(\partial H/\partial V)_T$.
\end{question}
\begin{solution}[name=Differentialquotient]
  Für ein ideales Gas wäre $(\partial H/\partial V)_T = 0$.  Mit der
  vereinfachten Zustandsgleichung für reale Gase, bei der das Kovolumen
  berücksichtigt wird, erwarten wir einen Wert ungleich Null.

  Zunächst setzen wir $H=U+pV$ in den gesuchten Differentialquotienten ein:
  \begin{equation}
    \biggl(\frac{\partial H}{\partial V}\biggr)_T
      = \biggl( \frac{\partial(U + pV)}{\partial V} \biggr)_T
      = \biggl(\frac{\partial U}{\partial V}\biggr)_T
        + \biggl(\frac{\partial(pV)}{\partial V}\biggr)_T
      = \biggl(\frac{V\partial p}{\partial V}\biggr)_T
        + \biggl(\frac{p\partial V}{\partial V}\biggr)_T \label{eq:dHdV}
  \end{equation}
  In Gleichung~\eqref{eq:dHdV} setzt man nun die Zustandsgleichung ein:
  \begin{multline}
    \biggl(\frac{V\partial p}{\partial V}\biggr)_T
      + \biggl(\frac{p\partial V}{\partial V}\biggr)_T
      = \frac{\partial}{\partial V} \biggl(\frac{RT}{V-b}\biggr) \cdot V  + p \\
      = -\frac{RT}{(V-b)^2} \cdot V + p
      = -p \cdot \frac{V}{V-b} + p
      = p \cdot \biggl(1-\frac{V}{V-b}\biggr)
  \end{multline}
  Wenn $b=0$ ist gibt das Ergebnis ganz korrekt $0$ für ein ideales Gas.  Für
  ein reales Gas, das der angegebenen Zustandsfunktion gehorcht, ergibt sich
  für $(\partial H/\partial V)_T$ wie erwartet ein positives Ergebnis.
\end{solution}

\begin{question}[name=Chemiebetrieb]
In einem Chemiebetrieb wird bei \SI{60}{\celsius} und einem Behälterdruck von
\SI{5}{\mega\pascal} durch falsches Öffnen der Ventile ein
\SI{200}{\liter}"=Behälter mit Siliziumtetrachlorid (\ch{SiCl4}) mit einem
\SI{1300}{\liter}"=Behälter Ammoniak (\ch{NH3}) verbunden.  Die spontan
ablaufende Reaktion führt zur stöchiometrisch vollständigen Umsetzung in
Siliziumnitrid \ch{Si3N4\sld} und Ammoniumchlorid \ch{NH4Cl\sld}.

Welcher Druck stellt sich ein, wenn das Volumen der Feststoffe berücksichtigt
und Ammoniak nach der Reaktion als reales Gas im niedrigen Druckbereich
betrachtet wird?

Verwenden Sie folgende Daten:
\begin{itemize}
  \item \ch{NH3}:
    $a = \SI{0.442}{\newton\metre\tothe4\per\mole\squared}$,
    $b = \SI{37.3e-6}{\metre\cubed\per\mole}$;
  \item \ch{Si3N4}:
    $\varrho = \SI{3.2}{\gram\per\centi\metre\cubed}$;
  \item \ch{NH4Cl}:
    $\varrho = \SI{1.54}{\gram\per\centi\metre\cubed}$
\end{itemize}
\end{question}
\begin{solution}[name=Chemiebetrieb]
  Die Reaktion findet gemaß der Reaktionsgleichung
  \begin{reaction*}
    3 SiCl4 + 16 NH3 -> Si3N4 + 12 NH4Cl
  \end{reaction*}
  statt.  Zunächst müssen die vorhandenen Stoffmengen der Edukte und die
  entstandenen Mengen der Produkte bestimmt werden.
  \[
    n_{\ch{SiCl4}} = \frac{pV}{RT} = \SI{361.2}{\mole}
    \qquad
    n_{\ch{NH3}}   = \frac{pV}{RT} = \SI{2347.6}{\mole}
  \]
  Da die Umsetzung im Verhältnis $3:16$ stattfindet, werden
  $\frac{16}{3}\cdot\SI{361.2}{\mole} = \SI{1926.4}{\mole}$ Ammoniak
  verbraucht, \SI{421.2}{\mole} verbleiben.  Es entstehen
  $\frac{1}{3}\cdot\SI{361.2}{\mole} = \SI{120.4}{\mole}$ \ch{Si3N4} und
  $4\cdot\SI{361.2}{\mole} = \SI{1444.8}{\mole}$ Ammoniumchlorid.

  Nun berechnen wir die Volumina der Feststoffe:
  \[
    V = \varrho \cdot m = \varrho\cdot M\cdot n
    \qquad
    V_{\ch{S13N4}} = \SI{54.1}{\liter}
    \quad
    V_{\ch{NH4Cl}} = \SI{119}{\liter}
  \]
  Damit verbleibt als Gesamtvolumen \SI{1.3269}{\metre\cubed}.  Über die
  van"=der"=Waalssche Gleichung lässt sich nun der Druck, den das verbleibende
  Ammoniak verursacht, berechnen.
  \[
    p = \frac{nRT}{V-nb} + \frac{an^2}{V^2}
      = \SI{9.34e5}{\pascal}
      = \SI{9.34}{\bar}
  \]
\end{solution}

\newpage
\addsec{Lösungen}
\printsolutions

\end{document}
