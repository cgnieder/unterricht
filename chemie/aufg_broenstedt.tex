% arara: pdflatex
% arara: pdflatex
\documentclass{scrartcl}

\usepackage[T1]{fontenc}
\usepackage[utf8]{inputenc}
\usepackage[supstfm=libertinesups]{superiors}
\usepackage[utopia]{mathdesign}
\usepackage[oldstyle]{libertine}
\usepackage{microtype}

\usepackage[greek,ngerman]{babel}
\usepackage{scrpage2}
\clearscrheadfoot
\pagestyle{scrheadings}
\chead{Seite \thepage}
\cfoot{\small C.\,Niederberger -- aktualisiert am \today}

\usepackage{upgreek}
\usepackage{chemmacros}
\chemsetup{
  option/language    = german,
  chemformula/format = \libertineLF
}
\renewrobustcmd*\mch[1][]{\ch{^{#1}-}}
\renewrobustcmd*\pch[1][]{\ch{^{#1}+}}

\DeclareChemReaction[star]{gthreactions}{gather}
\newcommand*\rctref[1]{\{\ref{#1}\}}

\usepackage{siunitx}
\sisetup{
  detect-all ,
%  per-mode = fraction ,
  per-mode = symbol ,
  fixed-exponent = 0 ,
%  scientific-notation = true ,
  exponent-product = \cdot ,
  list-final-separator = { und } ,
  list-pair-separator = { und } ,
  range-phrase = { bis }
}

\usepackage{chemfig}
\renewcommand*\printatom[1]{#1}
\setatomsep{1.8em}

\usepackage{pgfplots,tikz}
\pgfplotsset{
  compat = 1.7 ,
  scaled ticks = false
}

\usepackage[load-headings]{exsheets}
\SetupExSheets{
  totoc ,
  headings = block-rev
}

\usepackage{fnpct}

\usepackage{booktabs}
\usepackage{collcell}
\newcolumntype{C}{>{\collectcell\ch}l<{\endcollectcell}}

\usepackage{mdframed}
\newmdenv[
  backgroundcolor = red!20,
  hidealllines    = true,
  leftmargin      = 2em ,
  rightmargin     = 2em ,
  skipabove       = \baselineskip ,
  skipbelow       = \baselineskip ,
  splittopskip    = .5\baselineskip ,
  splitbottomskip = .5\baselineskip ,
  frametitle      = Definition ,
  frametitlefont  = \large\sffamily\scshape ,
  frametitlebackgroundcolor = red!40
]{definition}

\newmdenv[
  backgroundcolor = black!5,
  hidealllines    = true,
  leftmargin      = 2em ,
  rightmargin     = 2em ,
  skipabove       = \baselineskip ,
  skipbelow       = \baselineskip ,
  splittopskip    = .5\baselineskip ,
  splitbottomskip = .5\baselineskip ,
  frametitle      = Beispiel ,
  frametitlefont  = \large\sffamily\scshape ,
  frametitlebackgroundcolor = black!10
]{beispiel}


\usepackage{multicol}
\usepackage[colorlinks]{hyperref}

\begin{document}

\begin{center}
  \Huge\sffamily Säuren und Basen nach Br\o nstedt
\end{center}
\begin{multicols}{2}
  \tableofcontents
\end{multicols}

\section{Die Basis}
\subsection{Worum es geht}
\subsubsection{Säuren}
Die Basis für die Säure-Theorie nach Br\o nstedt ist die sogenannte
\emph{Protolyse}\footnote{Proton: \Hpl, griech. \textgreek{l'usic} (l\'ysis):
  Auflösung}"=Gleichung~\rctref{rct:protolyse}.  Hierbei wird ein Proton
(\Hpl) von einem Stoff, der \emph{Säure}, auf einen anderen Stoff,
\emph{Wasser}, übertragen.
\begin{reaction}
  HA + H2O <=> A- + H3O+ \label{rct:protolyse}
\end{reaction}
Dabei ist \ch{A} ein (beinahe) beliebiger Rest.  Jeder Stoff, der mit Wasser
diese Gleichgewichtsreaktion\footnote{Siehe dazu Thema \emph{chemisches
  Gleichgewicht}} eingehen kann, ist eine Säure.  Genauer wäre: \emph{eine
wässrige Lösung eines solchen Stoffes ist eine Säure}.  Einige Möglichkeiten
für solche Stoffe sind in Tabelle~\ref{tab:saeuren} aufgelistet.

\begin{table}
  \centering
  \caption{wichtige Säuren}\label{tab:saeuren}
  \begin{tabular}{lCCl}
    \toprule
      \bfseries Säure          &         & \textbf{Säurerest} \\
      Name                     & Formel  & Formel  & Name \\
    \midrule
      Salzsäure                & HCl     & Cl-     & Chlorid \\
      Bromwasserstoffsäure     & HBr     & Br-     & Bromid \\
      Schwefelwasserstoffsäure & H2S     & HS-     & Hyrdogensulfid \\
                               & HS-     & S^2-    & Sulfid \\
      Schwefelsäure            & H2SO4   & HSO4-   & Hydrogensufat \\
                               & HSO4-   & SO4^2-  & Sulfat \\
      schweflige Säure         & H2SO3   & HSO3-   & Hydrogensulfit \\
                               & HSO3-   & SO3^2-  & Sulfit \\
      Salpetersäure            & HNO3    & NO3-    & Nitrat \\
      salpetrige Säure         & HNO2    & NO2-    & Nitrit \\
      Kohlensäure              & H2CO3   & HCO3-   & Hydrogencarbonat \\
                               & HCO3-   & CO3^2-  & Carbonat \\
      Phosphorsäure            & H3PO4   & H2PO4-  & Dihydrogenphosphat \\
                               & H2PO4-  & HPO4^2- & Hydrogenphosphat \\
                               & HPO4^2- & PO4^3-  & Phosphat \\
      Blausäure                & HCN     & CN-     & Cyanid \\
    \bottomrule
  \end{tabular}
\end{table}

\begin{definition}
  Einen Stoff, der Protonen (\Hpl) abgeben kann, einen sogenannten
  \emph{Protonendonatoren}, nennt man Säure.
\end{definition}

Das, was die saure Wirkung einer Säure ausmacht, kann offensichtlich nicht das
Säuremolekül selbst sein: es reagiert schließlich (je nach Gleichgewichtslage
kaum bis nahezu vollständig).  Auch der Säurerest kann nicht verantwortlich
sein: er ist bei jeder Säure anders.  Übrig bleibt die Gemeinsamkeit aller
Protolyse-Reaktionen: das \emph{Oxonium}- oder
\emph{Hydronium}"=Ion \ch{H3O+}.

\subsubsection{Basen}
Auch für Basen gibt eine Reaktionsgleichung, die ihre Rolle festlegt:
\begin{reaction}
  B + H2O <=> HB+ + OH- \label{rct:basen_protolyse}
\end{reaction}
Hier gilt also das umgekehrte: eine Base gibt kein Proton ab, sondern nimmt
eines auf. Ein paar wichtige Basen sind in Tabelle~\ref{tab:basen} aufgelistet.
\begin{definition}
  Einen Stoff, der Protonen (\Hpl) aufnehmen kann, einen sogenannten
  \emph{Protonenakzeptoren}, nennt man Base.
\end{definition}

\subsection{Die Theorie}
Überträgt man obige Definitionen nochmals auf die
Reaktionen~\rctref{rct:protolyse} und~\rctref{rct:basen_protolyse},
\begin{reactions*}
  HA + H2O &<=> A- + H3O+ \\
  B  + H2O &<=> HB+ + OH-
\end{reactions*}
dann stellen wir folgendes fest: in Reaktion~\rctref{rct:protolyse} sind
\ch{HA} und \ch{H3O+} Säuren, denn sie können ein \ch{H+} abgeben.  \ch{H2O}
und \ch{A-} sind Basen, denn sie können ein \ch{H+} aufnehmen.  In
Reaktion~\rctref{rct:basen_protolyse} sind \ch{H2O} und \ch{HB+} Säuren und
\ch{B} und \ch{OH-} Basen.

Wir entdecken ein Prinzip:
\begin{reactions*}
       !(\color{red}Säure~A)( HA )  + !(\color{blue}Base~B)( H2O )
  &<=> !(\color{red}Base~A)( A- )   + !(\color{blue}Säure~B)( H3O+ ) \\
       !(\color{red}Base~A)( B )    + !(\color{blue}Säure~B)( H2O )
  &<=> !(\color{red}Säure~A)( HB- ) + !(\color{blue}Base~B)( OH- )
\end{reactions*}
Offenbar wird aus einer Säure eine Base und umgekehrt.  Man spricht daher von
\emph{korrespondierenden} oder auch \emph{konjugierten Säure/Base-Paaren}.
Und offenbar ist Wasser selbst auch sowohl eine Säure als auch eine Base, je
nachdem womit es reagiert.

\begin{definition}
  Ein Stoff, der sowohl Säure als auch Base ist, nennt man
  \emph{Ampholyt}. Ein solcher Stoff ist \emph{amphoter}.
\end{definition}

\begin{table}
  \centering
  \caption{wichtige Basen}\label{tab:basen}
  \begin{tabular}{lCCl}
    \toprule
      \bfseries Base &         & \textbf{Basenrest} & \\
      Name           & Formel  & Formel             & Name \\
    \midrule
      Natronlauge    & NaOH \\
      Kalilauge      & KOH  \\
      Ammoniak       & NH3     & NH4+               & Ammonium \\
      Hydrazin       & N2H4    & N2H5+              & Hydrazinium \\
      Methylamin     & CH3NH2  & CH3NH3+            & Methylammonium \\
      Anilin         & C6H5NH2 & C6H5NH3+           & Anilinium \\
      Pyridin        & C5H5N   & C5H5NH+            & Pyridinium \\
    \bottomrule
  \end{tabular}
\end{table}

Wenn nun Wasser ein Ampholyt ist, also Säure und Base, bedeutet das, es kann
mit sich selbst eine Protolyse-Reaktion machen:
\begin{reaction}
  H2O + H2O <<=> H3O+ + OH- \label{rct:autoprotolyse}
\end{reaction}
Diese Gleichung, eine sogenannte
\emph{Autoprotolyse}\footnote{griech. \textgreek{a>ut'o}: selbst}, wird bei
der Definition des \pH-Werts noch eine Rolle spielen.

\section{Quantitative Betrachtung}
\subsection{Starke und schwache Säuren}
Wie \emph{sauer} eine Lösung nun ist, hängt von der
Stoffmengenkonzentration\footnote{Stoffmengenkonzentration: $c=\frac{n}{V}$,
  $[c]=\si{\mole\per\liter}$.} des Oxonium"=Ions ab.  Diese wiederum hängt
über Reaktion~\rctref{rct:protolyse} von der eingesetzten Konzentration der
Säure sowie von der Lage des Gleichgewichts ab.  Zur Beschreibung der
Säurestärke benötigen wir also das \emph{Massenwirkungsgesetz} für
Reaktion~\rctref{rct:protolyse}.  Zur Minimierung der Schreibarbeit werden wir
ab hier $[\ch{A}]$ schreiben, wenn wir $c(\ch{A})$, also die
Stoffmengenkonzentration des Stoffes \ch{A}, meinen.
\begin{gather}
  K =
    \frac
      { [\ch{A-}] \cdot [\ch{H3O+}] }
      { [\ch{HA}] \cdot [\ch{H2O}] } \label{eq:protolyse_mwg} \\
  [\ch{H3O+}] =
    K \cdot \frac{ [\ch{HA}] \cdot [\ch{H2O}] }{ [\ch{A-}] }
\end{gather}
Nennen wir nun das Produkt aus $K \cdot [\ch{H2O}] = \Ka$, erhalten wir eine
relativ einfache Gleichung zur Berechnung der \ch{H3O+}-Konzentration,
vorausgesetzt wir kennen \Ka und die im Gleichgewicht vorliegenden
Konzentrationen:
\begin{equation}
  [\ch{H3O+}] = \Ka \cdot \frac{ [\ch{HA}] }{ [\ch{A-}] } \label{eq:cHtO_aus_Ks}
\end{equation}

Aus mehreren Gründen verwendet man diese Gleichung jedoch nicht.
\begin{itemize}
  \item zum einen ist die Konzentration im Gleichgewicht unbekannt und müsste
    vorher berechnet werden.
  \item zum zweiten können die Werte von $[\ch{H3O+}]$ zwischen
    ca. $\SI{0.00000000000001}{\mole\per\liter} = \SI{1e-14}{\mole\per\liter}$ und
    \SI{1}{\mole\per\liter} schwanken.
  \item zum dritten benötigt man den Wert von \Ka, der noch mehr schwanken
    kann.
\end{itemize}
Gleichung~\eqref{eq:cHtO_aus_Ks} ist -- nach \Ka umgestellt -- aber die
Definition der \emph{Säurekonstanten}:
\begin{equation}
  \Ka = \frac{ [\ch{H3O+}] \cdot [\ch{A-}] }{ [\ch{HA}] } \label{eq:Ks_definition}
\end{equation}
Der Wert von \Ka gibt auf eine Weise an, wieviel einer Säure
dissoziert\footnote{zerfallen} ist, wie weit also das Gleichgewicht auf Seite
der Produkte liegt.  Man legt fest:
\begin{align*}
  \Ka &>1 \quad\text{starke Säure} \\
  \Ka &<1 \quad\text{schwache Säure}
\end{align*}
Da dieser Wert \emph{sehr} groß oder \emph{sehr} klein sein kann, verwendet
man lieber den sogenannten \emph{\pKa-Wert} (mit $\log$ soll immer der
dekadische Logarithmus gemeint sein):
\begin{equation}
  \pKa = -\log\Ka
\end{equation}
Damit gilt jetzt:
\begin{align*}
  \pKa &<0 \quad\text{starke Säure} \\
  \pKa &>0 \quad\text{schwache Säure}
\end{align*}

Völlig analoge Überlegungen führen auf die \emph{Basenkonstante}
\begin{equation}
  \Kb = \frac{ [\ch{OH-}] \cdot [\ch{HB+}] }{ [\ch{B}] } \label{eq:Kb_definition}
\end{equation}
und die Festlegung:
\begin{align*}
  \Kb &>1 \quad\text{starke Base} \\
  \Kb &<1 \quad\text{schwache Base}
\end{align*}
Aus analogen Gründen führt man den \emph{\pKb-Wert} ein:
\begin{equation}
  \pKb = -\log\Kb
\end{equation}
\begin{align*}
  \pKb &<0 \quad\text{starke Base} \\
  \pKb &>0 \quad\text{schwache Base}
\end{align*}

Sowohl \pKa- als auch \pKb-Werte sind charakteristische Werte für Säuren
bzw. Basen.  Viele davon wurden sehr gründlich bestiommt und können jederzeit
nachgeschlagen werden.  Tabelle~\ref{tab:pKa_pKb} listet einige davon auf.

\subsection{Der \pH-Wert}
Knüpfen wir uns noch einmal Reaktion\rctref{rct:autoprotolyse} vor:.
\begin{reaction*}
  H2O + H2O <<=> H3O+ + OH-
\end{reaction*}
Das Massenwirkungsgesetz für diese Reaktion lautet:
\begin{equation}
  K = \frac{ [\ch{H3O+}] \cdot [\ch{OH-}] }{ [\ch{H2O}]^2 }
\end{equation}
Wir stellen diese Gleichung um
\begin{align}
  K \cdot [\ch{H2O}]^2  &= [\ch{H3O+}] \cdot [\ch{OH-}] \\
  \intertext{und taufen das Produkt \(K \cdot [\ch{H2O}]^2 = \Kw\)}
  \Kw  &= [\ch{H3O+}] \cdot [\ch{OH-}] \label{eq:ionenprodukt}
\end{align}
Gleichung~\eqref{eq:ionenprodukt} nennt man auch das \emph{Ionenprodukt des
  Wassers}.  Bei \SI{25}{\celsius} beträgt
\[
  \Kw = \SI{1e-14}{\mole\squared\per\liter\squared}
      = \SI{0.00000000000001}{\mole\squared\per\liter\squared} \,.
\]
Das bedeutet, dass in neutralem Wasser die Konzentration von \ch{H3O+}
\begin{equation}
  [\ch{H3O+}] = [\ch{OH-}] = \SI{1e-7}{\mole\per\liter}
\end{equation}
beträgt.

Sauer wird es erst, wenn die Konzentration mehr als \SI{1e-7}{\mole\per\liter}
beträgt.  Liegt sie darunter, so ist die Lösung basisch.  Um die Angabe
einfacher zu machen, macht man nun folgendes:
\begin{align}
  [\ch{H3O+}]      &= \SI{1e-7}{\mole\per\liter} \\
  \log[\ch{H3O+}]  &= -7 \\
  -\log[\ch{H3O+}] &= 7 \\
  \intertext{Jetzt definiert man den \pH:}
  \pH &\equiv -\log[\ch{H3O+}] \label{eq:pH_definition}
\end{align}
Für neutrales Wasser gilt also bei \SI{25}{\celsius} $\pH = 7$.  Eine höhere
Konzentration würde einen niedrigeren \pH bedeuten.  Also ist es sauer, wenn
man einen $\pH < 7$ vorliegen hat, basisch bei $\pH > 7$ und neutral bei $\pH
= 7$.  Das alles gilt streng genommen nur für die schon erwähnten
\SI{25}{\celsius}, darum werden wir uns aber nicht mehr weiter kümmern.

\begin{definition}
  Der \pH{} (von \textit{\textbf{p}otentia \textbf{h}ydrogenii}\footnote{Das ist
    nicht \emph{völlig} richtig, soll uns aber auch nicht weiter
    beschäftigen.}) ist definiert als der negative dekadische Logarithmus der
  Konzentration der Oxonium"=Ionen (in \si{\mole\per\liter}).  Analog dazu ist
  der \pOH{} definiert als der negative dekadische Logarithmus der
  Konzentration der Hydroxid"=Ionen.
  \begin{align}
    \pH  &\equiv -\log[\ch{H3O+}] \tag{\ref{eq:pH_definition}} \\
    \pOH &\equiv -\log[\ch{OH-}]
  \end{align}
\end{definition}

Gleichung~\eqref{eq:ionenprodukt} führt auf einen Zusammenhang zwischen \pH
und \pOH:
\begin{align}
  \Kw         &= [\ch{H3O+}] \cdot [\ch{OH-}] \tag{\ref{eq:ionenprodukt}} \\
  \num{1e-14} &= [\ch{H3O+}] \cdot [\ch{OH-}] \\
  14          &= -\log[\ch{H3O+}] -\log[\ch{OH-}] \\
  14          &= \pH + \pOH \label{eq:ionensumme}
\end{align}

Eine ähnliche Vereinfachung kann man bei Gleichung~\eqref{eq:cHtO_aus_Ks}
vornehmen:
\begin{align}
  [\ch{H3O+}]      &=
    \Ka \cdot \frac{ [\ch{HA}] }{ [\ch{A-}] } \tag{\ref{eq:cHtO_aus_Ks}} \\
  \log[\ch{H3O+}]  &=
    \log\Ka + \log\biggl(\frac{ [\ch{HA}] }{ [\ch{A-}] }\biggr) \\
  -\log[\ch{H3O+}] &=
    -\log\Ka - \log\biggl(\frac{ [\ch{HA}] }{ [\ch{A-}] }\biggr) \\
  \pH             &=
    \pKa - \log\biggl(\frac{ [\ch{HA}] }{ [\ch{A-}] }\biggr)
    \label{eq:henderson_hasselbalch}
\end{align}
Gleichung~\eqref{eq:henderson_hasselbalch} nennt man die
\emph{Henderson"=Hasselbalch"=Gleichung}.  Sie ist allgemein gültig und vor
allem für die hier nicht besprochenen Pufferlösungen nützlich.

\subsection{\pH-Werte berechnen}
Stellen wir uns vor, wir hätten eine \SI{2.0}{\Molar} Essigsäure-Lösung, deren
$\Ka = \num{1.8e-5}$ beträgt, und wollten nun wissen, welche
\ch{H3O+}-Konzentration in dieser Lösung vorliegt, welchen \pH{} die Lösung also
hat.
\begin{reaction}
  !(Essigsäure)( CH3COOH ) + H2O <<=> !(Acetat)( CH3COO- ) + H3O+
\end{reaction}
Die Gleichgewichtskonzentration der Essigsäure beträgt nun
\(\SI{2.0}{\mole\per\liter}-x\), da ein gewisser Teil zerfallen ist.  Dafür
betragen die Acetat"=Konzentration und die Oxonium"=Konzentration beide \(x\),
entsprechen also beide dem zefallenen Teil der Essigsäure.
\begin{align*}
  [\ch{H3O+}] &= \Ka \cdot \frac{ [\ch{HA}] }{ [\ch{A-}] } \\
  x   &= \num{1.8e-5} \cdot \frac{ 2.0 - x }{ x } \\
  0   &= x^2 +\num{1.8e-5}x - \num{3.6e-5} \\
  x_1 &= \SI{0.006}{\mole\per\liter} \\
  x_2 &= \SI{-0.006}{\mole\per\liter}
\end{align*}
\(x_2\) kann offensichtlich nicht die Lösung sein.  Schließlich gibt es keine
negativen Konzentrationen.  Also beträgt die Oxonium"=Konzentration
\SI{0.006}{\mole\per\liter}.  Damit ist der \pH-Wert der Lösung
\begin{align*}
  \pH &= -\log(\SI{0.006}{\mole\per\liter}) \\
  \pH &= \num{2.2}
\end{align*}
Der \pOH{} beträgt damit \(\pOH = \num{11.8}\).

Bedeutet das nun, dass wir jedesmal, wenn wir den \pH wissen wollen, eine
quadratische Gleichung lösen müssen?  Nun, obwohl das auch nicht dramatisch
wäre, lautet die Antwort: Nein.

Bei einer starken Säure darf man dafür ausgehen, dass sie nahezu vollständig
dissoziert ist.  Das bedeutet, dass die Oxonium"=Konzentration in etwa der der
ursprunglich vorhandenen Säure \([\ch{HA}]_0\) entpsricht.
\begin{equation}
  \pH = -\log[\ch{HA}]_0 \qquad \text{starke Säure}
\end{equation}

Auch bei einer schwachen Säure gibt es eine näherungsweise Bestimmung:
\begin{gather}
  [\ch{H3O+}] = \Ka \cdot \frac{ [\ch{HA}] }{ [\ch{A-}] }
    \qquad  x = \Ka \cdot \frac{ [\ch{HA}]_0 - x }{ x } \\
  0 = x^2 + \Ka \cdot x - \Ka \cdot [\ch{HA}]_0
\end{gather}
Da bei einer schwachen Säure \Ka sehr klein ist und auch $x$ eher ein kleiner
Wert ist, vereinfachen wir die quadratische Gleichung weiter:
\begin{gather}
  0 = x^2 - \Ka \cdot [\ch{HA}]_0 \qquad x^2 = \Ka \cdot [\ch{HA}]_0 \\
  x = \sqrt{ \Ka \cdot [\ch{HA}]_0 } \\
  \pH = -\log\biggl(\Ka \cdot [\ch{HA}]_0 \biggr)^\frac{1}{2} \\
  \pH = \frac{1}{2} \biggl( -\log(\Ka) - \log[\ch{HA}]_0 \biggr) \\
  \pH = \frac{1}{2} \biggl( \pKa - \log[\ch{HA}]_0 \biggr)
    \qquad \text{schwache Säure}\label{eq:schwache_saeure}
\end{gather}

Wieder führen völlig analoge Überlegungen zu den entsprechenden Gleichungen
für Basen:
\begin{align}
  \pOH &= -\log[\ch{B}]_0 \quad \text{starke Base} \\
  \pOH &= \frac{1}{2} \biggl( \pKb - \log[\ch{B}]_0 \biggr)
    \quad \text{schwache Base}
\end{align}

Zuletzt gilt dann auch noch diese praktische Beziehung zwischen \pKa{} und
\pKb{}  korrespondierender
Säure/Base-Paare
\begin{equation}
  14 = \pKa + \pKb  \,,
\end{equation}
die wir unbewiesen hinnehmen wollen\footnote{Alle Interessierten können den
  Beweis gerne versuchen, er ist nicht schwer.}. Tabelle~\ref{tab:formeln}
fasst alle Formeln, die Sie kennen sollten, noch einmal zusammen.

\begin{table}
  \centering
  \caption{Übersicht über alle Formeln}\label{tab:formeln}
  \renewcommand*\arraystretch{1.75}
  \begin{tabular}{ll}
    \toprule
      \bfseries Anwendung   & \bfseries Formel \\
   \midrule
      \pH{} und \pOH
        & $\displaystyle\pH + \pOH = 14$ \\
      \pKa und \pKb
        & $\displaystyle\pKa + \pKb = 14$ \\
      starke Säure
        & $\displaystyle\pH = -\log[\text{Säure}]$ \\
      schwache Säure
        & $\displaystyle\pH = \frac{1}{2}(\pKa -\log[\text{Säure}])$ \\
      starke Base
        & $\displaystyle\pOH = -\log[\text{Base}]$ \\
      schwache Base
        & $\displaystyle\pOH = \frac{1}{2}(\pKb -\log[\text{Base}])$ \\
      Henderson-Hasselbalch
        & $\displaystyle\pH = \pKa - \log\biggl(\frac{ [\ch{HA}] }{ [\ch{A-}] }\biggr)$\\
    \bottomrule
  \end{tabular}
\end{table}

\section{Beispiele für Rechenaufgaben}
\begin{beispiel}
  Berechnen Sie den \pH-Wert einer \SI{2.4}{\Molar} Ammoniak-Lösung (\(\pKb =
  \num{4.7}\)) auf exaktem Weg.

  \noindent Lösung:
  \begin{reaction*}
    NH3 + H2O <=> NH4+ + OH-
  \end{reaction*}
  Die exakte Berechnung erfolgt über das Gleichgewicht, also über des
  Massenwirkungsgesetz:
  \begin{gather*}
    \Kb = \frac{[\ch{NH4+}]\cdot[\ch{OH-}]}{[\ch{NH3}]}
      \qquad \Kb = \frac{x^2}{2.4-x}
  \end{gather*}
  Damit ergibt sich eine quadratische Gleichung:
  \begin{gather*}
    0 = x^2 + \Kb\cdot x - 2.4\cdot\Kb \\
    0 = x^2 + 10^{-\pKb}\cdot x - 2.4 \cdot 10^{-\pKb} \\
    0 = x^2 + \num{2.0e-5}\cdot x - \num{4.8e-5}
  \end{gather*}
  Wir erhalten die folgende Lösung:
  \begin{gather*}
    x = [\ch{OH-}] = \SI{.0069}{\mole\per\liter} \\
    \Rightarrow\quad
    \pOH = \num{2.2}
    \quad\text{und}\quad
    \pH  = \num{11.8}
  \end{gather*}
\end{beispiel}

\begin{beispiel}
  Berechnen Sie den \pH-Wert einen Salzsäure (\(\pKa=\num{-6.2}\)) Lösung mit
  \(c = \SI{.5}{\milli\mole\per\liter}\).
  \noindent Lösung:
  \begin{reaction*}
    HCl + H2O <=> Cl- + H3O+
  \end{reaction*}
  Salzsäure ist \emph{die} klassische starke Säure, \(\pKa=\num{-6.2}\) ist
  deutlich unter Null.  Achten Sie darauf, die richtige Einheit für die
  Konzentration zu verwenden!
  \begin{gather*}
    \pH = -\log [\ch{HCl}] \\
    \pH = -\log \SI{.5e-3}{\mole\per\liter} \\
    \pH = \num{3.3}
  \end{gather*}
  Die Lösung hat \(\pH = \num{3.3}\).
\end{beispiel}

\begin{beispiel}
  Berechnen Sie den \pH-Wert einer Blausäure (\(\pKa = \num{9.4}\)) mit \(c =
  \SI{.25}{\mole\per\liter}\).
  \noindent Lösung:
  \begin{reaction*}
    HCN + H2O <=> CN- + H3O+
  \end{reaction*}
  Blausäure ist eine recht schwache wenn auch eine äußerst giftige Säure,
  \(\pKa=\num{9.4}\) ist klar über Null.
  \begin{gather*}
    \pH = \frac{1}{2}\cdot(\pKa - \log [\ch{HCN}]) \\
    \pH = \frac{1}{2}\cdot(9.4 - \log \SI{.25}{\mole\per\liter}) \\
    \pH = \num{5.0}
  \end{gather*}
  Die Lösung hat \(\pH = \num{5.0}\).
\end{beispiel}

\newpage
\section{Aufgaben}
\begin{question}
  Welche ist die korrespondierende Base von:
  \begin{tasks}(6)
    \task \ch{H3PO4}
    \task \ch{H2PO4-}
    \task \ch{NH3}
    \task \ch{HS-}
    \task \ch{H2SO4}
    \task \ch{HCO3-}
  \end{tasks}
\end{question}
\begin{solution}
  \begin{tasks}(6)
    \task \ch{H2PO4-}
    \task \ch{HPO4^2-}
    \task \ch{HH2^-}
    \task \ch{S^2-}
    \task \ch{HSO4-}
    \task \ch{CO3^2-}
  \end{tasks}
\end{solution}

\begin{question}
  Welche ist die korrespondierende Säure von:
  \begin{tasks}(6)
    \task \ch{H2O}
    \task \ch{HS-}
    \task \ch{NH3}
    \task \ch{H2AsO4-}
    \task \ch{F-}
    \task \ch{NO2-}
  \end{tasks}
\end{question}
\begin{solution}
  \begin{tasks}(6)
    \task \ch{H3O+}
    \task \ch{H2S}
    \task \ch{NH4+}
    \task \ch{H3AsO4}
    \task \ch{HF}
    \task \ch{HNO2}
  \end{tasks}
\end{solution}

\begin{question}
  Identifizieren Sie alle Br\o nstedt"=Säuren und -Basen:
  \begin{tasks}(2)
    \task \ch{NH3 + HCl <=> NH4+ + Cl-}
    \task \ch{HSO4- + CN- <=> HCN + SO4^2-}
    \task \ch{H2PO4- + CO3^2- <=> HPO4^2- + HCO3-}
    \task \ch{H3O+ + HS- <=> H2S + H2O}
    \task \ch{N2H4 + HSO4- <=> N2H5+ + SO4^2-}
    \task \ch{H2O + NH4+ <=> NH3 + H3O+}
  \end{tasks}
\end{question}
\begin{solution}
  \begin{tasks}(2)
    \task \ch{!(Base)( NH3 ) + !(Säure)( HC )l
      <=> !(Säure)( NH4+ ) + !(Base)( Cl- ) }
    \task \ch{!(Säure)( HSO4- ) + !(Base)( CN- )
      <=> !(Säure)( HCN ) + !(Base)( SO4^2- )}
    \task \ch{!(Säure)( H2PO4- ) + !(Base)( CO3^2- )
      <=> !(Base)( HPO4^2- ) + !(Säure)( HCO3- )}
    \task \ch{!(Säure)( H3O+ ) + !(Base)( HS- )
      <=> !(Säure)( H2S ) + !(Base)( H2O )}
    \task \ch{!(Base)( N2H4 ) + !(Säure)( HSO4- )
      <=> !(Säure)( N2H5+ ) + !(Base)( SO4^2- )}
    \task \ch{!(Base)( H2O ) + !(Säure)( NH4+ )
      <=> !(Base)( NH3 ) + !(Säure)( H3O+ )}
  \end{tasks}
\end{solution}

\begin{question}
  Formulieren Sie das Protolyse"=Gleichgewicht folgender Br\o nstedt"=Säuren:
  \begin{tasks}(6)
    \task \ch{H2O}
    \task \ch{HF}
    \task \ch{HSO3-}
    \task \ch{NH4+}
    \task \ch{HOCl}
  \end{tasks}
\end{question}
\begin{solution}
  \begin{tasks}(2)
    \task \ch{H2O + H2O <=> H3O+ + OH-}
    \task \ch{HF + H2O <=> H3O+ + F-}
    \task \ch{HSO3- + H2O <=> H3O+ + SO3^2-}
    \task \ch{NH4+ + H2O <=> H3O+ + NH3}
    \task \ch{HOCl + H2O <=> H3O+ + ^-OCl}
  \end{tasks}
\end{solution}

\begin{question}
  Formulieren Sie das Protolyse"=Gleichgewicht folgender Br\o nstedt"=Basen:
  \begin{tasks}(6)
    \task \ch{OH-}
    \task \ch{N^3-}
    \task \ch{H2O}
    \task \ch{HCO3-}
    \task \ch{O^2-}
    \task \ch{SO4^2-}
  \end{tasks}
\end{question}
\begin{solution}
  \begin{tasks}(2)
    \task \ch{OH- + H2O <=> OH- + H2O}
    \task \ch{N^3- + H2O <=> OH- + HN^2-}
    \task \ch{H2O + H2O <=> OH- + H3O+}
    \task \ch{HCO3- + H2O <=> OH- + H2CO3}
    \task \ch{O^2- + H2O <=> OH- + OH-}
    \task \ch{SO4^2- + H2O <=> OH- + HSO4-}
  \end{tasks}
\end{solution}

\begin{question}
  Bestimmen Sie die Konzentration \([\ch{H+}]\) und \([\ch{OH-}]\) in folgenden
  Lösungen:
  \begin{tasks}(2)
    \task \SI{.015}{\mole\per\liter} \ch{HNO3}
    \task \SI{.0025}{\mole\per\liter} \ch{Ba(OH)2}
    \task \SI{.00030}{\mole\per\liter} \ch{HCl}
    \task \SI{.016}{\mole\per\liter} \ch{Ca(OH)2}
  \end{tasks}
\end{question}
\begin{solution}
  \begin{tasks}
    \task $[\ch{H+}] = \SI{.015}{\mole\per\liter}$, $[\ch{OH-}] =
      \SI{6.67e-13}{\mole\per\liter}$
    \task $[\ch{H+}] = \SI{2.0e-12}{\mole\per\liter}$, $[\ch{OH-}] =
      \SI{.005}{\mole\per\liter}$
    \task $[\ch{H+}] = \SI{.00030}{\mole\per\liter}$, $[\ch{OH-}] =
      \SI{3.3e-11}{\mole\per\liter}$
    \task $[\ch{H+}] = \SI{3.1e-13}{\mole\per\liter}$, $[\ch{OH-}] =
      \SI{.032}{\mole\per\liter}$
  \end{tasks}
\end{solution}

\begin{question}
  Welchen \pH-Wert haben folgende Lösungen:
  \begin{tasks}(2)
    \task \([\ch{H+}] = \SI{7.3e-5}{\mole\per\liter}\)
    \task \([\ch{H+}] = \SI{.084}{\mole\per\liter}\)
    \task \([\ch{H+}] = \SI{3.9e-8}{\mole\per\liter}\)
    \task \([\ch{OH-}] = \SI{3.3e-4}{\mole\per\liter}\)
    \task \([\ch{OH-}] = \SI{.042}{\mole\per\liter}\)
  \end{tasks}
\end{question}
\begin{solution}
  \begin{tasks}(5)
    \task $\pH = 4.1$
    \task $\pH = 1.1$
    \task $\pH = 7.4$
    \task $\pH = 10.5$
    \task $\pH = 12.6$
  \end{tasks}
\end{solution}

\begin{question}
  Wie groß sind \([\ch{H+}]\) und \([\ch{OH-}]\) wenn folgende Werte gemessen
  wurden:
  \begin{tasks}(5)
    \task \(\pH = \num{1.23}\)
    \task \(\pH = \num{10.92}\)
    \task \(\pOH = \num{4.32}\)
    \task \(\pOH = \num{12.34}\)
    \task \(\pOH = \num{0.16}\)
  \end{tasks}
\end{question}
\begin{solution}
  \begin{tasks}
    \task $[\ch{H+}] = \SI{0.0589}{\mole\per\liter}$, $[\ch{OH-}] =
      \SI{1.70e-13}{\mole\per\liter}$
    \task $[\ch{H+}] = \SI{1.20e-11}{\mole\per\liter}$, $[\ch{OH-}] =
      \SI{8.32e-4}{\mole\per\liter}$
    \task $[\ch{H+}] = \SI{2.09e-10}{\mole\per\liter}$, $[\ch{OH-}] =
      \SI{4.79e-5}{\mole\per\liter}$
    \task $[\ch{H+}] = \SI{0.02188}{\mole\per\liter}$, $[\ch{OH-}] =
      \SI{5.571e-13}{\mole\per\liter}$
    \task $[\ch{H+}] = \SI{1.4e-14}{\mole\per\liter}$, $[\ch{OH-}] =
      \SI{0.69}{\mole\per\liter}$
  \end{tasks}
\end{solution}

\begin{question}
  Welchen \pH-Wert hat eine Lösung von \SI{.30}{\mole\per\liter} \ch{NH3}?
\end{question}
\begin{solution}
  \[
    \pH = 14 - \frac{1}{2}(4.7 -\log(0.30)) = 11.4
  \]
\end{solution}

\begin{question}
  Wieviel Mol Fluss"-säure benötigt man, um \SI{500}{\milli\liter} einer Lösung
  mit \(\pH=\num{2.60}\) herzustellen?
\end{question}
\begin{solution}
  \begin{gather*}
    \pH = \frac{1}{2}(\pKa -\log[\ch{HF}]) \\
    \log[\ch{HF}] = \pKa - 2\cdot\pH \\
    [\ch{HF}] = 10^{\pKa-2\cdot\pH} = \SI{0.02}{\mole\per\liter} \\
    n(\ch{HF}) = \SI{0.01}{\mole}
  \end{gather*}
\end{solution}

\begin{question}
  Welchen \pH-Wert hat eine Lösung von \SI{.12}{\mole} Cyansäure (\ch{HOCN},
  \(\pKa=\num{3.9}\)) pro Liter?
\end{question}
\begin{solution}
  \[
    \pH = \frac{1}{2}\biggl(\pKa
          -\log\biggl(\frac{n(\ch{HOCN})}{V}\biggr)\biggr)
        = 2.41
  \]
\end{solution}

\begin{question}
  Dichloressigsäure (\ch{Cl2HCCOOH}), eine einprotonige Säure, ist bei einer
  Konzentration von \SI{.20}{\mole\per\liter} zu \SI{33}{\percent}
  dissoziiert.  Wie groß sind \Ka, \pKa, \pH{} und \pOH?
\end{question}
\begin{solution}
  Es gilt die Reaktionsgleichung
  \begin{reaction*}
    Cl2CHCOOH + H2O <=> Cl2HCOO- + H3O+
  \end{reaction*}
  und das Massenwirkungsgesetz
  \[
    \Ka = \frac{[\ch{Cl2HCOO-}][\ch{H3O+}]}{[\ch{Cl2HCOOH}]} \,.
  \]
  Dass die Säure zu \SI{33}{\percent} disoziiert ist, bedeutet, dass
  im Gleichgewicht gilt $[\ch{Cl2HCOOH}] \approx 2\cdot[\ch{Cl2HCOO-}]$ und
  dass \SI{67}{\percent} der eingesetzten Säure im Gleichgewicht noch
  vorhanden sind.  Damit ergeben sich die Gleichgewichts"=Konzentrationen
  $[\ch{Cl2HCOOH}] = \SI{.134}{\mole\per\liter}$ und $[\ch{Cl2HCOO-}] =
  [\ch{H3O+}] = \SI{0.066}{\mole\per\liter}$.  Daraus errechnet sich die
  Säurekonstante zu $\Ka = \SI{0.033}{\mole\per\liter}$ und $\pKa = 1.5$.
  Dichloressigsäure ist also eine vergleichsweise starke Säure.  Aus der
  Gleichgewichts"=Konzentration ergeben sich $\pH = 1.2$ und $\pOH = 12.8$.
\end{solution}

\begin{question}
  In einer Lösung von Benzylamin (\ch{C6H5CH2NH2}) mit einer Konzentration von
  \SI{250}{\milli\mole\per\liter} ist \([\ch{OH-}] =
  \SI{2.4e-3}{\mole\per\liter}\).  Wie groß sind \pKb{} und \pH?
\end{question}
\begin{solution}
  Für die Reaktion
  \begin{reaction*}
    C6H5CH2NH2 + H2O <=> C6H5CH2NH3+ + OH-
  \end{reaction*}
  gilt das Massenwirkungsgesetz
  \[
    \Kb = \frac{[\ch{C6H5CH2NH3+}][\ch{OH-}]}{[\ch{C6H5CH2NH2}]} \,.
  \]
  Da im Gleichgewicht $[\ch{C6H5CH2NH3+}] = [\ch{OH-}]$ gilt und außerdem die
  Beziehung $[\ch{C6H5CH2NH2}] + [\ch{C6H5CH2NH3+}] =
  \SI{0.250}{\mole\per\liter}$ erfüllt sein muss, ergeben sich $\Kb =
  \SI{2.3e-5}{\mole\per\liter}$ und $\pKb = 4.6$.  Die Lösung hat damit den
  $\pH = 11.4$.
\end{solution}

\begin{question}
  Für Milchsäure ist $\Ka = \SI{1.5e-4}{\mole\per\liter}$.  Wie groß ist der
  \pH, wenn \SI{.16}{\mole\per\liter} Milchsäure in Lösung sind?
\end{question}
\begin{solution}
  Mit Gleichung~\eqref{eq:schwache_saeure} gilt $\pH = \frac{1}{2}(-\log\Ka -
  \log[\text{Milchsäure}])$, woraus sich $\pH = 2.3$ errechnet.
\end{solution}

\begin{question}
  Wie groß sind die Konzentrationen von \ch{N2H5+}, \ch{OH-} und \ch{N2H4}
  (Hydrazin), in einer Lösung von \SI{.15}{\mole\per\liter} Hydrazin?
\end{question}
\begin{solution}
  Es gilt die Reaktionsgleichung
  \begin{reaction*}
    N2H4 + H2O <=> N2H5+ + OH-
  \end{reaction*}
  mit dem Massenwirkungsgesetz
  \[
    \Kb = \frac{ [\ch{N2H5+}][\ch{OH-}] }{ [\ch{N2H4}] } \,.
  \]
  Für eine exakte Berechnung ergibt sich also die quadratische Gleichung
  \[
    \num{9.8e-7} = \frac{x\cdot x}{0.15 - x}
  \]
  mit den Lösungen $x_1 = \num{-3.8e-4}$ und $x_2 = \num{3.8e-4}$.  Da es
  keine negativen Konzentrationen geben kann, ist die zweite Lösung die
  richtige und die Gleichgewichtskonzentrationen betragen $[\ch{N2H5+}] =
  [\ch{OH-}] = \SI{3.8e-4}{\mole\per\liter}$ und $[\ch{N2H4}] =
  \SI{0.15}{\mole\per\liter}$.  (Beachten Sie die Rechen- und
  Messgenauigkeit.  Selbst wenn sich rechnerisch $[\ch{N2H4}] =
  \SI{0.1496}{\mole\per\liter}$ ergeben, erlauben die gegebenen Zahlenwerte
  nur eine Genauigkeit von zwei signifikanten Stellen.)
\end{solution}

\begin{table}
  \centering
  \caption{\pKa- und \pKb-Werte einiger Säuren und Basen.}\label{tab:pKa_pKb}
  \begin{tabular}{lCSS}
    \toprule
      \bfseries Säure     & \textbf{Formel}  &
        {\bfseries\Ka in \si[detect-all]{\mole\per\liter}} & {\bfseries\pKa} \\
    \midrule
      Oxonium             & H3O+    & 5.5     & -1.7 \\
      Wasser              & H2O     & 2.0e-16 & 15.7 \\
      Salzsäure           & HCl     & 1.3e6   & -6.1 \\
      Bromwasserstoff     & HBr     & 7.9e8   & -8.9 \\
      Flusssäure          & HF      & 6.6e-4  & 3.2 \\
      Schwefelsäure       & H2SO4   & 1.0e3   & -3.0 \\
                          & HSO4-   & 1.2e-2  & 1.9 \\
      Salpetersäure       & HNO3    & 20.9    & -1.3 \\
      Phosphorsäure       & H3PO4   & 7.4e-3  & 2.1 \\
                          & H2PO4-  & 6.3e-8  & 7.2 \\
                          & HPO4^2- & 4.4e-13 & 12.4 \\
      Kohlensäure         & H2CO3   & 3.0e-7  & 6.5 \\
                          & HCO3-   & 4.0e-11 & 10.4 \\
      Blausäure           & HCN     & 4.0e-10 & 9.4 \\
      Essigsäure          & CH3COOH & 1.8e-5  & 4.7 \\
      Ameisensäure        & HCOOH   & 1.8e-4  & 3.7 \\
      Schwefelwasserstoff & H2S     & 1.2e-7  & 6.9 \\
                          & HS-     & 1.0e-13 & 13.0 \\
    \midrule
      \bfseries Base      & \textbf{Formel}
        & {\bfseries \Kb} & {\bfseries \pKb} \\
    \midrule
      Hydroxid            & OH-     & 5.5     & -1.7 \\
      Wasser              & H2O     & 2.0e-16 & 15.7 \\
      Ammoniak            & NH3     & 1.8e-5  & 4.7 \\
      Hydrazin            & N2H4    & 9.8e-7  & 6.0 \\
      Methylamin          & CH3NH2  & 5.4e-4  & 3.3 \\
      Anilin              & C6H5NH2 & 4.3e-10 & 9.3 \\
      Pyridin             & C5H5N   & 1.5e-9  & 8.8 \\
    \bottomrule
  \end{tabular}
\end{table}

\clearpage
\addsec{Lösungen}
\printsolutions

\end{document}
