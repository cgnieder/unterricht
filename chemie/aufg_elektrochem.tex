\documentclass{scrartcl}

\usepackage[T1]{fontenc}
\usepackage[utf8]{inputenc}
\usepackage{libertine}
\usepackage[ngerman]{babel}
\usepackage{scrpage2}
\clearscrheadfoot
\pagestyle{scrheadings}
\chead{Seite \thepage}
\cfoot{\small C.\,Niederberger -- aktualisiert am \today}

\usepackage{upgreek}
\usepackage{chemmacros}
\chemsetup{
  option/language         = german ,
  chemformula/name-format = \centering
}
\usepackage{siunitx}
\sisetup{
  per-mode = fraction
}
\usepackage{enumitem}
\usepackage{exsheets}
\SetupExSheets{
  totoc,
  toc-level = section ,
  headings-format = \bfseries\sffamily
}
\usepackage{fnpct}

\usepackage{elektrochemie}

\usepackage{mdframed}
\newmdenv[
  skipabove=\baselineskip,
  skipbelow=\baselineskip,
  hidealllines=true,
  backgroundcolor=red!10
  ]{definition}

\usepackage[colorlinks]{hyperref}



\begin{document}
\begin{figure}
 \centering
 \setechemie{elektrolyse}
 \begin{tikzpicture}
  \drawcell
  % Teilchen:
  \foreach \x/\y in
    {
       .5*\elwidth+.2/.75*\elheight ,
       .5*\elwidth+.2/.5*\elheight ,
       .5*\elwidth+.3/.125*\elheight ,
       .5*\elwidth+.1/-.1 ,
      -.5*\elwidth-.2/.25*\elheight ,
      -.5*\elwidth-.2/.625*\elheight
    }
    { \draw[] ($(anode)+(\x,\y)$) circle (2pt) ; }
  \foreach \x/\y in
    {
      -.5*\elwidth-.2/.8*\elheight ,
      -.5*\elwidth-.15/.55*\elheight ,
      -.5*\elwidth-.3/-.1 ,
       .5*\elwidth+.1/.1 ,
       .5*\elwidth+.15/.6*\elheight ,
       .5*\elwidth+.15/.73*\elheight
    }
    { \draw[] ($(cathode)+(\x,\y)$) circle (2pt) ; }
  % Elektrolytbewegung:
  \node at ($(cathode)+(1.5,2.5)$) {\ch{->[ SO4^2- ]}};
  \node at ($(cathode)+(1.7,1.5)$) {\ch{<-[ Na+ ]}};
  \node at ($(anode)+(-1.5,1.8)$)  {\ch{<-[ Na+ ]}};
  \node at ($(anode)+(-2,1.2)$)    {\ch{->[ SO4^2- ]}};
  % Reaktionsgleichungen:
  \node[below,align=center] at ($(cathode)+(0,-1)$)
    {
      Kathode \\
      \ch{2 e- + 2 H2O -> H2\gas{} + 2 OH-} \\
    } ;
  \node[below,align=center] at ($(anode)+(0,-1)$)
    {
      Anode \\
      \ch{2 H2O -> O2\gas{} + 4 H+ + 4 e-} \\
    } ;
 \end{tikzpicture}
 \caption{Versuchsanordnung zur Elektrolyse einer wässrigen \ch{Na2SO4}-Lösung}
\end{figure}

\begin{figure}
 \centering
 \setechemie{
   membrane ,
   electrodes = {Cu}{Zn} ,
   anode-color = {red!20} ,
   electrolyte-color = {blue!5}
 }
 \begin{tikzpicture}
  \drawcell
 \end{tikzpicture}
\end{figure}


\end{document}