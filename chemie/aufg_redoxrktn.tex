\documentclass{scrartcl}

\usepackage[T1]{fontenc}
\usepackage[utf8]{inputenc}
\usepackage{libertine}
\usepackage{microtype}
\usepackage[ngerman]{babel}
\usepackage{scrpage2}
\clearscrheadfoot
\pagestyle{scrheadings}
\chead{Seite \thepage}
\cfoot{\small C.\,Niederberger -- aktualisiert am \today}

\usepackage{upgreek}
\usepackage{chemmacros}
\chemsetup{
  option/language         = german ,
  chemformula/name-format = \centering
}
\usepackage{siunitx}
\sisetup{
  per-mode = fraction
}
\usepackage{enumitem}
\usepackage{exsheets}
\SetupExSheets{
  totoc,
  toc-level = subsection ,
  headings-format = \bfseries\sffamily
}
\usepackage{fnpct}

\usepackage{xcolor}
\usepackage{mdframed}
\newmdenv[
  skipabove=\baselineskip,
  skipbelow=\baselineskip,
  hidealllines=true,
  backgroundcolor=red!10
  ]{definition}

\usepackage[colorlinks]{hyperref}

\begin{document}

\begin{center}
  \Huge\sffamily Redoxreaktionen
\end{center}

\addsec{Oxidationszahlen}

\begin{definition}
  \textbf{Oxidationszahlen} entsprechen bei Atomen und atomaren Ionen der
  Ladung des entsprechenden Teilchens. Bei Molekülen entsprechen die
  Oxidationszahlen der Ladung, den eine entsprechende ionische Verbindung
  hätte.
\end{definition}%
Um Oxidationszahlen aufstellen zu können, wendet man folgende Regeln an. Wenn
die Regeln sich widersprechen, wendet man die mit der kleineren Nummer an.
\begin{enumerate}[label=\arabic*.]
  \item Reine Elemente haben OZ $=0$.
  \item Metalle haben \emph{immer} positive OZ, bei Hauptgruppenmetallen
    entspricht sie der Hauptgruppennummer.
  \item Die OZ in Verbindungen addieren sich \emph{immer} zur Gesamtladung der
    Verbindung.
  \item Fluor (\ch{F}) hat die OZ $=-1$.
  \item Wasserstoff (\ch{H}) hat die OZ $=+1$.
  \item Sauerstoff (\ch{O}) hat die OZ $=-2$; Ausnahmen: in Peroxiden ($-1$)
    und \ch{OF2} ($+2$)
  \item In Verbindungen von Nichtmetallen hat das elektronegativere Element
    die Oxidationszahl entsprechend seiner Hauptgruppennummer.
\end{enumerate}

\addsec{Aufstellen von Redoxgleichungen}

Redoxreaktionen sind ohne Übung und eine bestimmte Vorgehensweise oft nicht
leicht aufzustellen. Darum werden hier die Regeln aufgestellt, die -- richtig
angewendet -- immer sicher zum Ziel führen.
\begin{enumerate}[label=\arabic*.]
  \item Zuerst muss man mit Hilfe der Oxidationszahlen das \emph{Oxidations-}
    und das \emph{Reduktionspaar} bestimmen.
  \item Dann stellt man die Oxidations- und Reduktionsreaktion
    (Halbreaktionen) auf, um die übertragene \emph{Elektronenzahl anzupassen}.
  \item Setzt man die Reaktion wieder zusammen, muss anschließend ein
    \emph{Ladungsausgleich} mit \ch{H+} bzw. \ch{OH-} je nach saurem oder
    basischen Milieu durchgeführt werden.
  \item Nun muss man die \emph{Sauerstoff- und Wasserstoff-Anzahl} mit
    \ch{H2O} \emph{ausgleichen}.
  \item Eventuell müssen nun noch an der Reaktion nicht beteiligte Ionen
    vervollständigt werden.
\end{enumerate}

\minisec{Beispiel 1}

In saurer Lösung regieren \ch{I2} und \ch{ClO3-} (Chlorat) unter Bildung von
\ch{IO3-} (Iodat) und \ch{Cl-}:
\begin{reactions*}
  I2 + ClO3- &-> IO3- + Cl- \\
  \intertext{Zuerst werden die Oxidationszahlen bestimmt.}
  "\ox{0,I}" {}_2 + "\ox{+5,Cl}" "\ox{-2,O}" {}_3-
    &-> "\ox{+5,I}" "\ox{-2,O}" {}_3 + "\ox{-1,Cl}" ^- \\
  \intertext{Jetzt muss man herausfinden, wieviele \el\ übertragen
    wurden. Dazu stellt man die Oxi\-da\-tions- und die Reduktionsreaktion
    auf.  \emph{Achten Sie darauf, dass hier die Anzahl der Iod-Atome links
      und rechts übereinstimmen muss!}}
  "\ox[parse=false]{\textcolor{red}{0},I}" {}_2 &-> 2
  "\ox[parse=false]{\textcolor{red}{V},I}" O_3-
  + $2\cdot{\color{red}5}$ e- & OX \\
  "\ox[parse=false]{\textcolor{red}{V},Cl}" O_3- + ${\color{red}6}$ e- &->
  "\ox[parse=false]{\textcolor{red}{$-$I},Cl}" ^- & RED \\
  \intertext{Da natürlich gleich viele Elektronen abgegeben wie aufgenommen
    werden müssen, muss man die Elektronenzahl angleichen.}
  ${\color{red}3}$ I2 &-> ${\color{red}3}\cdot 2$ IO3- + ${\color{red}3}\cdot 10$ e- & OX  \\
  ${\color{red}5}$ ClO3- + ${\color{red}5}\cdot 6$ e- &-> ${\color{red}5}$ Cl- & RED \\
  \intertext{Setzt man die beiden Gleichungen wieder zusammen, ist man fast am
    Ziel.}
  3 I2 + 5 ClO3- &-> 6 IO3- + 5 Cl- \\
  \intertext{Links und rechts des Reaktionspfeils müssen die Ladungen
    übereinstimmen.  Dafür sorgt man, indem man eine entsprechende Anzahl an
    \Hpl\ (im Sauren, so wie hier) oder \Hyd\ (im Basischen) hinzugibt.}
  3 I2 + 5 ClO3- &-> 6 IO3- + 5 Cl- \color{red} + 6 H+ \\
  \intertext{Zuletzt muss noch die Anzahl der Sauerstoff- und
    Wasserstoff-Atome mit \water\ ausgeglichen werden.}  3 I2 + 5 ClO3-
  \color{red} + 3 H2O\normalcolor &-> 6 IO3- + 5 Cl- + 6 H+
\end{reactions*}

\minisec{Beispiel 2}

Permanganat (\ch{MnO4-}) und Hydrazin (\ch{N2H4}) reagieren in basischer
Lösung zu Braunstein (\ch{MnO2}) und Stickstoff:
\begin{reactions*}
  "\ox{+7,Mn}\ox{-2,O}" {}_4- + "\ox{-2,N}" {}_2 "\ox{+1,H}" {}_4 &->
    "\ox{+4,Mn}\ox{-2,O}" {}_2 + "\ox{0,N}" {}_2 \\
  \intertext{Halbreaktionen:}
  "\ox[parse=false]{\textcolor{red}{VII},Mn}" O4- + ${\color{red}3}$ e- &->
    "\ox[parse=false]{\textcolor{red}{IV},Mn}" O2 & RED \\
  "\ox[parse=false]{\textcolor{red}{$-$II},N}" {}_2 H4 &->
    "\ox[parse=false]{\textcolor{red}{0},N}" {}_2 + ${\color{red}4}$ e- & OX \\
  \intertext{Elektronenzahl anpassen:}
  4 "\ox{+7,Mn}" O4- + ${\color{red}12}$ e- &-> 4 "\ox{+4,Mn}" O2 & RED \\
  3 "\ox{-2,N}" 2H4 &-> 3 "\ox{0,N}" {}_2 + ${\color{red}12}$ e- & OX \\ \\
  4 MnO4- + 3 N2H4 &-> 4 MnO2 + 3 N2 \\
  \intertext{Ladungsausgleich:}
  4 MnO4- + 3 N2H4 &-> 4 MnO2 + 3 N2 \color{red} + 4 OH- \\
  \intertext{H- und O-Anzahl ausgleichen:}
  4 MnO4- + 3 N2H4 &-> 4 MnO2 + 3 N2 + 4 OH- \color{red} + 4 H2O
\end{reactions*}

\addsec{Übungen}

\begin{question}
  Bestimmen Sie die Oxidationszahlen in folgenden Verbindungen.
  \begin{tasks}(4)
    \task \ch{U2Cl10}
    \task \ch{BiO+}
    \task \ch{Na6V10O28}
    \task \ch{K2SnO3}
    \task \ch{Ta6O19^8-}
    \task \ch{K2Ti2O5}
    \task \ch{Mg(BF4)2}
    \task \ch{Cs2TeF8}
    \task \ch{K2W4O13}
    \task \ch{N2H4}
    \task \ch{H2NOH}
    \task \ch{S2O5Cl2}
    \task \ch{Na3P3O9}
    \task \ch{CaN2O2}
    \task \ch{XeO6^4-}
    \task \ch{TaOCl3}
    \task \ch{Ca2Sb2O7}
    \task \ch{B2Cl4}
    \task \ch{H6TeO6}
    \task \ch{UO2^2+}
    \task \ch{BrF6-}
  \end{tasks}
\end{question}
\begin{solution}
  \begin{tasks}(4)
    \task \ch{ "\ox{5,U}" {}2 "\ox{-1,Cl}" {}10}
    \task \ch{ "\ox{3,Bi}\ox{-2,O}" {}+}
    \task \ch{ "\ox{1,Na}" {}6 "\ox{5,V}" {}10 "\ox{-2,O}" {}28}
    \task \ch{ "\ox{1,K}" {}2 "\ox{2,Sn}\ox{-2,O}" {}3}
    \task \ch{ "\ox{5,Ta}" {}6 "\ox{-2,O}" {}19^8-}
    \task \ch{ "\ox{1,K}" {}2 "\ox{4,Ti}" {}2 "\ox{-2,O}" {}5}
    \task \ch{ "\ox{2,Mg}" {}( "\ox{3,B}\ox{-1,F}" {}4)2}
    \task \ch{ "\ox{1,Cs}" {}2 "\ox{3,Te}\ox{-1,F}" {}8}
    \task \ch{ "\ox{1,K}" {}2 "\ox{6,W}" {}4 "\ox{-2,O}" {}13}
    \task \ch{ "\ox{-2,N}" {}2 "\ox{1,H}" {}4}
    \task \ch{ "\ox{1,H}" {}2 "\ox{-1,N}\ox{-2,O}\ox{1,H}"}
    \task \ch{ "\ox{6,S}" {}2 "\ox{-2,O}" {}5 "\ox{-1,Cl}" {}2}
    \task \ch{ "\ox{1,Na}" {}3 "\ox{5,P}" {}3 "\ox{-2,O}" {}9}
    \task \ch{ "\ox{2,Ca}\ox{1,N}" {}2 "\ox{-2,O}" {}2}
    \task \ch{ "\ox{8,Xe}\ox{-2,O}" {}6^4-}
    \task \ch{ "\ox{5,Ta}\ox{-2,O}\ox{-1,Cl}" {}3}
    \task \ch{ "\ox{2,Ca}" {}2 "\ox{6,Sb}" {}2 "\ox{-2,O}" {}7}
    \task \ch{ "\ox{2,B}" {}2 "\ox{-1,Cl}" {}4}
    \task \ch{ "\ox{1,H}" {}6 "\ox{6,Te}\ox{-2,O}" {}6}
    \task \ch{ "\ox{6,U}\ox{-2,O}" {}2^2+}
    \task \ch{ "\ox{5,Br}\ox{-1,F}" {}6-}
  \end{tasks}
\end{solution}

\begin{question}
  Formulieren Sie für die angegebenen Reaktionen Teilgleichungen für die
  Oxidation und Reduktion und die Redoxgleichung für diese Reaktion.
  Kennzeichnen Sie in der Redoxreaktion Reduktionsmittel und Oxidationsmittel.
  \begin{tasks}
    \task Aluminium reagiert mit Chlor zu Aluminiumchlorid.
    \task Natrium reagiert beim Kontakt mit Wasser zu Natronlauge und Wasserstoff.
    \task Leitet man Chlorgas in Ammoniakgas ein, so kommt es zur Bildung von
      Stickstoff und Chlorwasserstoffgas.
    \task Wenn man konzentrierte Schwefelsäure mit Kohlenstoff (\ch{C}) erhitzt, dann
      entstehen Schwefeldioxid und ein anderes, farbloses Gas. Wenn man dieses Gas in
      Calciumhydroxidlösung ("`Kalkwasser"') leitet, entsteht eine weiße Trübung
      (Niederschlag).
    \task Schwefelwasserstoff wird in Chlorwasser eingeleitet. Als Reaktionsprodukt
      entstehen Chloridionen und ein gelber Feststoff.
    \task Reaktion einer Kaliumpermanganatlösung mit Wasserstoffperoxidlösung im
      alkalischen Medium. Es entstehen Braunstein (\ch{MnO2}) und Sauerstoff.
    \task Chlor reagiert mit Natronlauge. Es entstehen Chlorid und Hypochlorit
      (\ch{OCl-}).
  \end{tasks}
\end{question}
\begin{solution}
  Oxidationsmittel ist immer dasjenige, das selbst reduziert wird. Umgekehrt
  ist das Reduktionsmittel dasjenige, das oxidiert wird.
  \begin{tasks}
    \task \ch{2 Al + 3 Cl2 -> 2 AlCl3},\\
      Oxidation: \ch{Al -> Al^3+ + 3 e-},\\
      Reduktion: \ch{Cl2 + 2 e- -> 2 Cl-}
    \task \ch{2 Na + 2 H2O -> 2 NaOH + H2},\\
      Oxidation: \ch{Na -> Na+ + e-},\\
      Reduktion: \ch{2 H2O + 2 e- -> 2 OH- + H2}
    \task \ch{3 Cl2 + 2 NH3 -> N2 + 6 HCl},\\
      Oxidation: \ch{2 NH3 -> N2 + 6 e- + 6 H+},\\
      Reduktion: \ch{Cl2 + 2 e- -> 2 Cl-}
    \task \ch{2 SO4^2- + C -> 2 SO2 + CO2 + 2 H2O},\\
      Oxidation: \ch{C + 2 H2O -> CO2 + 4 e- + 4 H+},\\
      Reduktion: \ch{4 H+ + SO4^2- + 2 e- -> SO2 + 2 H2O}
    \task \ch{H2S + Cl2 -> 2 Cl- + S + 2 H+},\\
      Oxidation: \ch{S^2- -> S + 2 e-},\\
      Reduktion: \ch{Cl2 + 2 e- -> 2 Cl-}
    \task \ch{2 KMnO4 + 3 H2O2 -> 2 MnO2 + 3 O2 + 2 KOH + 2 H2O},\\
      Oxidation: \ch{H2O2 + 2 OH- -> O2 + 2 H2O + 2 e-},\\
      Reduktion: \ch{MnO4- + 2 H2O + 3 e- -> MnO2 + 4 OH-}
    \task \ch{Cl2 + 2 NaOH -> NaCl + NaOCl + H2O},\\
      Oxidation: \ch{Cl2 + 4 OH- -> 2 OCl- + 2 H2O + 2 e-},\\
      Reduktion: \ch{Cl2 + 2 e- -> 2 Cl-}
  \end{tasks}
\end{solution}

\begin{question}\chemsetup[ox]{pos=side}
  Stellen Sie unter Verwendung der Schrittfolge die Reaktionsgleichungen für
  folgende Redoxprozesse auf. Alle Reaktionen finden in wässriger Lösung
  statt.
  \begin{tasks}
    \task \ox{3,Eisen}-Ionen reagieren mit Iodid-Ionen zu \ox{2,Eisen}-Ionen und Iod.
    \task Dichromat-Ionen (\ch{Cr2O7^2-}) reagieren mit Iodid-Ionen zu Iod und
      \ox{3,Chrom}-Ionen. Die Reaktion findet im sauren \pH-Wert-Bereich statt.
    \task Schweflige Säure (\ch{H2SO3}) reagiert mit Iod zu Schwefelsäure und
      Iodwasserstoff.
    \task \ox{3,Chrom}-Oxid reagiert mit Nitrat-Ionen zu Chromat-Ionen (\ch{CrO4^2-})
      und Nitrit-Ionen (\ch{NO2-}). Dabei werden \Hpl-Ionen frei.
    \task Quecksilber reagiert mit Salpetersäure (\Hpl\ und \ch{NO3-}) zu
      \ox{2,Quecksilber}-Ionen und Stickstoffmonoxid. Als Nebenprodukt entsteht Wasser.
    \task Iod und Chlor reagieren zu Iodat-Ionen (\ch{IO3-}) und Chlorid-Ionen.
    \task Stickstoffmonoxid und Salpetersäure reagieren zu Distickstofftetroxid und
      Wasser.
  \end{tasks}
\end{question}
\begin{solution}
  \begin{tasks}
    \task \ch{2 Fe^3+ + 2 I- -> 2 Fe^2+ + I2}
    \task \ch{Cr2O7^2- + 6 I- + 14 H+ -> 2 Cr^3+ + 3 I2 + 7 H2O}
    \task \ch{H2SO3 + I2 + H2O -> H2SO4 + 2 HI}
    \task \ch{2 Cr^3+ + 3 NO3- + 5 H2O -> 2 CrO4^2- + 3 NO2- + 10 H+}
    \task \ch{3 Hg + 2 HNO3 + 6 H+ -> 3 Hg^2+ + 2 NO + 4 H2O}
    \task \ch{I2 + 5 Cl2 + 6 H2O -> 2 IO3^- + 10 Cl- + 12 H+}
    \task \ch{2 NO + 4 HNO3 -> 3 N2O4 + 2 H2O}\\
      Diese Reaktion ist ein Beispiel für eine Komproportionierung.
  \end{tasks}
\end{solution}

\begin{question}
  Vervollständigen Sie die folgenden Redoxreaktionen und erläutern Sie anhand
  dieser Gleichungen die Begriffe Oxidation, Reduktion, Disproportionierung
  und Komproportionierung.
  \begin{tasks}(2)
     \task \ch{Br2 + OH- -> Br- + BrO3-}
     \task \ch{Mn^2+ + MnO4- + OH- -> MnO2}
  \end{tasks}
\end{question}
\begin{solution}
  \begin{tasks}
    \task \ch{6 Br2 + 6 H2O -> 10 Br- + 2 BrO3- + 12 H+}\\
      Diese Reaktion ist eine Disproportionierung. Brom ändert seine
      Oxidationsstufe gleichzeitig von $\pm0$ zu $-1$ (Bromid) und zu
      $+5$ (Bromat).
    \task \ch{3 Mn^2+ + 3 MnO4- + 2 H2O -> 5 MnO2 + 4 H+}\\
      Diese Reaktion ist eine Komproportionierung. Mangan hat zu Beginn die
      Oxidationsstufen $+2$ und $+7$ (Permanganat), die sich beide zu $+4$
      ändern.
  \end{tasks}
\end{solution}

\begin{question}
  Vervollständigen Sie die folgenden Gleichungen für Redoxreaktionen, die in
  saurer wässriger Lösung ablaufen.
  \begin{tasks}(2)
    \task \ch{Cr2O7^2- + H2S -> Cr^3+ + S}
    \task \ch{P4 + HOCl -> H3PO4 + Cl-}
    \task \ch{Cu + NO3- -> Cu^2+ + NO}
    \task \ch{PbO2 + I- -> PbI2 + I2}
    \task \ch{ClO3- + I- -> Cl- + I2}
    \task \ch{Zn + NO3- -> Zn^2+ + NH4+}
    \task \ch{H3AsO3 + BrO3- -> H3AsO4 + Br-}
    \task \ch{H2SeO3 + H2S -> Se + HSO4-}
    \task \ch{ReO2 + Cl2 -> HReO4 + Cl-}
    \task \ch{AsH3 + Ag+ -> As4O6 + Ag}
    \task \ch{Mn^2+ + BiO3- -> MnO4- + Bi^3+}
    \task \ch{NO + NO3- -> N2O4}
    \task \ch{MnO4- + HCN + I- -> Mn^2+ + ICN}
    \task \ch{Zn + H2MoO4 -> Zn^2+ + Mo^3+}
    \task \ch{IO3- + N2H4 -> I- + N2}
    \task \ch{S2O3^2- + IO3- + Cl- -> SO4^2- + ICl2-}
    \task \ch{Se + BrO3- -> H2SeO3 + Br-}
    \task \ch{H5IO6 + I- -> I2}
    \task \ch{Pb3O4 -> Pb^2+ + PbO2}
    \task \ch{As2S3 + ClO3- -> H3AsO4 + S + Cl-}
    \task \ch{XeO3 + I- -> Xe + I3-}
  \end{tasks}
\end{question}
\begin{solution}
  \begin{tasks}
    \task \ch{Cr2O7^2- + 3 H2S + 8 H+ -> 2 Cr^3+ + S + 7 H2O}
    \task \ch{P4 + 10 HOCl + 6 H2O -> 4 H3PO4 + 10 Cl- + 10 H+}
    \task \ch{3 Cu + 2 NO3- + 8 H+ -> 3 Cu^2+ + 2 NO + 4 H2O}
    \task \ch{PbO2 + 4 I- + 4 H+ -> PbI2 + I2 + 2 H2O}
    \task \ch{ClO3- + 6 I- + 6 H+ -> Cl- + 3 I2 + 3 H2O}
    \task \ch{4 Zn + NO3- + 10 H+ -> 4 Zn^2+ + NH4+ + 3 H2O}
    \task \ch{3 H3AsO3 + BrO3- -> 3 H3AsO4 + Br-}
    \task \ch{2 H2SeO3 + H2S -> 2 Se + HSO4- + 2 H2O + H+}
    \task \ch{2 ReO2 + 3 Cl2 + 4 H2O -> 2 HReO4 + 6 Cl- + 6 H+}
    \task \ch{4 AsH3 + 24 Ag+ + 6 H2O -> As4O6 + 24 Ag + 24 H+}
    \task \ch{2 Mn^2+ + 5 BiO3- + 14 H+ -> 2 MnO4- + 5 Bi^3+ + 7 H2O}
    \task \ch{2 NO + 4 NO3- + 4 H+ -> 3 N2O4 + 2 H2O}
    \task \ch{2 MnO4- + 5 HCN + 5 I- + 11 H+ -> 2 Mn^2+ + 5 ICN + 8 H2O}
    \task \ch{3 Zn + 2 H2MoO4 + 12 H+ -> 3 Zn^2+ + 2 Mo^3+ + 8 H2O}
    \task \ch{2 IO3- + 3 N2H4 -> 2 I- + 3 N2 + 6 H2O}
    \task \ch{S2O3^2- + IO3- + Cl- -> SO4^2- + ICl2-}
    \task \ch{3 Se + 2 BrO3- 3 H2O -> 3 H2SeO3 + 2 Br-}
    \task \ch{H5IO6 + 7 I- + 7 H+ -> 4 I2 + 6 H2O}
    \task \ch{Pb3O4 + 4 H+ -> 2 Pb^2+ + PbO2 + 2 H2O}\\
      \ch{Pb3O4}, auch Mennige genannt, enthält zwei \ox*{2,Pb}- und ein
      \ox*{4,Pb}-Atom pro Einheit. Das bedeutet, dass hier gar keine
      Redox-Reaktion stattgefunden hat.
    \task \ch{3 As2S3 + 5 ClO3- + 9 H2O -> 6 H3AsO4 + 9 S + 5 Cl-}
    \task \ch{XeO3 + 9 I- + 6 H+ -> Xe + 3 I3- + 3 H2O}
  \end{tasks}
\end{solution}

\begin{question}
  Vervollständigen Sie die folgenden Gleichungen für Redoxreaktionen, die in
  basischer wässriger Lösung ablaufen.
  \begin{tasks}(2)
    \task \ch{ClO2- -> ClO2 + Cl-}
    \task \ch{MnO4- + I- -> MnO4^2- + IO4-}
    \task \ch{P4 -> H2PO2- + PH3}
    \task \ch{SbH3 -> Sb(OH)4- + H2}
    \task \ch{OC(NH2)2 + OBr- -> CO3^2- + N2 + Br-}
    \task \ch{Mn(OH)2 + O2 -> Mn(OH)3}
    \task \ch{Cl2 -> ClO3- + Cl-}
    \task \ch{S^2- + I2 -> SO4^2- + I-}
    \task \ch{CN- + MnO4- -> OCN- + MnO2}
    \task \ch{Au + CN- + O2 -> Au(CN)2-}
    \task \ch{Si -> SiO3^2- + H2}
    \task \ch{Cr(OH)3 + OBr- -> CrO4^2- + Br-}
    \task \ch{I2 + Cl2 -> H3IO6^2- + Cl-}
    \task \ch{Al -> Al(OH)4- + H2}
    \task \ch{Al + NO3- -> Al(OH)4- + NH3}
    \task \ch{Ni^2+ + Br2 -> NiO(OH) + Br-}
    \task \ch{S -> SO3^2- + S^2-}
    \task \ch{S^2- + HO2- -> SO4^2-}
  \end{tasks}
\end{question}
\begin{solution}
  \begin{tasks}
    \task \ch{5 ClO2- + 2 H2O -> 4 ClO2 + Cl- + 4 OH-}
    \task \ch{8 MnO4- + I- 8 OH- -> 8 MnO4^2- + IO4- + 4 H2O}
    \task \ch{P4 + 3 H2O + 3 OH- -> 3 H2PO2- + PH3}
    \task \ch{SbH3 + OH- + 3 H2O -> [Sb(OH)4]- + 3 H2}
    \task \ch{OC(NH2)2 + 3 OBr- 2 OH- -> CO3^2- + N2 + 3 Br- + 3 H2O}
    \task \ch{4 Mn(OH)2 + O2 + 2 H2O -> 4 Mn(OH)3}
    \task \ch{3 Cl2 + 6 OH- -> ClO3- + 5 Cl- + 3 H2O}
    \task \ch{S^2- + 4 I2 + 8 OH- -> SO4^2- + 8 I- + 4 H2O}
    \task \ch{3 CN- + 2 MnO4- + H2O -> 3 OCN- + 2 MnO2 + 2 OH-}
    \task \ch{4 Au + 8 CN- + O2 + 2 H2O -> 4 [Au(CN)2]- + 4 OH-}
    \task \ch{Si + 2 OH- + H2O -> SiO3^2- + 2 H2}
    \task \ch{2 Cr(OH)3 + 3 OBr- + 4 OH- -> 2 CrO4^2- + 3 Br- + 5 H2O}
    \task \ch{I2 + 7 Cl2 + 18 OH- -> 2 H3IO6^2- + 14 Cl- + 6 H2O}
    \task \ch{2 Al + 2 OH- + 6 H2O -> 2 [Al(OH)4]- + 3 H2}
    \task \ch{8 Al + 3 NO3- 5 OH- + 18 H2O -> 8 [Al(OH)4]- + 3 NH3}
    \task \ch{2 Ni^2+ + Br2 + 6 OH- -> 2 NiO(OH) + 2 Br- + 2 H2O}
    \task \ch{3 S + 6 OH- -> SO3^2- + 2 S^2- + 3 H2O}
    \task \ch{S^2- + 4 HO2- -> SO4^2- + 4 OH-}
  \end{tasks}
\end{solution}

\begin{question}
  Vervollständigen Sie die folgenden Gleichungen, die nicht in wässriger
  Lösung sondern in der Schmelze oder in der Gasphase ablaufen. (Das heißt,
  dass Sie weder einen Ladungsausgleich mit \Hpl/\Hyd{} machen können noch
  einen Wasserstoff-/Sauerstoff-Ausgleich mit \ch{H2O}.
  \begin{tasks}
    \task \ch{Cr2O3 + NO3- + CO3^2- -> CrO4^2- + NO2- + CO2}
    \task \ch{Ca3(PO4)2 + C + SiO2 -> P4 + CaSiO3 + CO}
    \task \ch{Mn3O4 + Na2O2 -> Na2MnO4 + Na2O}
    \task \ch{Si3N4 + SrO + C + N2 -> Sr2Si5N8 + CO}
    \task \ch{NH3\gas{} + O2\gas{} -> NO\gas{} + H2O\gas}
  \end{tasks}
\end{question}
\begin{solution}
  \begin{tasks}
    \task \ch{Cr2O3 + 3 NO3- + 2 CO3^2- -> 2 CrO4^2- + 3 NO2- + 2 CO2}
    \task \ch{2 Ca3(PO4)2 + 10 C + 6 SiO2 -> P4 + 6 CaSiO3 + 10 CO}
    \task \ch{Mn3O4 + 5 Na2O2 -> 3 Na2MnO4 + 2 Na2O}
    \task \ch{5 Si3N4 + 6 SrO + 6 C + 2 N2 -> 3 Sr2Si5N8 + 6 CO}
    \task \ch{4 NH3\gas{} + 5 O2\gas{} -> 4 NO\gas{} + 6 H2O\gas}
  \end{tasks}
\end{solution}

\addsec{Lösungen}
\printsolutions

\end{document}
