% arara: pdflatex: { shell: on }
% arara: pdflatex: { shell: on }
\documentclass{arbeitsblatt}
\usepackage[T1]{fontenc}
\usepackage[utf8]{inputenc}

\title{(Quadratische) Funktionen}

\usepackage{silence}
\WarningFilter{microtype}{I cannot find a protrusion list}

\usepackage[sort,only-used=false]{acro}
\acsetup{
  first-style=short ,
  short-format=\scshape ,
  list-short-format = ,
  page-ref = plain ,
  list-heading = section ,
  list-name = Begriffe
}

\input{mathe-kl10.acro}

\begin{document}

\maketitle

\begin{abstract}
  Diese Übersicht soll grundsätzlich nur quadratische Funktionen behandeln und
  alle typischen Aufgabestellungen enthalten, wie sie in einer
  Realschulprüfung vorkommen können.  Da dort das Thema "`quadratische
  Funktionen"' aber nicht klar von den linearen Funktionen zu trennen sind,
  werden diese hier auch kurz behandelt.
  
  Am Ende der Übersicht befindet sich eine Liste häufig verwendeter Begriffe
  mit jeweils einer kurzen Erklärung.  Begriff, die dort auftauchen, sind mit
  einem Pfeil gekennzeichnet, so wie hier \begriff{Funktion}.
\end{abstract}

\tableofcontents

\section{Allgemeines -- Begriffsklärung}

Unter einer \begriff{Funktion} versteht man eine \emph{eindeutige}
\begriff{Zuordnung}.  Zuordung bedeutet, dass einer Zahl aus einer Gruppe
(oder besser: Menge) eine Zahl aus einer zweiten Gruppe zugeordnet bekommt, so
dass Zahlenpaare entstehen.  Die erste Gruppe nennt man
\begriff{Definitionsmenge} $\menge{D}$ und eine unbekannte Zahl dieser Gruppe
bekommt das Symbol $x$.  Die zweite Gruppe nennt man \begriff{Wertemenge}
$\menge{W}$, eine Zahl dieser Gruppe bekommt das Symbol $y$ oder $f(x)$.  Eine
Tabelle, in der einige der Zahlenpaare (man nennt sie auch \begriff{Tupel})
aufgelistet sind, nennt man \begriff{Wertetabelle}.  Ein Beispiel dafür ist
Tabelle~\vref{tab:wertetabelle}.

\begin{table}
  \centering
  \caption{Beispiel für eine Wertetabelle.}
  \label{tab:wertetabelle}
  \begin{tabular}{>{$}l<{$}||*6{>{$}C{1cm}<{$}|}>{$}C{1cm}<{$}}
     x & -3 & -2 & -1 &  0 &  1 &  2 &  3  \\
    \hline
     y &  3 &  0 & -1 &  0 &  3 &  8 & 15
  \end{tabular}
\end{table}

Diese Zahlenpaare kann man in einem Koordinatensystem einzeichen.  Für die
Werte aus Tabelle~\vref{tab:wertetabelle} wurde das in
Abbildung~\vref{fig:wertepaare} gemacht.

\begin{figure}
  \centering
  \begin{minipage}[t]{.45\linewidth}
    \begin{tikzpicture}[scale=.5]
      \tkzInit[xmin=-5,xmax=5,ymin=-2,ymax=9]
      \tkzGrid
      \tkzAxeXY
      \tkzDefPoints{-3/3/a,-2/0/b,-1/-1/c,0/0/d,1/3/e,2/8/f}
      \tkzDrawPoints[color=red,shape=cross out](a,b,c,d,e,f)
    \end{tikzpicture}
    \subcaption{Die Wertepaare aus Tabelle~\vref{tab:wertetabelle} in einem
      Koordinatensystem.}
    \label{fig:wertepaare-punkte}
  \end{minipage}
  \hfill
  \begin{minipage}[t]{.45\linewidth}
    \begin{tikzpicture}[scale=.5]
      \tkzInit[xmin=-5,xmax=5,ymin=-2,ymax=9]
      \tkzGrid
      \tkzAxeXY
      \tkzDefPoints{-3/3/a,-2/0/b,-1/-1/c,0/0/d,1/3/e,2/8/f}
      \tkzDrawPoints[color=red,shape=cross out](a,b,c,d,e,f)
      \tkzFct[color=red]{(x+1)**2-1}
    \end{tikzpicture}
    \subcaption{Die Wertepaare aus Tabelle~\vref{tab:wertetabelle} liegen alle
      auf einer Linie.}
    \label{fig:wertepaare-linie}
  \end{minipage}
  \caption{Wertepaare einer Funktion bilden in der Regel einen Graphen aus
    einer Linie.}
  \label{fig:wertepaare}
\end{figure}

Bei einer \begriff{Funktion} liegen alle Punkte dann in der Regel auf einer
Linie, dem \begriffa{Graph} der \begriff{Funktion}.  In
Abbildung~\vref{fig:wertepaare-linie} wurde das für die Punkte aus
Tabelle~\vref{tab:wertetabelle} gemacht.

\begin{figure}
  \centering

\end{figure}

Für uns sind zwei Arten von Funktionen von Belang: zum einen die
\emph{linearen Funktionen} (siehe Abschnitt~\vref{sec:line-funkt-gerad}), die
hier nur ganz kurz wiederholt werden, und zum anderen die \emph{quadratischen
  Funktionen} (siehe Abschnitt~\vref{sec:quadr-funkt-parab}), um die es
hauptsächlich gehen wird.

Zuerst werden die Grundlagen der linearen Funktionen noch einmal wiederholt,
da sie die einfachste Sorte von Funktionen sind, und weil viele der
Prüfungsaufgaben zum Thema "`quadratische Funktionen"' auch lineare Funktionen
beinhalten.  Man sollte sie also auch kennen.

\section{Lineare Funktionen -- Geraden}\label{sec:line-funkt-gerad}
\subsection{Funktionsgleichung, Steigung und $y$-Achsenabschnitt}

Lineare Funktionen sind eine spezielle Sorte von Funktionen, bei denen alle
Wertepaare im Koordinatensystem auf einer geraden Linie liegen.  Man nennt sie
auch \begriffa{Gerade}.  Lineare Funktionen werden nach einer bestimmten
Vorschrift gebildet.  Ihre allgemeine Form lautet
\begin{equation}
  \label{eq:lineare-funktion}
  y = m\cdot x + c \quad.
\end{equation}
Man nennt $m$ die \begriff{Steigung} und $c$ den \begriff{y-Achsenabschnitt}.
Diese beiden Begriffe und Gleichung~\vref{eq:lineare-funktion} sollte man sich
unbedingt merken: sie werden unter Garantie prüfungsrelevant sein!

\begin{beispiel}\label{bsp:wertetabelle}
  Für die Gerade $y=2x+1$ kann man folgende \begriff{Wertetabelle} aufstellen:
  \begin{center}
    \begin{tabular}{>{$}l<{$}||*6{>{$}C{1cm}<{$}|}>{$}C{1cm}<{$}}
     x & -3 & -2 & -1 &  0 &  1 &  2 &  3  \\
    \hline
     y & -5 & -3 & -1 &  1 &  3 &  5 &  7
  \end{tabular}
  \end{center}
  Die $y$-Werte kann man jeweils ausrechnen, indem man die
  $x$-Werte\footnote{Sie werden zufällig ausgewählt; von $-3$ bis $3$ ist
    meistens ein guter Bereich.} in die Gleichung einsetzt.  Zum Beispiel:
  \begin{equation*}
    y = 2\cdot(-2)+1=-4+1=-3
  \end{equation*}
\end{beispiel}

Der \begriff{y-Achsenabschnitt} $c$ gibt den Wert an, an dem der
\begriff{Graph} die $y$-Achse schneidet.  Das ist das Zahlenpaar, dessen
$x$-Wert $0$ ist: $\punkt{S_y}(0,c)$.
\begin{equation}
  \label{eq:lin-funkt-achsenabschnitt}
  y = m \cdot 0 + c = c
\end{equation}

Die \begriff{Steigung} $m$ ist ein Maß dafür, wie steil oder flach der
\begriff{Graph} der \begriffa{Gerade} im Koordinatensystem liegt.  Man kann
sie bestimmen, indem man an den \begriffa{Graph} der \begriffa{Gerade} ein
rechtwinkliges Dreieck einzeichnet, dessen eine Kathete parallel zur $x$-Achse
ist und dessen zweite Kathete parallel zur $y$-Achse ist.  Die Größe und
Position des Dreiecks spielt dabei keine Rolle.  Der Quotient aus der
vertikalen Seite $\Delta y$ und der horizontalen Seite $\Delta x$ ergibt dann
$m$:
\begin{equation}
  \label{eq:lin-funkt-steigung}
  m = \frac{\Delta y}{\Delta x}
\end{equation}
Ein solches Dreieck ist in Abbildung~\vref{fig:lin-funkt-steigung} gezeigt.

\begin{figure}
  \centering
  \begin{tikzpicture}[scale=.75]
    \tkzInit[xmin=-2,xmax=5,ymin=-1,ymax=6]
    \tkzGrid
    \tkzAxeXY
    \tkzDefPoints{1/1/a,3/1/b,3/5/c}
    \tkzFillPolygon[color=green!10](a,b,c)
    \tkzDrawSegments[color=green](a,b b,c c,a)
    \tkzLabelSegment[below](a,b){$\Delta x$}
    \tkzLabelSegment[right](b,c){$\Delta y$}
    \tkzFct{2*x-1}
  \end{tikzpicture}
  \caption{Steigung einer Geraden.}
  \label{fig:lin-funkt-steigung}
\end{figure}

\subsection{Bestimmung einer Geraden aus zwei Punkten}
Hat man zwei Punkte $\punkt P(x_p,y_p)$ und $\punkt Q(x_q,y_q)$ gegeben, dann
ergibt die Differenz ihrer $y$-Werte $\Delta y$ und die Differenz ihrer
$x$-Werte $\Delta x$.  Damit lässt sich die \begriff{Steigung} dann auch
berechnen:
\begin{equation}
  \label{eq:steigung-aus-zwei-punkten}
  m = \frac{y_q-y_p}{x_q-x_p} \quad.
\end{equation}
Wenn man damit $m$ berechnet hat, kann man $m$ und die Werte eines der
gegebenen Punkte in die allgemeine Form (Gleichung~\vref{eq:lineare-funktion})
einsetzen und auch den \begriff{y-Achsenabschnitt} $c$ berechnen.  Dadurch
kann man eine \begriff{Funktionsgleichung} einer Geraden ohne Zeichnung
bestimmen.

\begin{beispiel}\label{bsp:steigung-berechnen}
  Eine Gerade geht durch die Punkte $\punkt A(-2,-3)$ und $\punkt B(1,3)$.
  Mit Gleichung~\vref{eq:steigung-aus-zwei-punkten} lässt sich nun die
  Steigung berechnen:
  \begin{equation*}
    m = \frac{3-(-3)}{1-(-2)}=\frac{6}{3}=2
  \end{equation*}
  Nun setzt man den Wert der Steigung und die Werte eines der
  Punkte\footnote{Die Werte eines Punktes entsprechen immer einem $x$-Wert und
    einem $y$-Wert (alphabetisch!).  Sie werden also für $x$ und $y$
    eingesetzt.} in Gleichung~\vref{eq:lineare-funktion} ein und berechnet $c$:
  \begin{align*}
    -3 &= 2\cdot(-2) +c \\
    -3 &= -4 +c && \auftrag{+4} \\
     1 &= c
  \end{align*}
  Damit ergibt sich die gesuchte Gleichung zu
  \begin{equation*}
    y = 2x+1 \quad.
  \end{equation*}
\end{beispiel}

\subsection{Steigungswinkel}
Jede Gerade hat auch einen \begriff{Steigungswinkel}.  Er ist in
Abbildung~\vref{fig:steigungswinkel} eingezeichnet.  Man kann dort auch
erkennen, dass der gleiche Winkel im Steigungsdreieck auftaucht (Stichwort
Stufenwinkel).  Da das Steigungsdreieck ein rechtwinkliges Dreieck ist, lässt
sich der \begriff{Steigungswinkel} mit dem Tangens\footnote{Zur Erinnerung:
  Gegenkathete durch Ankathete.} aus der Steigung berechnen.

\begin{figure}
  \centering
  \begin{tikzpicture}
    \tkzInit[xmin=-2,xmax=5,ymin=-1,ymax=6]
    \tkzGrid
    \tkzAxeXY
    \tkzDefPoints{.5/0/N,2/0/X,1.5/2/a,3/2/b,3/5/c}
    \tkzDrawSegments(a,b b,c c,a)
    \tkzLabelSegment[below](a,b){$\Delta x$}
    \tkzLabelSegment[right](b,c){$\Delta y$}
    \tkzMarkAngle[color=green,fill=green!10](X,N,a)
    \tkzLabelAngle[pos=.5](X,N,a){$\alpha$}
    \tkzMarkAngle(b,a,c)
    \tkzLabelAngle[pos=.5](b,a,c){$\alpha$}
    \tkzFct{2*x-1}
  \end{tikzpicture}
  \caption{Steigungswinkel von Geraden.}
  \label{fig:steigungswinkel}
\end{figure}

\begin{equation}
  \label{eq:steigungswinkel}
  \tan(\alpha) = \frac{\Delta y}{\Delta x} = m
  \quad \Leftrightarrow \quad
  \alpha = \arctan(m)
\end{equation}

$\arctan$ ist die "`echte"' Schreibweise von $\atan$, die heutzutage immer
weiter verbreitet ist, da Taschenrechner damit beschriftet sind.  Korrekt
nennt man das "`Arcustangens"'\footnote{Das entsprechende git übrigens auch
  für Sinus und Kosinus: $\asin\equiv\arcsin$ und $\acos\equiv\arccos$.}.  Wir
werden hier ab jetzt auch $\atan$ verwenden.

\begin{beispiel}\label{bsp:steigungswinkel}
  Der Steigungswinkel der Geraden $y=3x-1$ berechnet sich zu
  \begin{equation*}
    \alpha = \atan(3) = \ang{71.6} \quad.
  \end{equation*}
  Achtung: bei Geraden mit negativer Steigung spuckt der Taschenrechner einen
  negativen Winkel aus!  In diesem Fall muss man \ang{180} addieren.  Bei
  einer Geraden mit $m=-2$ ergibt sich mit
  \begin{equation*}
    \alpha' = \atan(-2) = \ang{-63.4}
  \end{equation*}
  ein negativer Winkel.  Der Steigungswinkel beträgt damit $\alpha=\ang{180} -
  \ang{63.4} = \ang{116.6}$.
\end{beispiel}

\section{Quadratische Funktionen -- Parabeln}\label{sec:quadr-funkt-parab}
\subsection{Normalparabel}

\subsection{Allgemeine Form}

\subsection{Scheitelform}

\section{Schnittpunkte mit den Koordinatenachsen, Nullstellen}

\section{Gemeinsame Punkte von Funktionen, Schnittpunkte}

\section{Bestimmung von Funktionsgleichungen}

\printacronyms

\end{document}
